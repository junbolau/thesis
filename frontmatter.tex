%
%
% UCSD Doctoral Dissertation Template
% -----------------------------------
% http://ucsd-thesis.googlecode.com
%
%


%% REQUIRED FIELDS -- Replace with the values appropriate to you

% No symbols, formulas, superscripts, or Greek letters are allowed
% in your title.
\title{$p$-adic Integration on Modular Curves and Code-Based Cryptography}

\author{Jun Bo Lau}
\degreeyear{\the\year}

% Master's Degree theses will NOT be formatted properly with this file.
\degreetitle{Doctor of Philosophy}

\field{Mathematics}

\chair{Professor Kiran Kedlaya}
% Uncomment the next line iff you have a Co-Chair
% \cochair{Professor Cochair Semimaster}
%
% Or, uncomment the next line iff you have two equal Co-Chairs.
%\cochairs{Professor Chair Masterish}{Professor Chair Masterish}

%  The rest of the committee members  must be alphabetized by last name.
\othermembers{
Professor Russell Impagliazzo\\
Professor Jonathan Novak\\
Professor Cristian Popescu\\
Professor Claus Sorensen
}
\numberofmembers{5} % |chair| + |cochair| + |othermembers|


%% START THE FRONTMATTER
%
\begin{frontmatter}

%% TITLE PAGES
%
%  This command generates the title, copyright, and signature pages.
%
\makefrontmatter

%% DEDICATION
%
%  You have three choices here:
%    1. Use the ``dedication'' environment.
%       Put in the text you want, and everything will be formated for
%       you. You'll get a perfectly respectable dedication page.
%
%
%    2. Use the ``mydedication'' environment.  If you don't like the
%       formatting of option 1, use this environment and format things
%       however you wish.
%
%    3. If you don't want a dedication, it's not required.
%
%
%\begin{dedication}
%  To two, the loneliest number since the number one.
%\end{dedication}


% \begin{mydedication} % You are responsible for formatting here.
%   \vspace{1in}
%   \begin{flushleft}
% 	To me.
%   \end{flushleft}
%
%   \vspace{2in}
%   \begin{center}
% 	And you.
%   \end{center}
%
%   \vspace{2in}
%   \begin{flushright}
% 	Which equals us.
%   \end{flushright}
% \end{mydedication}



%% EPIGRAPH
%
%  The same choices that applied to the dedication apply here.
%
%\begin{epigraph} % The style file will position the text for you.
%  \emph{A careful quotation\\
%  conveys brilliance.}\\
%  ---Smarty Pants
%\end{epigraph}

% \begin{myepigraph} % You position the text yourself.
%   \vfil
%   \begin{center}
%     {\bf Think! It ain't illegal yet.}
%
% 	\emph{---George Clinton}
%   \end{center}
% \end{myepigraph}


%% SETUP THE TABLE OF CONTENTS
%
\tableofcontents
%\listoffigures  % Comment if you don't have any figures
%\listoftables   % Comment if you don't have any tables



%% ACKNOWLEDGEMENTS
%
%  While technically optional, you probably have someone to thank.
%  Also, a paragraph acknowledging all coauthors and publishers (if
%  you have any) is required in the acknowledgements page and as the
%  last paragraph of text at the end of each respective chapter. See
%  the OGS Formatting Manual for more information.
%
\begin{acknowledgements}
I would like to express my heartfelt gratitude to my thesis advisor Kiran Kedlaya, for his invaluable guidance and support throughout my time at UCSD. His expertise and constructive feedback have been pivotal to my academic growth.

The first project started as a collaboration with Mingjie Chen and Kiran Kedlaya. I thank Kiran Kedlaya for suggesting the problem of computing Coleman integrals on modular curves using their instrinsic geometry; Mingjie Chen for many helpful discussions and code-debugging sessions; and Jennifer Balakrishnan, Pietro Mercuri and Samir Siksek for constructive conversations. The second project began as part of the Rethinking Number Theory workshops. I would like to thank the organisers Allechar Serrano López, Heidi Goodson and Mckenzie West for making this possible; my collaborators Tyler Raven Billingsley, Daniel Rayor Hast, Ray Perlner and Angela Robinson for many discussions that launched my interest in cryptography; Marco Baldi, Christine Kelley and Thomas Richardson for helpful feedbacks on our paper; and Lily Chen, Dustin Moody, Ray Perlner and Angela Robinson for hosting me at NIST for a research visit on this topic.

I am grateful to my parents for helping me to foster my passion for the subject, giving me many opportunities to grow and supporting me in my major decisions. I thank my partner Shu Ting for being my companion throughout this journey and for many years of love and support. You have been a source of comfort and motivation.

There are many friends who have made my academic journey memorable. I would like to thank Mingjie Chen, Woonam Lim, Zeyu Liu, Zongze Liu, Baiming Qiao, Nandagopal Ramachandran, Sindhana P.S., Shubham Sinha, Wei Yin, Minxin Zhang and many others. I would also like to thank the men's ultimate frisbee team for reigniting my love for the sport and being fun teammates at every occasion.

This thesis was written with support from NSF grant \todo{check}

Part 1 is, in full, being prepared for submission for publication. The dissertation author was the collaborator and the coauthor for the material below.

\begin{itemize}
\item Mingjie Chen, Kiran Kedlaya, Jun Bo Lau ``Coleman integration on modular curves"
\end{itemize}

Part 2 includes material that are in publication. The dissertation author was the collaborator and the coauthor for the material below.

\begin{itemize}
\item Sarah Arpin, Tyler Raven Billingsley, Daniel Rayor Hast, Jun Bo Lau, Ray Perlner, and
 Angela Robinson. A Study of Error Floor Behavior in QC-MDPC Codes. In Post-Quantum
 Cryptography: 13th International Workshop, PQCrypto 2022, Virtual Event, September
 28–30, 2022, Proceedings, page 89–103, Berlin, Heidelberg, 2022.
 Springer-Verlag.
\end{itemize}

\end{acknowledgements}


%% VITA
%
%  A brief vita is required in a doctoral thesis. See the OGS
%  Formatting Manual for more information.
%
\begin{vitapage}
\begin{vita}
  \item[2017] Master of Mathematics, University of Warwick, U.K.
  \item[2017-2023] Graduate Teaching Assistant, University of California San Diego, U.S.A.
  \item[2023] Doctor of Philosophy in Mathematics, University of California San Diego, U.S.A.
\end{vita}
\end{vitapage}


%% ABSTRACT
%
%  Doctoral dissertation abstracts should not exceed 350 words.
%   The abstract may continue to a second page if necessary.
%
\begin{abstract}

\begin{center}
$p$-adic Integration on Modular Curves and Code-Based Cryptography

\vspace{8mm}

by

\vspace{8mm}

Jun Bo Lau

\vspace{8mm}

Doctor of Philosophy in Mathematics

\vspace{3mm}

University of California San Diego, 2023

\vspace{3mm}

Professor Kiran Kedlaya, Chair

\end{center}

\vspace{5mm}

Falting's theorem states that there are only finitely many rational points on curves of genus greater than 1. An explicit determination of all such points on a curve remains a hard problem. There are various approaches to computing rational points on higher genus curves and we use Coleman's theory of $p$-adic line integrals to study modular curves. In join work with Chen and Kedlaya, we implement a new algorithm that does not use the models of the modular curves and illustrate this method through the computation of several examples.

In anticipation of the development of powerful quantum computers in the next few decades, we study cryptosystems that rely on the hard problems in coding theory. In joint work with Arpin, Bilingsley, Hast, Perlner and Robinson, we investigate BIKE, one of the candidates for the National Institute of Standards and Technology Post-Quantum Cryptography Standardization Process. We identified several factors that affect the security of the code-based cryptosystem through extensive simulations.

\vspace{8mm}


\end{abstract}


\end{frontmatter}
