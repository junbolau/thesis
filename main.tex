%
%
% UCSD Doctoral Dissertation Template
% -----------------------------------
% https://github.com/ucsd-thesis/ucsd-thesis
%
%   Lots of information can be found on the project wiki:
%   http://code.google.com/p/ucsd-thesis/wiki/GettingStarted
%
% HELP/CONTACT:
%
%   If you need help try the ucsd-thesis google group:
%   http://groups.google.com/group/ucsd-thesis
%
%
%
%
% ----------------------------------------------------------------------
% More control of the formatting of your thesis can be achieved through
% modifications of the included LaTeX class files:
%
%   * ucsd.cls    -- Class file
%   * uct10.clo   -- Configuration files for font sizes 10pt, 11pt, 12pt
%     uct11.clo                            
%     uct12.clo
%
% ----------------------------------------------------------------------



% Setup the documentclass 
% default options: 12pt, oneside, final
%
% fonts: 10pt, 11pt, 12pt -- are valid for UCSD dissertations.
% sides: oneside, twoside -- note that two-sided theses are not accepted 
%                            by OGS.
% mode: draft, final      -- draft mode switches to single spacing, 
%                            removes hyperlinks, and places a black box
%                            at every overfull hbox (check these before
%                            submission).
% chapterheads            -- Include this if you want your chapters to read:
%                              Chapter 1
%                              Title of Chapter
%
%                            instead of
%                              1 Title of Chapter
\documentclass[12pt,chapterheads]{ucsd}



% Include all packages you need here.  
% Some standard options are suggested below.
%
% See the project wiki for information on how to use 
% these packages. Other useful packages are also listed there.
%
%   http://code.google.com/p/ucsd-thesis/wiki/GettingStarted



%% AMS PACKAGES - Chances are you will want some or all 
%    of these if writing a dissertation that includes equations.
  \usepackage{amsmath, amscd, amssymb, amsthm,mathtools}

%% MARGIN REQUIREMENTS IN TITLES - Hyphenation in a Section Title does not always respect margin settings in Latex.  To force no hyphentation, uncomment the package below.
%  \usepackage[raggedright]{titlesec} 

%% GRAPHICX - This is the standard package for 
%    including graphics for latex/pdflatex.
\usepackage{scrextend}
\usepackage{pslatex}
\usepackage{graphicx}

\usepackage{fullpage,hyperref,tikz-cd}
\usepackage[all]{xy}
\usepackage[normalem]{ulem}


\newtheorem{theorem}{Theorem}[section]
\newtheorem{lemma}[theorem]{Lemma}
\newtheorem{cor}[theorem]{Corollary}
\newtheorem{problem}[theorem]{Problem}
\newtheorem{algorithm}[theorem]{Algorithm}
\theoremstyle{definition}
\newtheorem{defn}[theorem]{Definition}
\newtheorem{remark}[theorem]{Remark}
\newtheorem{conj}[theorem]{Conjecture}
\newtheorem{example}[theorem]{Example}
\newtheorem{proposition}[theorem]{Proposition}

\newcommand{\FF}{\mathbb{F}}
\newcommand{\QQ}{\mathbb{Q}}
\newcommand{\RR}{\mathbb{R}}
\newcommand{\ZZ}{\mathbb{Z}}
\newcommand{\HH}{\mathbb{H}}
\newcommand{\CC}{\mathbb{C}}

\newcommand{\frakp}{\mathfrak{p}}
\newcommand{\PP}{\mathbb{P}}

\newcommand{\calM}{\mathcal{M}}
\newcommand{\calO}{\mathcal{O}}
\newcommand{\calS}{\mathcal{S}}

\newcommand{\slz}{\SL_2(\ZZ)}


\DeclareMathOperator{\coker}{coker}
\DeclareMathOperator{\End}{End}
\DeclareMathOperator{\Ext}{Ext}
\DeclareMathOperator{\GL}{GL}
\DeclareMathOperator{\SL}{SL}
\DeclareMathOperator{\Gal}{Gal}
\DeclareMathOperator{\Gr}{Gr}
\DeclareMathOperator{\Hom}{Hom}
\DeclareMathOperator{\OG}{OG}
\DeclareMathOperator{\PGL}{PGL}
\DeclareMathOperator{\Proj}{Proj}
\DeclareMathOperator{\PSL}{PSL}
\DeclareMathOperator{\SO}{SO}
\DeclareMathOperator{\SpG}{SpG}
\DeclareMathOperator{\Stab}{Stab}
\DeclareMathOperator{\Sym}{Sym}
\DeclareMathOperator{\Trace}{Trace}
\DeclareMathOperator{\Det}{Det}
\DeclareMathOperator{\Div}{Div}


\newcommand{\todo}[1]{\textcolor{red}{\textbf{TODO: #1}}}

\newcommand{\Gap}{\textsc{GAP}}
\newcommand{\Magma}{\textsc{Magma}{}}
\newcommand{\SageMath}{\textsc{SageMath}}
\newcommand{\Singular}{\textsc{Singular}}

%% Links to arXiv
\newcommand{\arXiv}[3]{\href{https://arxiv.org/abs/#1}{arXiv:#1v#2} (#3)}

%% CAPTION
% This overrides some of the ugliness in ucsd.cls and
% allows the text to be double-spaced while letting figures,
% tables, and footnotes to be single-spaced--all OGS requirements.
% NOTE: Must appear after graphics and ams math
\makeatletter
\gdef\@ptsize{2}% 12pt documents
\let\@currsize\normalsize
\makeatother
\usepackage{setspace}
\doublespace
\usepackage[font=small, width=0.9\textwidth]{caption}

\usepackage[capposition=bottom]{floatrow} %force captions below figure per OGS requirement

%% SUBFIG - Use this to place multiple images in a
%    single figure.  Subfig will handle placement and
%    proper captioning (e.g. Figure 1.2(a))
% \usepackage{subfig}

%% TIMES FONT - replacements for Computer Modern
%%   This package will replace the default font with a
%%   Times-Roman font with math support.
 \usepackage[T1]{fontenc}
 \usepackage{mathptmx}



%% INDEX
%   Uncomment the following two lines to create an index: 
% \usepackage{makeidx}
% \makeindex
%   You will need to uncomment the \printindex line near the
%   bibliography to display the index.  Use the command
% \index{keyword} 
%   within the text to create an entry in the index for keyword.
%   To compile a LaTeX document with an index the 'makeindex'
%   command will need to be run.  See the wiki for more details.

%% HYPERLINKS
%   To create a PDF with hyperlinks, you need to include the hyperref package.
%   THIS HAS TO BE THE LAST PACKAGE INCLUDED!
%   Note that the options plainpages=false and pdfpagelabels exist
%   to fix indexing associated with having both (ii) and (2) as pages.
%   Also, all links must be black according to OGS.
%   See: http://www.tex.ac.uk/cgi-bin/texfaq2html?label=hyperdupdest
%   Note: This may not work correctly with all DVI viewers (i.e. Yap breaks).
%   NOTE: hyperref will NOT work in draft mode, as noted above.
% \usepackage[colorlinks=true, pdfstartview=FitV, 
%             linkcolor=black, citecolor=black, 
%             urlcolor=black, plainpages=false,
%             pdfpagelabels]{hyperref}
% \hypersetup{ pdfauthor = {Your Name Here}, 
%              pdftitle = {The Title of The Dissertation}, 
%              pdfkeywords = {Keywords for Searching}, 
%              pdfcreator = {pdfLaTeX with hyperref package}, 
%              pdfproducer = {pdfLaTeX} }
% \urlstyle{same}
% \usepackage{bookmark}


%% CITATIONS
% Sets citation format
% and fixes up citations madness
\usepackage{microtype}  % avoids citations that hang into the margin


%% FOOTNOTE-MAGIC
% Enables footnotes in tables, re-referencing the same footnote multiple times.
\usepackage{footnote}
\makesavenoteenv{tabular}
\makesavenoteenv{table}


%% TABLE FORMATTING MADNESS
% Enable all sorts of fun table tricks
\usepackage{rotating}  % Enables the sideways environment (NCPW)
\usepackage{array}  % Enables "m" tabular environment http://ctan.org/pkg/array
\usepackage{booktabs}  % Enables \toprule  http://ctan.org/pkg/array



\begin{document}

%% FRONT MATTER
%
%  All of the front matter.
%  This includes the title, degree, dedication, vita, abstract, etc..
%  Modify the file template_frontmatter.tex to change these pages.
%
%
% UCSD Doctoral Dissertation Template
% -----------------------------------
% http://ucsd-thesis.googlecode.com
%
%


%% REQUIRED FIELDS -- Replace with the values appropriate to you

% No symbols, formulas, superscripts, or Greek letters are allowed
% in your title.
\title{$p$-adic Integration on Modular Curves and Code-Based Cryptography}

\author{Jun Bo Lau}
\degreeyear{\the\year}

% Master's Degree theses will NOT be formatted properly with this file.
\degreetitle{Doctor of Philosophy}

\field{Mathematics}

\chair{Professor Kiran Kedlaya}
% Uncomment the next line iff you have a Co-Chair
% \cochair{Professor Cochair Semimaster}
%
% Or, uncomment the next line iff you have two equal Co-Chairs.
%\cochairs{Professor Chair Masterish}{Professor Chair Masterish}

%  The rest of the committee members  must be alphabetized by last name.
\othermembers{
Professor Russell Impagliazzo\\
Professor Jonathan Novak\\
Professor Cristian Popescu\\
Professor Claus Sorensen
}
\numberofmembers{5} % |chair| + |cochair| + |othermembers|


%% START THE FRONTMATTER
%
\begin{frontmatter}

%% TITLE PAGES
%
%  This command generates the title, copyright, and signature pages.
%
\makefrontmatter

%% DEDICATION
%
%  You have three choices here:
%    1. Use the ``dedication'' environment.
%       Put in the text you want, and everything will be formated for
%       you. You'll get a perfectly respectable dedication page.
%
%
%    2. Use the ``mydedication'' environment.  If you don't like the
%       formatting of option 1, use this environment and format things
%       however you wish.
%
%    3. If you don't want a dedication, it's not required.
%
%
%\begin{dedication}
%  To two, the loneliest number since the number one.
%\end{dedication}


% \begin{mydedication} % You are responsible for formatting here.
%   \vspace{1in}
%   \begin{flushleft}
% 	To me.
%   \end{flushleft}
%
%   \vspace{2in}
%   \begin{center}
% 	And you.
%   \end{center}
%
%   \vspace{2in}
%   \begin{flushright}
% 	Which equals us.
%   \end{flushright}
% \end{mydedication}



%% EPIGRAPH
%
%  The same choices that applied to the dedication apply here.
%
%\begin{epigraph} % The style file will position the text for you.
%  \emph{A careful quotation\\
%  conveys brilliance.}\\
%  ---Smarty Pants
%\end{epigraph}

% \begin{myepigraph} % You position the text yourself.
%   \vfil
%   \begin{center}
%     {\bf Think! It ain't illegal yet.}
%
% 	\emph{---George Clinton}
%   \end{center}
% \end{myepigraph}


%% SETUP THE TABLE OF CONTENTS
%
\tableofcontents
%\listoffigures  % Comment if you don't have any figures
%\listoftables   % Comment if you don't have any tables



%% ACKNOWLEDGEMENTS
%
%  While technically optional, you probably have someone to thank.
%  Also, a paragraph acknowledging all coauthors and publishers (if
%  you have any) is required in the acknowledgements page and as the
%  last paragraph of text at the end of each respective chapter. See
%  the OGS Formatting Manual for more information.
%
\begin{acknowledgements}
I would like to express my heartfelt gratitude to my thesis advisor Kiran Kedlaya, for his invaluable guidance and support throughout my time at UCSD. His expertise and constructive feedback have been pivotal to my academic growth.

The first project started as a collaboration with Mingjie Chen and Kiran Kedlaya. I thank Kiran Kedlaya for suggesting the problem of computing Coleman integrals on modular curves using their instrinsic geometry; Mingjie Chen for many helpful discussions and code-debugging sessions; and Jennifer Balakrishnan, Pietro Mercuri and Samir Siksek for constructive conversations. The second project began as part of the Rethinking Number Theory workshops. I would like to thank the organisers Allechar Serrano López, Heidi Goodson and Mckenzie West for making this possible; my collaborators Sarah Arpin, Tyler Raven Billingsley, Daniel Rayor Hast, Ray Perlner and Angela Robinson for many discussions that launched my interest in cryptography; Marco Baldi, Christine Kelley and Thomas Richardson for helpful feedbacks on our paper; and Lily Chen, Dustin Moody, Ray Perlner and Angela Robinson for hosting me at NIST for a research visit on this topic.

I am grateful to my parents for helping me to foster my passion for the subject, giving me many opportunities to grow and supporting me in my major decisions. I thank my partner Shu Ting for being my companion throughout this journey and for many years of love and support. You have been a source of comfort and motivation.

There are many friends who have made my academic journey memorable. I would like to thank Mingjie, Woonam, Zeyu, Zongze, Baiming, Nandagopal, Sindhana, Shubham, Wei, Minxin and many others. I would also like to thank the men's ultimate frisbee team for reigniting my love for the sport and being fun teammates at every occasion.

This thesis was written with support from the NSF grants DMS-1844206 and DMS-1802161.

Part 1 is, in full, being prepared for submission for publication and is included in one of the collaborator's thesis. The dissertation author was the collaborator and the coauthor for the material in the first bullet point.

\begin{itemize}
\item Mingjie Chen, Kiran Kedlaya, Jun Bo Lau ``Coleman Integration on Modular Curves"
\item Mingjie Chen. Arithmetic of Algebraic Curves. UC San Diego. ProQuest ID: \texttt{Chen\_ucsd\_0033D\_21000}. Merritt ID: \texttt{ark:/13030/m5qz9cpg}. Retrieved from \href{https://escholarship.org/uc/item/2v180848}{https://escholarship.org/uc/item/2v180848}
\end{itemize}

Part 2 includes material that are in publication. The dissertation author was the collaborator and the coauthor for the material below.

\begin{itemize}
\item Sarah Arpin, Tyler Raven Billingsley, Daniel Rayor Hast, Jun Bo Lau, Ray Perlner, and
 Angela Robinson. A Study of Error Floor Behavior in QC-MDPC Codes. In Post-Quantum
 Cryptography: 13th International Workshop, PQCrypto 2022, Virtual Event, September
 28–30, 2022, Proceedings, page 89–103, Berlin, Heidelberg, 2022.
 Springer-Verlag.
\end{itemize}

\end{acknowledgements}


%% VITA
%
%  A brief vita is required in a doctoral thesis. See the OGS
%  Formatting Manual for more information.
%
\begin{vitapage}
\begin{vita}
  \item[2017] Master of Mathematics, University of Warwick, U.K.
  \item[2017-2023] Graduate Teaching Assistant, University of California San Diego, U.S.A.
  \item[2023] Doctor of Philosophy in Mathematics, University of California San Diego, U.S.A.
\end{vita}
\end{vitapage}


%% ABSTRACT
%
%  Doctoral dissertation abstracts should not exceed 350 words.
%   The abstract may continue to a second page if necessary.
%
\begin{abstract}

\begin{center}
$p$-adic Integration on Modular Curves and Code-Based Cryptography

\vspace{8mm}

by

\vspace{8mm}

Jun Bo Lau

\vspace{8mm}

Doctor of Philosophy in Mathematics

\vspace{3mm}

University of California San Diego, 2023

\vspace{3mm}

Professor Kiran Kedlaya, Chair

\end{center}

\vspace{5mm}

Falting's theorem states that there are only finitely many rational points on curves of genus greater than 1. An explicit determination of all such points on a curve remains a hard problem. There are various approaches to computing rational points on higher genus curves and we use Coleman's theory of $p$-adic line integrals to study modular curves. In join work with Chen and Kedlaya, we implement a new algorithm that does not use the models of the modular curves and illustrate this method through the computation of several examples.

In anticipation of the development of powerful quantum computers in the next few decades, we study cryptosystems that rely on the hard problems in coding theory. In joint work with Arpin, Bilingsley, Hast, Perlner and Robinson, we investigate BIKE, one of the candidates for the National Institute of Standards and Technology Post-Quantum Cryptography Standardization Process. We identified several factors that affect the security of the code-based cryptosystem through extensive simulations.

\vspace{8mm}


\end{abstract}


\end{frontmatter}






%% DISSERTATION

% A common strategy here is to include files for each of the chapters. I.e.,
% Place the chapters is separate files: 
%   chapter1.tex, chapter2.tex
% Then use the commands:
%   \include{chapter1}
%   \include{chapter2}
%
% Of course, if you prefer, you can just start with
%   \chapter{My First Chapter Name}
% and start typing away.  

%%%%%%%%%%%%%%%%%%%%%% \include{sample_unused}

\part{Coleman Integration on Modular Curves}
\chapter{Preliminaries}

All curves in this paper are smooth, projective and geometrically irreducible with good reduction at a prime $p$.

\section{Introduction}

Some of the oldest questions in number theory can be reformulated in modern terms: given a finite list of polynomials, what are the integer or rational solutions to this set of equations? In fact, these solutions can be viewed as integer or rational solutions of geometric objects -- curves, surfaces or higher dimensional objects.

In this project, we focus on the case of curves. A remarkable result, formulated by Mordell in 1922 and proved by Faltings in 1983, states that for curves of higher genus, there are only finitely many rational points on them. 

\begin{theorem}{(Mordell's conjecture/ Faltings's theorem)} Let $X/\QQ$ be a curve of genus $g \geq 2$, then the set of rational points $X(\QQ)$ is finite.
\end{theorem}

However, Faltings's proofs are not effective, i.e., there is no way of explicitly determining the complete set of rational points on the curve. Before Faltings, Chabauty developed a method in this direction with the condition that if the rank of the Jacobian of the curve is strictly less than the genus, then one could compute this set of points. In \cite{Coleman2,Coleman3} Coleman defined $p$-adic line integrals and re-interpreted Chabauty's method to explicitly compute the set of rational points. These Coleman integrals provide an effective method to problems in arithmetic geometry, including but not limited to, torsion points on Jacobians of curves ( Manin-Mumford conjecture), $p$-adic heights on curves, $p$-adic polylogarithms, Mordell conjecture (rational points), etc. In \cite{BD1,BD2}, Balakrishnan and Dogra developed quadratic Chabauty as a computational tool to study the set of rational points as long as the curve satisfies a certain quadratic Chabauty bound, involving the rank of the Jacobian, genus and N\'{e}ron-Severi rank of the Jacobian.

There are several approaches to numerically compute these Coleman integrals. Wetherell \cite{wetherell} combined the certan properties of Coleman integrals and the arithmetic of the Jacobian to compute $\int_D \omega$, where $D$ is a divisor in the Picard group and $\omega$ is a holomorphic differential on the curve. The next approach relies on computing the Frobenius action in $p$-adic cohomology following Dwork's principle of analytic continuation along the Frobenius \cite{BBK10,Tui16,Tui17,BT_coleman}. However, both of these approaches have their shortcomings -- Wetherell's method requires an explicit divisor in order to reduce the computation to a power series integration (``tiny integrals") and the second method requires an explicit equation of the curves as input.

We turn our attention to computing Coleman integrals on modular curves. The set of rational points on modular curves has special arithmetic meaning. For instance, the set of rational points $X_0(N)(\QQ)$ correspond to the torsion points of elliptic curves (Mazur's theorem). Another motivation to study modular curves comes from Serre's Uniformity Conjecture. Let $E$ be an elliptic curve defined over $K$. The group of $p$-torsion points $E[p](\bar{K})$ is isomorphic to $(\ZZ/p\ZZ)^2$ and is acted upon by the absolute Galois group $\Gal(\bar{K}/K)$, giving rise to a representation $\rho_ {p,E}: \Gal(\bar{K}/K) \rightarrow \GL_2(\FF_p)$. In \cite{serre72}, Serre proved the following:

\begin{theorem}
    Suppose that $E$ does not have complex multiplication. Then there exists a number $N(E)$ such that $\rho_{p,E}$ is surjective for all $p > N(E)$.
\end{theorem}

In the same paper, he posed the following question:

\begin{conj}{(Serre's Uniformity Conjecture)}
Given a number field $K$, then there exist a constant $N_K>0$ such that for any elliptic curve $E$ defined over $K$ without complex multiplication, the corresponding Galois representation $\rho_{p,E}$ is surjective for all primes $p > N_K$.
\end{conj}

Since modular curves parametrise elliptic curves with torsion data, this can be formulated in terms of rational points on modular curves:

\begin{conj}{(Serre's Uniformity Conjecture)}
    Let $H \leq \GL_2(\FF_p)$ be a proper subgroup such that the determinant map $\det: H \rightarrow \FF_p^\times$ is surjective, then there exist a constant $N_K>0$ such that for any prime $p> N_K$, the associated modular curve $X_H(p)$ has $K$-rational points coming only from cusps and elliptic curves with complex multiplication.
\end{conj}




If $\rho_{p,E}$ is not surjective, the image lies inside some maximal proper subgroup of $\GL_2(\FF_p)$. Therefore, one could prove the conjecture by showing that for $p$ large enough, the image of $\rho_{p,E}$ does not lie in any maximal subgroup. The classification of maximal subgroups of $\GL_2(\FF_p)$ is known, originally due to \cite{dickson}:

\begin{theorem}
    Let $H \leq \GL_2(\ZZ/p\ZZ)$ not containing $\SL_2(\ZZ/p\ZZ)$. Up to conjugacy, $H$ is one of the following:
    \begin{itemize}
        \item (Borel) $H \subseteq B_0(p) = \{ \begin{psmallmatrix} \ast \ \ast \\ 0 \ \ast \end{psmallmatrix} \}$ 
        \item (Normaliser of split Cartan) $H \subseteq N_s^+(p) = \{ \begin{psmallmatrix} \alpha \ 0 \\ 0 \ \beta \end{psmallmatrix}, \begin{psmallmatrix} 0 \ \alpha \\ \beta \ 0 \end{psmallmatrix}: \alpha,\beta \in \FF_p^\times \}$ 
        \item (Normaliser of non-split Cartan) $H \subseteq N_s^+(p) = \{ \begin{psmallmatrix} \alpha \ 0 \\ 0 \ \alpha^p \end{psmallmatrix}, \begin{psmallmatrix} 0 \ \alpha \\ \alpha^p \ 0 \end{psmallmatrix}: \alpha \in \FF_{p^2}^\times \}$ 
        \item (Exceptional) The image of $H$ in $\PGL_2(\FF_p)$ is isomorphic to $A_4,S_4$ or $A_5$.
    \end{itemize}
\end{theorem}

Most of the cases have been resolved \cite{borel,BP1,BP2,serre72}, except for the normaliser of non-split Cartan. There has been some progress using quadratic Chabauty to find the rational points of the modular curve corresponding to the nonsplit Cartan of level 13 \cite{cursed-curve} and level 17 \cite{BDMTV}.

Since most modular curves satisfy the quadratic Chabauty bound \cite{Siksek}, we provide a model-free algorithm to compute Coleman integrals on modular curves arising  arising from Serre's Uniformity Conjecture.

\section{Background}

\subsection{Modular forms}

In this section, we give a brief introduction of modular forms, following \cite{Shurman}.

Let $\HH := \{ \tau \in \CC: Im(\tau) > 0 \}$ be the upper half complex plane. The special linear group $\SL_2(\ZZ)$ acts on $\HH$ via fractional linear transformations:

\[
\gamma \cdot \tau = \frac{a\tau + b}{c \tau + d}
\] 

where $\gamma = \begin{psmallmatrix} a & b \\ c & d \end{psmallmatrix}, \tau \in \HH$. 

\begin{defn}
Let $f: \HH \rightarrow \CC$ be a function and $k \in \ZZ$. 

\begin{itemize}
    \item The \textit{automorphy factor} is a function \begin{align*}j: \GL_2^+(\RR) \times \HH &\rightarrow \CC \\ (\gamma,z) &\mapsto cz+d \end{align*}
    where $\gamma = \begin{psmallmatrix} a & b \\ c & d \end{psmallmatrix}$.
    \item The \textit{weight-$k$ slash operator} is defined as 
    \begin{align*}
( \ \cdot \ )|_k (\ \cdot \ ): \Hom(\HH,\CC) \times \GL_2^+(\RR) &\rightarrow \Hom(\HH,\CC) \\
(f(z),\gamma) &\mapsto (f|_k \gamma)(z) := (\det \gamma)^{k-1} j(\gamma,z)^{-k} f(\gamma \cdot z).
    \end{align*}
    \end{itemize}
\end{defn}

The automorphy factory satisfies a cocycle relation $j(\gamma_1\gamma2,z) = j(\gamma_1,\gamma_2 z)j(\gamma_2,z)$ which implies that $\GL_2^+(\RR)$ acts on $\Hom(\HH,\CC)$ via $f|_k\gamma_1\gamma_2 = (f|_k\gamma_1)|_k\gamma_2$.

Consider the projection map $\pi: \SL_2(\ZZ) \rightarrow \SL_2(\ZZ/N\ZZ)$. We define congruence subgroups in the following way.

\begin{example}\label{example:1_congsgp}
Here are some common examples of preimages of $\pi$:

\begin{itemize}
    \item $\Gamma(N) = \pi^{-1} (\begin{psmallmatrix}
        1 & 0 \\ 0 & 1
    \end{psmallmatrix}) = \{ \begin{psmallmatrix}
        a & b \\ c & d 
    \end{psmallmatrix} \in \SL_2(\ZZ): \begin{psmallmatrix}
        a & b \\ c & d 
    \end{psmallmatrix} \equiv \begin{psmallmatrix}
        1 & 0 \\ 0 & 1 
    \end{psmallmatrix} \pmod{N}\}$.
    \item $\Gamma_1(N) = \pi^{-1} (\{\begin{psmallmatrix}
        1 & \ast \\ 0 & 1
    \end{psmallmatrix}\}) = \{ \begin{psmallmatrix}
        a & b \\ c & d 
    \end{psmallmatrix} \in \SL_2(\ZZ): \begin{psmallmatrix}
        a & b \\ c & d 
    \end{psmallmatrix} \equiv \begin{psmallmatrix}
        1 & \ast \\ 0 & 1 
    \end{psmallmatrix} \pmod{N}\}$.
    \item $\Gamma_0(N) = \pi^{-1} (\{\begin{psmallmatrix}
        \ast & \ast \\ 0 & \ast
    \end{psmallmatrix}\}) = \{ \begin{psmallmatrix}
        a & b \\ c & d 
    \end{psmallmatrix} \in \SL_2(\ZZ): \begin{psmallmatrix}
        a & b \\ c & d 
    \end{psmallmatrix} \equiv \begin{psmallmatrix}
        \ast & \ast \\ 0 & \ast 
    \end{psmallmatrix} \pmod{N}\}$.
\end{itemize}
\end{example}

\begin{defn}
$\Gamma \leq \SL_2(\ZZ)$ is a \textit{congruence subgroup} if there exists an integer $N \geq 1$ such that $\Gamma(N) \leq \Gamma$. The minimal such $N$ is called the \textit{level} of $\Gamma$.

It follows immediately that congruence subgroups of $\SL_2(\ZZ)$ have finite index and correspond to subgroups of $\SL_2(\ZZ/N\ZZ)$. The above examples are all congruence subgroups of level $N$.
\end{defn}

\begin{defn}
    Let $\Gamma \leq \SL_2(\ZZ)$ be a congruence subgroup of level $N$, $k \geq 0$ an integer. We say a function $f: \HH \rightarrow \CC$ is a \textit{modular form of weight $k$ with level $\Gamma$} if

    \begin{enumerate}
        \item $f$ is holomorphic,
        \item $f$ is weight-$k$ invariant under $\Gamma$, i.e., $f|_k\gamma = f$ for all $\gamma \in \Gamma$,
        \item $f|_k\alpha $ is holomorphic at $\infty$ for all $\alpha \in \SL_2(\ZZ)$, i.e., $(f|_k \alpha )(z)$ is bounded as $z \rightarrow i\infty$.
    \end{enumerate}

    If, in addition, $f|_k \alpha$ vanishes at infinity for all $\alpha \in \SL_2(\ZZ)$, we say that $f$ is a \textit{cusp form}. We denote the set of weight-$k$ modular forms with respect to $\Gamma$ (resp. cusp forms) as $\mathcal{M}_k(\Gamma)$ (resp. $\mathcal{S}_k(\Gamma)$).
\end{defn}

Suppose $f$ is a modular form of weight $k$ with level $\Gamma$. Since $\Gamma$ is a congruence subgroup, $\begin{psmallmatrix}
    1 & h \\ 0 & 1
\end{psmallmatrix} \in \Gamma$ for some minimal integer $h\geq 1$, this integer is the \textit{width} of the cusp $\infty$. Since a modular form satisfies $f|_k \gamma = f$ for $\gamma \in \Gamma$, we have $(f|_k \begin{psmallmatrix}
    1 & h \\ 0 & 1
\end{psmallmatrix})(z) = f(z+h) = f(z) $, so $f(z)$ is $h\ZZ$-periodic and admits a Fourier expansion $f(\tau) = \sum_{n=0}^\infty a_n q_{h}^n$ where $q_h = \exp(2\pi i\tau / h)$. The third condition of modular forms implies that the Fourier expansion begins at index $0$ and cusp forms satisfy $a_0 = 0$.

\begin{example}
Let $G_k(\tau) = \sum_{(c,d) \not = (0,0)} 1/(c\tau + d)^k$. This is a modular form of weight $k$ for $\SL_2(\ZZ)$ called \textit{Eisenstein series}. 

    The \textit{$j$-invariant} is a modular form of weight $0$, i.e., a modular function and an element of $\CC(X(\SL_2(\ZZ)))$, with $q$-expansion:

    \begin{align*}
        j: \HH \rightarrow \CC, j(\tau) = 1728 \frac{(60 G_4(\tau))^3}{(60 G_4(\tau))^3 - 27 (140 G_6(\tau))^2} = \frac{1}{q} + 744 + 196884q + \ldots.
    \end{align*}
\end{example}

It is a standard result that $\mathcal{M}_k(\Gamma) \supseteq \mathcal{S}_k(\Gamma)$ are finite dimensional complex vector spaces. Modular forms and modular curves are related by the fact that there is an isomorphism between the space of weight $2$ cusp forms and the space of holomorphic differentials on the modular curve $X(\Gamma)$. 

\begin{align*}
\mathcal{S}_2(\Gamma) &\xrightarrow{\cong} H^0(X(\Gamma),\Omega^1) \\
f(\tau) &\mapsto f(\tau) d\tau.
\end{align*}



\subsection{Modular curves}
In this section, we define our object of study. Modular curves have rich structures as Riemann surfaces, algebraic curves and moduli spaces of elliptic curves (with some torsion information). We frequently use properties from various perspectives interchangeably.

\subsubsection{As Riemann surfaces}
Let $\Gamma \leq \SL_2(\ZZ)$ be a subgroup of finite index. $\HH$ inherits the Euclidean topology from $\CC$ and so $Y(\Gamma) := \Gamma \backslash \HH$ carries the quotient topology that is Hausdorff. $Y(\Gamma)$ can be compactified by adjoining cusps, which are orbits of $\PP^1(\QQ)$ under the action of $\Gamma$. The resulting quotient space $X(\Gamma) := \Gamma \backslash \HH^*$ where $\HH^* := \HH \cup \PP^1(\QQ)$ is called the modular curve associated to $\Gamma$. One could further show that by considering elliptic points and cusps, one can choose suitable charts, therefore giving $Y(\Gamma)$ and $X(\Gamma)$ the structure of Riemann surface.

This approach allows us to use techniques from Riemann surfaces, e.g., genus/ramification theory, Riemann-Hurwitz formula, Riemann-Roch, etc. to study modular curves.

\subsubsection{As algebraic curves}

For a finite index subgroup $\Gamma \leq \SL_2(\ZZ)$. The associated modular curve $X(\Gamma)$ has the structure of a compact Riemann surface. Compact Riemann surfaces and complex algebraic curves are equivalent notions \cite{forster}. Note that we are also considering modular curves where the determinant map on the subgroup $H \leq \GL_2(\ZZ/N\ZZ)$ is surjective. By Theorem 7.6.3 in \cite{Shurman}, these algebraic curves are in fact defined over $\QQ$. We have a Galois-theoretic correspondence between curves and their function fields:

\begin{theorem}{(Curves-Fields Correspondence)} For any field $k$, there is a bijection:

\begin{align*}
\{\text{$C/k$ smooth projective algebraic curves}\}/\cong &\leftrightarrow \{\text{$K/k$ function field extensions over $k$}\}/\sim \\
C &\mapsto k(C)
\end{align*}

Furthermore, this is contravariant: a nonconstant morphism from  algebraic curves $C$ to $C'$ over $k$ corresponds to a field morphism from $k(C')$ to $k(C)$.

\end{theorem}

The above theorem allows us to work with simpler objects, i.e., we can replace curves and their morphisms by fields and field injections. In particular, the function field of the modular curve $X(\Gamma)$ consists of modular functions of weight $0$ and level $\Gamma$. 

\subsubsection{As moduli spaces of elliptic curves}

For each $\tau \in \HH$, one could associate it with a lattice $\Lambda_\tau := \ZZ + \tau \cdot \ZZ \subseteq \CC$. The resulting quotient space $\CC/\Lambda_\tau$ is a compact Riemann surface of genus $1$, an elliptic curve. Conversely, for any elliptic curve, as a genus 1 compact Riemann surface, the homology group of the elliptic curve $H_1(E,\ZZ)$ is generated by two loops, $\gamma_1, \gamma_2$. For an invariant differential $\omega$ of the elliptic curve, we can construct the lattice generated by the periods $\Lambda_E = (\int_{\gamma_1} \omega) \cdot \ZZ + (\int_{\gamma_2} \omega) \cdot \ZZ$. This can be renormalised so that $\Lambda_E = \ZZ + \tau \cdot \ZZ$ with $\tau = (\int_{\gamma_1} \omega) /(\int_{\gamma_2} \omega) \in \HH$. In particular the points on $\HH$ correspond to elliptic curves.

For $\Gamma \leq \SL_2(\ZZ)$, the modular curve $X(\Gamma)(\bar{\QQ})$ parametrise elliptic curves with some torsion data, i.e., a pair $(E, \phi)$ where $E$ is an elliptic curve defined over $\bar{\QQ}$ and $\phi$ is an isomorphism of its $N$-torsion points $\phi: E[N] \rightarrow (\ZZ/N\ZZ)^2$. Two points $(E_1,\phi_1), (E_2,\phi_2)$ are isomorphic if there is an isomorphism of elliptic curves $\psi: E_1 \rightarrow E_2$ and some matrix $M \in \Gamma$ such that the diagram commutes:

\begin{center}
  \begin{tikzcd}
    E_1[N] \arrow[r, "\phi_1"] \arrow[d, "\psi"] & (\ZZ/N\ZZ)^2 \arrow[d,  "M"] \\
    E_2[N] \arrow[r, "\phi_2"]                &  (\ZZ/N\ZZ)^2.                               
  \end{tikzcd}
\end{center}


Furthermore, there is an action of the absolute Galois group $Gal(\bar{\QQ}/\QQ)$ on $(E,\phi)$ and we say that $(E,\phi)$ is a $\QQ$-rational point if it is invariant under the action. We can view points on modular curves as elliptic curves with certain torsion structures which allows us to apply properties of elliptic curves to study the rational points on $X(\Gamma)$.

\begin{example}
Let $H \leq \GL_2(\ZZ/N\ZZ)$ be a subgroup such that
\begin{itemize}
    \item $-I \in H$,
    \item the determinant map $\det: H \rightarrow (\ZZ/N\ZZ)^\times$ is surjective.
\end{itemize}
Then for an integer $N \geq 1$, we have the congruence subgroup $\Gamma_H(N) = \{ A \in \SL_2(\ZZ) : A \pmod{N} \in H \}$, which gives rise to the modular curves $X_H := X(\Gamma_H(N))$.

Following Example \ref{example:1_congsgp}, the corresponding modular curves parametrise:

\begin{itemize}
    \item $X(N):= X(\Gamma(N))$ consists of $(E,(P,Q))$ an elliptic curve and a pair of points generating the $N$-torsion subgroup of $E$.
    \item $X_1(N) := X(\Gamma_1(N))$ consists of $(E,Q)$ an elliptic curve and a point of order $N$.
    \item $X_0(N) := X(\Gamma_0(N))$ consists of $(E,C)$ an elliptic curve and a cyclic subgroup of order $N$.
\end{itemize}
\end{example}

\subsection{Hecke operators}

We begin with the definition of Hecke operators as operators on spaces of modular forms. These are used in conjunction with spectral theory to show that the inner product space of modular forms contains a basis of modular forms that are eigenvectors under the Hecke operators $\{T_p\}_p$. Hecke operators are defined on modular forms and modular curves. We use both the transcendental and algebraic/geometric definitions of Hecke operators in our algorithm.

\begin{defn}
Let $\Gamma_1, \Gamma_2$ be congruence subgroups of $SL_2(\ZZ)$ and $\alpha \in GL_2^+(\QQ)$. 

\begin{itemize}
    \item We define the \textit{double coset} $\Gamma_1 \alpha \Gamma_2$ as the set 

\[
\Gamma_1 \alpha \Gamma_2 := \{\gamma_1 \alpha \gamma_2 : \gamma_1 \in \Gamma_1, \gamma_2 \in \Gamma_2\}
\]
\item This gives rise to the \textit{double coset operators}:

\begin{align*}
(\ \cdot \ )|_k [\Gamma_1 \alpha \Gamma_2] : \mathcal{M}_k(\Gamma_1) &\rightarrow \mathcal{M}_k(\Gamma_2) \\
f(\tau) &\mapsto f|_k \Gamma_1 \alpha \Gamma_2 := \sum_i f|_k \beta_i
\end{align*}

where $\Gamma_1 \alpha \Gamma_2 = \bigcup_i \Gamma_1 \beta_i$ is a (finite) disjoint coset decomposition that does not depend on the choice of decomposition. This map restricts to an operator on the space of cusp forms $(\ \cdot \ )|_k [\Gamma_1 \alpha \Gamma_2] : \mathcal{S}_k(\Gamma_1) \rightarrow \mathcal{S}_k(\Gamma_2)$.
\end{itemize}
\end{defn}

We follow the approach \cite{Assaf2020} to define Hecke operators.

\begin{defn}
Fix a congruence subgroup $\Gamma$ with $\bar{\Gamma} \leq SL_2(\ZZ/N\ZZ)$.
Let $\alpha \in M_2(\ZZ)$ such that $\det (\alpha) \in \det (\bar{\Gamma})$ and $\alpha \pmod{N} \in \bar{\Gamma}$. We define the Hecke operator as

\[
T_p = T_\alpha = ( \ \cdot \ )|_k [\Gamma \alpha \Gamma]
\]
\end{defn}

\begin{example} (\cite{Shurman} Prop. 5.2.1)
The theory of Hecke operators can be made explicit for certain congruence subgroups. The Hecke operator $T_p = [ \Gamma_1(N) \begin{psmallmatrix} 1 & 0 \\ 0 & p \end{psmallmatrix} \Gamma_1(N)]_k $ on $\mathcal{M}_k(\Gamma_1(N))$ has the following formulae:

\[T_p f =
\begin{cases}
\sum_{i = 0}^{p-1} f|_k \begin{psmallmatrix}
    1 & j \\ 0 & p \end{psmallmatrix}, & \text{if $p | N$,} \\
\sum_{i = 0}^{p-1} f|_k \begin{psmallmatrix}
    1 & j \\ 0 & p \end{psmallmatrix} + f|_k (\begin{psmallmatrix} m & n \\ N & p \end{psmallmatrix} \begin{psmallmatrix}
        p & 0 \\ 0 & 1
    \end{psmallmatrix}), & \text{if $p\not | N$, where $mp - nN = 1$.} \\
\end{cases} 
\]
\end{example}

%Since we can view $\tau \in \HH$ as lattices $\Lambda_\tau$, modular forms can be viewed as functions on rank $1$ lattices. In particular, the Hecke operators above can be reinterpreted as 

%\[T_p f (\Lambda) = p^{k-1} \sum_{\Lambda' \subseteq \Lambda, [\Lambda:\Lambda'] = p} f(\Lambda')\]

There is also an algebraic/geometric interpretation of the double coset operator as a morphism of divisor groups. For $\Gamma_1, \Gamma_2$ congruence subgroups, $\alpha \in GL_2^+(\QQ) $, $\Gamma_3 := \alpha^{-1} \Gamma_1 \alpha \cap \Gamma_2$ and $\Gamma_3' := \alpha \Gamma_3 \alpha^{-1}$. Since points on the modular curve $X(\Gamma)$ have the form $\Gamma \tau$, we have a diagram at the level of groups which induces a diagram on modular curves:


\begin{align*}
\Gamma_2 \hookleftarrow \Gamma_3 \xrightarrow{\cong} \Gamma_3' \hookrightarrow \Gamma_1 \\
X_2 \xleftarrow{\pi_2} X_3 \xrightarrow{\cong} X_3' \xrightarrow{\pi_1} X_1
\end{align*}

Suppose $\Gamma_3 / \Gamma_2 = \bigcup_j \Gamma_3 \gamma_{2,j}$ and $\beta_j = \alpha \gamma_{2,j}$. Then the double coset operator induces a map on the divisor groups  after $\ZZ$-linear extension:

\begin{align*}
    \Div(X_2) &\rightarrow \Div(X_1) \\
    \Gamma_2 \tau &\mapsto \sum_j \Gamma_1 \beta_j \tau
\end{align*}

Specialising to the case of Hecke operator, we obtain a similar diagram. 

We could benefit from the moduli interpretation of modular curves for the case of Hecke operators by defining it as a correspondence. For $H \leq GL_2(\ZZ/N\ZZ)$ and $p$ coprime to $N$, we obtain the modular curve $X_H$ and its fiber product $X_H(p) := X_0(p) \times_{X(1)} X_H$. There are two degeneracy maps $\alpha,\beta: X_H(p) \rightarrow X_H$ defining the Hecke operator at $p$ where one forgets the cyclic group of order $p$ and the other quotients out by the cyclic group of order $p$.


\[
\begin{tikzcd}[column sep=small]
 & X_H(p) \arrow{dl}[swap]{\alpha} \arrow{dr}{\beta} & \\
X_H \arrow[rr,dashed] & & X_H
\end{tikzcd}
\]

By Picard functoriality, for a point $(E,\mathfrak{n}) \in X_H$ where the level structure $\mathfrak{n}$ is determined by $H$, we have an algebraic description of the Hecke operator at $p$: \[T_p(E,\mathfrak{n)} := \alpha^* \beta_* (E,\mathfrak{n}) = \sum_{f:E\rightarrow E', deg(f) = p} (E',f(\mathfrak{n})).\]


\subsection{Coleman integrals}\label{sec:coleman_integration}

Coleman's construction of $p$-adic line integrals share many similar properties as their complex-analytic analogue. Below we record some properties of Coleman integrals from \cite{Coleman1,coleman85} that will be used in our calculations.

\begin{theorem} \label{coleman_def}
Let $X/\QQ_p$ be a smooth, projective, and geometrically irreducible curve with good reduction at $p$, let $J$ be the Jacobian of $X$.Then there is a $p$-adic integral 

\[ \int_P^Q \omega \in \overline{\QQ}_p\]

with $P,Q \in X(\overline{\QQ}_p), \omega \in H^0(X,\Omega^1)$ satisfying:

\begin{enumerate}
    \item The integral is $\overline{\QQ}_p$ linear in $\omega$,
    \item We have additivity of endpoints:
    \begin{equation*}
        \int_P^Q \omega = \int_P^R \omega + \int_R^Q \omega,
    \end{equation*}
    \item 
    \begin{equation*}
        \int_P^Q \omega + \int_{P'}^{Q'} \omega = \int_P^{Q'} \omega + \int_{P'}^Q \omega
    \end{equation*}
        Thus, we can define $\int_D \omega$, where $D \in Div^0_X(\overline{\QQ}_p)$. Also, if $D$ is principal, $\int_D \omega = 0$,
        
    \item There is an open subgroup of $J(\QQ_p)$ such that $\int_P^Q \omega$ can be computed in terms of power series in some uniformiser by formal term-by-term integration. In particular, $\int_P^P \omega = 0$,



        \item The integral is compatible with the action of $Gal(\overline{\QQ}_p/\QQ_p)$. In particular, if $P,Q \in X(\QQ_p)$ then $\int_P^Q \omega \in \QQ_p$.
        \item Let $P_0 \in X(\overline{\QQ}_p)$ be fixed and $\omega \neq 0$. Then the set of $P \in X(\overline{\QQ}_p)$ reducing to $X(\overline{\FF}_p)$ such that $\int_{P_0}^P \omega = 0$ is finite,

        \item If $U \subseteq X, V \subseteq Y $ are wide open subspaces of the rigid analytic spaces $X,Y$, $\omega$ a 1-form on $V$, and $\phi:U \rightarrow V$ a rigid analytic map, then we have the change of variables formula:
        
        \begin{equation*}
            \int_P^Q \phi^* \omega = \int_{\phi(P)}^{\phi(Q)} \omega,
        \end{equation*}
        \item We have an analogue of the Fundamental Theorem of Calculus: $\int_P^Q df = f(Q) - f(P)$,

\end{enumerate}
\end{theorem}

\begin{defn}\label{def:tiny_integral}
The Coleman integral $\int_P^Q \omega$ is called a \emph{tiny integral} if $P$ and $Q$ reduce to the same point in $X_{\FF_p}(\overline{\FF}_p)$, i.e., they lie in the same residue disc.

\end{defn}


Explicitly, if $P$ and $Q$ are in the same residue disc, then the differential form can be expressed as a power series in terms of a uniformiser at $P$. The tiny integral can be computed by formally integrating the power series and evaluated at the endpoints: \[\int_P^Q \omega = \int_{t(P)}^{t(Q)} \omega(t) = \int_{t(P)}^{t(Q)}\sum a_i t^i dt= \sum \frac{a_i}{i+1} (t(Q) - t(P))^{i+1}.\]


Coleman’s construction is suitable for computations. In \cite{BBK10}, the authors demonstrated an algorithm to compute single Coleman integrals for hyperelliptic curves. Their method is based on Kedlaya's algorithm for computing the Frobenius action on the de Rham cohomology of hyperelliptic curves \cite{Kedlaya_coho_hyper} and this is generalized to arbitrary smooth curves \cite{balatuit, Tui16, Tui17}. Despite recent developments in this direction, the current implementations require nice affine plane models for the curves as inputs. Since modular curves tend to have large gonality, such models are not readily available and are often bottlenecks in existing algorithms.

\subsection{Modular curves}
In this section, we define our object of study. Modular curves have rich structures as Riemann surfaces, algebraic curves and moduli spaces of elliptic curves (with some torsion information). We frequently use properties from various perspectives interchangeably.

\subsubsection{As Riemann surfaces}
Let $\Gamma \leq \SL_2(\ZZ)$ be a subgroup of finite index. $\HH$ inherits the Euclidean topology from $\CC$ and so $Y(\Gamma) := \Gamma \backslash \HH$ carries the quotient topology that is Hausdorff. $Y(\Gamma)$ can be compactified by adjoining cusps, which are orbits of $\PP^1(\QQ)$ under the action of $\Gamma$. The resulting quotient space $X(\Gamma) := \Gamma \backslash \HH^*$ where $\HH^* := \HH \cup \PP^1(\QQ)$ is called the modular curve associated to $\Gamma$. One could further show that by considering elliptic points and cusps, one can choose suitable charts, therefore giving $Y(\Gamma)$ and $X(\Gamma)$ the structure of Riemann surface.

This approach allows us to use techniques from Riemann surfaces, e.g., genus/ramification theory, Riemann-Hurwitz formula, Riemann-Roch, etc. to study modular curves.

\subsubsection{As algebraic curves}

For a finite index subgroup $\Gamma \leq \SL_2(\ZZ)$. The associated modular curve $X(\Gamma)$ has the structure of a compact Riemann surface. Compact Riemann surfaces and complex algebraic curves are equivalent notions \cite{forster}. Note that we are also considering modular curves where the determinant map on the subgroup $H \leq \GL_2(\ZZ/N\ZZ)$ is surjective. By Theorem 7.6.3 in \cite{Shurman}, these algebraic curves are in fact defined over $\QQ$. We have a Galois-theoretic correspondence between curves and their function fields:

\begin{theorem}{(Curves-Fields Correspondence)} For any field $k$, there is a bijection:

\begin{align*}
\{\text{$C/k$ smooth projective algebraic curves}\}/\cong &\leftrightarrow \{\text{$K/k$ function field extensions over $k$}\}/\sim \\
C &\mapsto k(C)
\end{align*}

Furthermore, this is contravariant: a nonconstant morphism from  algebraic curves $C$ to $C'$ over $k$ corresponds to a field morphism from $k(C')$ to $k(C)$.

\end{theorem}

The above theorem allows us to work with simpler objects, i.e., we can replace curves and their morphisms by fields and field injections. In particular, the function field of the modular curve $X(\Gamma)$ consists of modular functions of weight $0$ and level $\Gamma$. 

\subsubsection{As moduli spaces of elliptic curves}

For each $\tau \in \HH$, one could associate it with a lattice $\Lambda_\tau := \ZZ + \tau \cdot \ZZ \subseteq \CC$. The resulting quotient space $\CC/\Lambda_\tau$ is a compact Riemann surface of genus $1$, an elliptic curve. Conversely, for any elliptic curve, as a genus 1 compact Riemann surface, the homology group of the elliptic curve $H_1(E,\ZZ)$ is generated by two loops, $\gamma_1, \gamma_2$. For an invariant differential $\omega$ of the elliptic curve, we can construct the lattice generated by the periods $\Lambda_E = (\int_{\gamma_1} \omega) \cdot \ZZ + (\int_{\gamma_2} \omega) \cdot \ZZ$. This can be renormalised so that $\Lambda_E = \ZZ + \tau \cdot \ZZ$ with $\tau = (\int_{\gamma_1} \omega) /(\int_{\gamma_2} \omega) \in \HH$. In particular the points on $\HH$ correspond to elliptic curves.

For $\Gamma \leq \SL_2(\ZZ)$, the modular curve $X(\Gamma)(\bar{\QQ})$ parametrise elliptic curves with some torsion data, i.e., a pair $(E, \phi)$ where $E$ is an elliptic curve defined over $\bar{\QQ}$ and $\phi$ is an isomorphism of its $N$-torsion points $\phi: E[N] \rightarrow (\ZZ/N\ZZ)^2$. Furthermore, there is an action of the absolute Galois group $Gal(\bar{\QQ}/\QQ)$ on $(E,\phi)$ and we say that $(E,\phi)$ is a $\QQ$-rational point if it is invariant under the action. We can view points on modular curves as elliptic curves with certain torsion structures which allows us to apply properties of elliptic curves to study the rational points on $X(\Gamma)$.

\begin{example}
Let $H \leq \GL_2(\ZZ/N\ZZ)$ be a subgroup such that
\begin{itemize}
    \item $-I \in H$,
    \item the determinant map $\det: H \rightarrow (\ZZ/N\ZZ)^\times$ is surjective.
\end{itemize}
Then for an integer $N \geq 1$, we have the congruence subgroup $\Gamma_H(N) = \{ A \in \SL_2(\ZZ) : A \pmod{N} \in H \}$, which gives rise to the modular curves $X_H := X(\Gamma_H(N))$.

Following Example \ref{example:1_congsgp}, the corresponding modular curves parametrise:

\begin{itemize}
    \item $X(N):= X(\Gamma(N))$ consists of $(E,(P,Q))$ an elliptic curve and a pair of points generating the $N$-torsion subgroup of $E$.
    \item $X_1(N) := X(\Gamma_1(N))$ consists of $(E,Q)$ an elliptic curve and a point of order $N$.
    \item $X_0(N) := X(\Gamma_0(N))$ consists of $(E,C)$ an elliptic curve and a cyclic subgroup of order $N$.
\end{itemize}
\end{example}
\subsection{Hecke operators}

We begin with the definition of Hecke operators as operators on spaces of modular forms. These are used in conjunction with spectral theory to show that the inner product space of modular forms contains a basis of modular forms that are eigenvectors under the Hecke operators $\{T_p\}_p$. Hecke operators are defined on modular forms and modular curves. We use both the transcendental and algebraic/geometric definitions of Hecke operators in our algorithm.

\begin{defn}
Let $\Gamma_1, \Gamma_2$ be congruence subgroups of $SL_2(\ZZ)$ and $\alpha \in GL_2^+(\QQ)$. 

\begin{itemize}
    \item We define the \textit{double coset} $\Gamma_1 \alpha \Gamma_2$ as the set 

\[
\Gamma_1 \alpha \Gamma_2 := \{\gamma_1 \alpha \gamma_2 : \gamma_1 \in \Gamma_1, \gamma_2 \in \Gamma_2\}
\]
\item This gives rise to the \textit{double coset operators}:

\begin{align*}
(\ \cdot \ )|_k [\Gamma_1 \alpha \Gamma_2] : \mathcal{M}_k(\Gamma_1) &\rightarrow \mathcal{M}_k(\Gamma_2) \\
f(\tau) &\mapsto f|_k \Gamma_1 \alpha \Gamma_2 := \sum_i f|_k \beta_i
\end{align*}

where $\Gamma_1 \alpha \Gamma_2 = \bigcup_i \Gamma_1 \beta_i$ is a (finite) disjoint coset decomposition that does not depend on the choice of decomposition. This map restricts to an operator on the space of cusp forms $(\ \cdot \ )|_k [\Gamma_1 \alpha \Gamma_2] : \mathcal{S}_k(\Gamma_1) \rightarrow \mathcal{S}_k(\Gamma_2)$.
\end{itemize}
\end{defn}

We follow the approach \cite{Assaf2020} to define Hecke operators.

\begin{defn}
Fix a congruence subgroup $\Gamma$ with $\bar{\Gamma} \leq SL_2(\ZZ/N\ZZ)$.
Let $\alpha \in M_2(\ZZ)$ such that $\det (\alpha) \in \det (\bar{\Gamma})$ and $\alpha \pmod{N} \in \bar{\Gamma}$. We define the Hecke operator as

\[
T_p = T_\alpha = ( \ \cdot \ )|_k [\Gamma \alpha \Gamma]
\]
\end{defn}

\begin{example} (\cite{Shurman} Prop. 5.2.1)
The theory of Hecke operators can be made explicit for certain congruence subgroups. The Hecke operator $T_p = [ \Gamma_1(N) \begin{psmallmatrix} 1 & 0 \\ 0 & p \end{psmallmatrix} \Gamma_1(N)]_k $ on $\mathcal{M}_k(\Gamma_1(N))$ has the following formulae:

\[T_p f =
\begin{cases}
\sum_{i = 0}^{p-1} f|_k \begin{psmallmatrix}
    1 & j \\ 0 & p \end{psmallmatrix}, & \text{if $p | N$,} \\
\sum_{i = 0}^{p-1} f|_k \begin{psmallmatrix}
    1 & j \\ 0 & p \end{psmallmatrix} + f|_k (\begin{psmallmatrix} m & n \\ N & p \end{psmallmatrix} \begin{psmallmatrix}
        p & 0 \\ 0 & 1
    \end{psmallmatrix}), & \text{if $p\not | N$, where $mp - nN = 1$.} \\
\end{cases} 
\]
\end{example}

We summarise a well-known result from the theory of Hecke operators to show that there is a basis of cusp forms which are eigenvectors of the Hecke operators \cite{Shurman}:

\begin{theorem}
Let $\Gamma$ be a congruence subgroup of level $N$ and $n$ an integer coprime to $N$. Then,

\begin{itemize}
\item $\mathcal{M}_k(\Gamma),\mathcal{S}_k(\Gamma)$ are inner product spaces with respect to an inner product called the \textit{Petersson inner product}.
\item The Hecke operator $T_n$ is a normal operator with respect to this inner product. There is another family of Hecke operators called the diamond operators which is also a normal operator.
\end{itemize}

Therefore, we have a commuting family of operators on a finite dimensional inner product space and the spectral theorem implies that there is an orthogonal basis of simultaneous eigenvectors formed by the cusp forms. In this case, we say that the cusp forms are \textit{Hecke eigenforms} or simply \textit{eigenforms}.
\end{theorem}

There is also an algebraic/geometric interpretation of the double coset operator as a morphism of divisor groups. For $\Gamma_1, \Gamma_2$ congruence subgroups, $\alpha \in GL_2^+(\QQ) $, $\Gamma_3 := \alpha^{-1} \Gamma_1 \alpha \cap \Gamma_2$ and $\Gamma_3' := \alpha \Gamma_3 \alpha^{-1}$. Since points on the modular curve $X(\Gamma)$ have the form $\Gamma \tau$, we have a diagram at the level of groups which induces a diagram on modular curves:


\begin{align*}
\Gamma_2 \hookleftarrow \Gamma_3 \xrightarrow{\cong} \Gamma_3' \hookrightarrow \Gamma_1 \\
X_2 \xleftarrow{\pi_2} X_3 \xrightarrow{\cong} X_3' \xrightarrow{\pi_1} X_1
\end{align*}

Suppose $\Gamma_3 / \Gamma_2 = \bigcup_j \Gamma_3 \gamma_{2,j}$ and $\beta_j = \alpha \gamma_{2,j}$. Then the double coset operator induces a map on the divisor groups  after $\ZZ$-linear extension:

\begin{align*}
    \Div(X_2) &\rightarrow \Div(X_1) \\
    \Gamma_2 \tau &\mapsto \sum_j \Gamma_1 \beta_j \tau
\end{align*}

Specialising to the case of Hecke operator, we obtain a similar diagram. 

We could benefit from the moduli interpretation of modular curves for the case of Hecke operators by defining it as a correspondence. For $H \leq GL_2(\ZZ/N\ZZ)$ and $p$ coprime to $N$, we obtain the modular curve $X_H$ and its fiber product $X_H(p) := X_0(p) \times_{X(1)} X_H$. There are two degeneracy maps $\alpha,\beta: X_H(p) \rightarrow X_H$ defining the Hecke operator at $p$ where one forgets the cyclic group of order $p$ and the other quotients out by the cyclic group of order $p$.


\[
\begin{tikzcd}[column sep=small]
 & X_H(p) \arrow{dl}[swap]{\alpha} \arrow{dr}{\beta} & \\
X_H \arrow[rr,dashed] & & X_H
\end{tikzcd}
\]

By Picard functoriality, for a point $(E,\mathfrak{n}) \in X_H$ where the level structure $\mathfrak{n}$ is determined by $H$, we have an algebraic description of the Hecke operator at $p$: \[T_p(E,\mathfrak{n)} := \alpha^* \beta_* (E,\mathfrak{n}) = \sum_{f:E\rightarrow E', deg(f) = p} (E',f(\mathfrak{n})).\]



\subsection{Coleman integrals}\label{sec:coleman_integration}

Coleman's construction of $p$-adic line integrals share many similar properties as their complex-analytic analogue. Below we record some properties of Coleman integrals from \cite{Coleman1,coleman85} that will be used in our calculations.

\begin{theorem} \label{coleman_def}
Let $X/\QQ_p$ be a smooth, projective, and geometrically irreducible curve with good reduction at $p$, let $J$ be the Jacobian of $X$.Then there is a $p$-adic integral 

\[ \int_P^Q \omega \in \overline{\QQ}_p\]

with $P,Q \in X(\overline{\QQ}_p), \omega \in H^0(X,\Omega^1)$ satisfying:

\begin{enumerate}
    \item The integral is $\overline{\QQ}_p$ linear in $\omega$,
    \item There is an open subgroup of $J(\QQ_p)$ such that $\int_P^Q \omega$ can be computed in terms of power series in some uniformiser by formal term-by-term integration. In particular, $\int_P^P \omega = 0$,
    \item 
    \begin{equation*}
        \int_P^Q \omega + \int_{P'}^{Q'} \omega = \int_P^{Q'} \omega + \int_{P'}^Q \omega
    \end{equation*}
        Thus, we can define $\int_D \omega$, where $D \in Div^0_X(\overline{\QQ}_p)$. Also, if $D$ is principal, $\int_D \omega = 0$,
        \item The integral is compatible with the action of $Gal(\overline{\QQ}_p/\QQ_p)$,
        \item Let $P_0 \in X(\overline{\QQ}_p)$ be fixed. Then the set of $P \in X(\overline{\QQ}_p)$ reducing to $X(\overline{\FF}_p)$ such that $\int_{P_0}^P \omega = 0$ is finite,
        \item We have additivity of endpoints:
        \begin{equation*}
            \int_P^Q \omega = \int_P^R \omega + \int_R^Q \omega,
        \end{equation*}
        \item If $U \subseteq X, V \subseteq Y $ are wide open subspaces of the rigid analytic spaces $X,Y$, $\omega$ a 1-form on $V$, and $\phi:U \rightarrow V$ a rigid analytic map, then we have the change of variables formula:
        
        \begin{equation*}
            \int_P^Q \phi^* \omega = \int_{\phi(P)}^{\phi(Q)} \omega,
        \end{equation*}
        \item $\int_P^Q df = f(Q) - f(P)$,
        \item If $P,Q \in X(\QQ_p)$ then $\int_P^Q \omega \in \QQ_p$.
\end{enumerate}
\end{theorem}

\begin{defn}\label{def:tiny_integral}
The Coleman integral $\int_P^Q \omega$ is called a \emph{tiny integral} if $P$ and $Q$ reduce to the same point in $X_{\FF_p}(\overline{\FF}_p)$, i.e., they lie in the same residue disc.

\end{defn}


Explicitly, if $P$ and $Q$ are in the same residue disc, then the differential form can be expressed as a power series in terms of a uniformiser at $P$. The tiny integral can be computed by formally integrating the power series and evaluated at the endpoints: \[\int_P^Q \omega = \int_{t(P)}^{t(Q)} \omega(t) = \int_{t(P)}^{t(Q)}\sum a_i t^i dt= \sum \frac{a_i}{i+1} (t(Q) - t(P))^{i+1}.\]


Coleman’s construction is suitable for computations. In \cite{BBK10}, the authors demonstrated an algorithm to compute single Coleman integrals for hyperelliptic curves. Their method is based on Kedlaya's algorithm for computing the Frobenius action on the de Rham cohomology of hyperelliptic curves \cite{Kedlaya_coho_hyper} and this is generalized to arbitrary smooth curves \cite{balatuit, Tui16, Tui17}. Despite recent developments in this direction, the current implementations require nice affine plane models for the curves as inputs. Since modular curves tend to have large gonality, such models are not readily available and are often bottlenecks in existing algorithms.
\chapter{Coleman Integration on Modular Curves}

In this section, we introduce an algorithm that computes single Coleman integrals between any two points on modular curves. The modular curves in consideration have congruence subgroups $\Gamma_H \leq SL_2(\ZZ)$ where $H \leq GL_2(\ZZ/N\ZZ)$ and

\begin{itemize}
    \item $-I \in H$,
    \item $\det: H \rightarrow \ZZ/N\ZZ$ is surjective.
\end{itemize}

Furthermore, our method extends to the Atkin-Lehner quotients of modular curves with a slight modification. Another innovation is that the algorithm does not require affine models of the modular curves, which are often required as inputs in previously known algorithms.

The algorithm to compute $\int_Q^R \omega$ for any $P,Q \in X$, $\omega \in H^0(X, \Omega^1)$ has the following major steps:

\begin{enumerate}
    \item (Reduction) Write $\int_Q^R \omega$ into a sum of tiny integrals.
    \item (Basis and uniformiser) Find a basis of holomorphic $1$-forms and a suitable uniformiser.
    \item (Hecke operator) Compute the action of Hecke operator on cusp forms and points.
    \item (Power series expansion) Write the $1$-forms as a power series in the uniformiser. This involves algebraic approximations after solving a system of equations over $\CC$.
    \item (Evaluation) Formally integrate and evaluate at the end points.
\end{enumerate}


\section{Breaking the Coleman integrals into tiny integrals}

Let $X/\QQ$ be a modular curve associated to a congruence subgroup $\Gamma$, two points $Q,R \in X(\bar{\QQ})$, $\{\omega_1, \ldots, \omega_g\}$ a $\QQ$-basis of $H^0(X,\Omega^1)$ where $g$ is the genus of the curve and $p$ a prime of good reduction on $X$.

The Hecke operator at $p$, $T_p$, acts on the weight $2$ cusp forms, which corresponds to the holomorphic $1$-forms:

\[T_p^*\begin{pmatrix} \omega_1 \\\vdots \\ \omega_g \end{pmatrix}  = A\begin{pmatrix} \omega_1 \\\vdots \\ \omega_g \end{pmatrix}.\]

where $A$ is the Hecke matrix acting on the basis of cusp forms. Since the Hecke operators are linear, integrating between the points $Q$ and $R$ gives:

\[\begin{pmatrix} \int^Q_RT_p^*\omega_1 \\\vdots \\ \int^Q_RT_p^*\omega_g \end{pmatrix}  = A\begin{pmatrix} \int^Q_R\omega_1 \\\vdots \\ \int^Q_R\omega_g \end{pmatrix}.\]


For any $\omega \in H^0(X,\Omega^1)$, using the definition of Hecke operator as a correspondence and the functoriality of Coleman integrals, we obtain the following equality:

\[\int^Q_R T_p^*(\omega) = \int^{T_p(Q)}_{T_p(R)} \omega = \sum_{i=0}^{p} \int^{Q_i}_{R_i} \omega,\] where $T_p(Q) = \sum_{i=0}^p Q_i$ and  $T_p(R) = \sum_{i=0}^p R_i$. Note that there are $p+1$ subgroups of order $p$ in $E[p]$, giving rise to $p+1$ isogenies of degree $p$. These isogenies need not be defined over $\QQ$.

By considering $((p+1)\int_{Q}^R \omega - \int_Q^R T_p^* \omega)$, we have the following fundamental equation:

\begin{equation}\label{eq:fundamental-eqn}
   ((p+1)I-A)\begin{pmatrix} \int^Q_R\omega_1 \\\vdots \\ \int^Q_R\omega_g \end{pmatrix} =  \begin{pmatrix} \sum_{i=0}^{p}\int^Q_{Q_i} \omega_1 - \sum_{i=0}^{p}\int^R_{R_i} \omega_1 \\\vdots \\ \sum_{i=0}^{p}\int^Q_{Q_i} \omega_g - \sum_{i=0}^{p}\int^R_{R_i} \omega_g \end{pmatrix}.
\end{equation}

 The $Q_i$'s and $R_i$'s are by definition $p$-isogenous to $Q$ and $R$, therefore, the Eichler-Shimura relation (\cite{Shurman} Theorem 8.7.2) implies that they are in the same residue discs respectively. So the vector on the right hand side consists of sums of tiny integrals. On the left hand side, the matrix $((p+1)I - A)$ is invertible by the Ramanujan bound -- the Hecke matrix $A$ has eigenvalues $\{a_p\}$ which satisfy $|a_p| \leq 2 \sqrt{p}$.

From the above discussion, since any $\omega$ is a linear combination of the $\omega_j$'s, we can simultaneously compute the Coleman integrals $\int_Q^R \omega$ once we have evaluated the tiny integrals $\sum_{i=0}^p \int_{Q_i}^Q \omega$ and $\sum_{i=0}^p \int_{R_i}^R \omega$.

\section{Computing a basis of cusp forms}\label{basis:zyinwa}

The spaces of cusp forms for the congruence subgroups $\Gamma(N), \Gamma_1(N)$ and $\Gamma_0(N)$ are available in software packages \cite{sagemath} and \cite{magma}. For $H \leq GL_2(\ZZ/N\ZZ)$ satisfying the conditions above, $\mathcal{S}_2(\Gamma(N))^H$, the space of weight $2$ cusp forms invariant under $H$, is isomorphic to $\mathcal{S}_2(\Gamma_H)$ and therefore isomorphic to $H^0(X_H,\Omega^1)$.

The problem of computing a basis of cusp form reduces to computing the action of $H \leq \GL_2(\ZZ/N\ZZ)$ on $\mathcal{S}_2(\Gamma(N))$.We follow \cite{Zywina2020ComputingAO,Brunault2020} to compute the (well-defined) action. Note that $\SL_2(\ZZ)$ is freely generated by the two matrices $S = \begin{psmallmatrix} 0 & -1 \\ 1 & 0 \end{psmallmatrix}$ and $T = \begin{psmallmatrix} 1 & 1 \\ 0 & 1 \end{psmallmatrix}$. Since cusp forms of $\mathcal{S}_2(\Gamma(N))$ have $q_N$-expansions, where $q_N = e^{2\pi i /N}$, the slash-$k$ operator by $T$ introduces a factor $\zeta_N^n$ for the $n$-th Fourier coefficient. On the other hand, the action by $S$ is given by a linear combination of the basis of cusp forms on $\Gamma(N)$ where the coefficients depend on a certain Atkin-Lehner operator $W_N$. Since $\GL_2(\ZZ/N\ZZ)/\SL_2(\ZZ/N\ZZ) \xrightarrow{\cong} (\ZZ/N\ZZ)^\times$, we have:

\begin{itemize}
    \item There is an action of $\SL_2(\ZZ/N\ZZ)$ induced from $\SL_2(\ZZ)$ on the cusp forms,
    \item $\begin{psmallmatrix} 1 & 0 \\ 0 & d \end{psmallmatrix}$ acts on the coefficients of the $q_N$-expansion by $\zeta_N \mapsto \zeta_N^d$, where $\zeta_N$ is a $N$-th root of unity. 
\end{itemize}

For congruence subgroups $\Gamma_0^+(N) := \Gamma_0(N)/w_N$ with an Atkin-Lehner involution, we modify Zywina's Magma implementation to compute our examples.

\begin{remark}
In general, the map $H^0(X_H, \Omega_{X_H}^{\otimes k}) \rightarrow \mathcal{S}_{2k}(\Gamma(N),\QQ(\zeta_N))^H$ is injective \cite{Zywina2020ComputingAO}. It is an isomorphism when $k = 1$, which is what we use here.
\end{remark}

\section{Hecke operators as double coset operators}

Hecke operators act on both cusp forms and the divisor group of the modular curve. To compute them as a double coset operator, we need to compute the coset representatives $\Gamma_H \backslash \Gamma_H \alpha \Gamma_H$ for the congruence subgroup $\Gamma_H$. A few key lemmas will give us a procedure to compute the coset representatives.

\begin{lemma}{(\cite{Shurman} Lemmata 5.1.1, 5.1.2)}\label{lemma:coset_rep}
Let $\Gamma, \Gamma_1, \Gamma_2$ be congruence subgroups and  $\alpha \in GL_2^+(\QQ)$. Then,

\begin{enumerate}
    \item $\alpha^{-1} \Gamma \alpha \cap SL_2(\ZZ) \leq SL_2(\ZZ)$ is a congruence subgroup.
    \item There is a bijection:

    \begin{align*}
        (\alpha^{-1} \Gamma_1 \alpha \cap \Gamma_2 )\backslash \Gamma_2 &\leftrightarrow \Gamma_1 \backslash \Gamma_1 \alpha \Gamma_2 \\
         (\alpha^{-1} \Gamma_1 \alpha \cap \Gamma_2 )\gamma_2 &\mapsto \Gamma_1 \alpha \gamma_2
    \end{align*}

    More concretely, $\{\gamma_{2,i}\}$ is a set of coset representatives for $(\alpha^{-1} \Gamma_1 \alpha \cap \Gamma_2 )\backslash \Gamma_2$ if and only if $\{\alpha \gamma_{2,i}\}$ is a set of coset representatives of $\Gamma_1 \backslash \Gamma_1 \alpha \Gamma_2$.
\end{enumerate}
\end{lemma}

\begin{lemma}{(\cite{shimura} Lemma 3.29(5))} \label{lemma:shimura_coset}
Let $\alpha \in M_2(\ZZ)$ be such that $\det(\alpha) = p$ and $\alpha \pmod{N} \in H$. If $\Gamma_H \alpha \Gamma_H = \bigcup_i \Gamma_H \alpha_i$ is a disjoint union, then $SL_2(\ZZ) \alpha SL_2(\ZZ) = \bigcup_i SL_2(\ZZ)\alpha_i$ is a disjoint union.
\end{lemma}

Using the double coset operator definition of Hecke operators as in Equation \ref{eq:hecke_formula}, the Hecke operators can be computed as follows:

\begin{enumerate}
    \item Find $\alpha \in M_2(\ZZ)$ satisfying $\det(\alpha) = p$, $\alpha \pmod{N} \in H$,
    \item Find the coset representatives $\{\alpha_i\}$ in $(\alpha^{-1} SL_2(\ZZ) \alpha \cap  SL_2(\ZZ))\backslash SL_2(\ZZ)$. Usually, $\alpha$ will be chosen such that $(\alpha^{-1} SL_2(\ZZ) \alpha \cap  SL_2(\ZZ))$ has a clear description. By Lemma \ref{lemma:coset_rep}, $SL_2(\ZZ)\backslash SL_2(\ZZ) \alpha SL_2(\ZZ)$ has coset representatives $\{\alpha \alpha_i\}$,
    \item By Lemma \ref{lemma:shimura_coset}, for each $\alpha \alpha_i$, find $\beta_i \in SL_2(\ZZ)$ such that $\beta_i \alpha \alpha_i \in \Gamma_H$. Then $\{ \beta_i \alpha \alpha_i\}$ will be the desired coset representatives for $\Gamma_H \backslash \Gamma_H \alpha \Gamma_H$.
\end{enumerate}

On the other hand, the Hecke operators act on points on the modular curves. Since we are expressing the Hecke images in terms of the chosen uniformiser, the Hecke images, which correspond to elliptic curves that are $p$-isogeneous to our point, arise as roots of modular polynomials. A table of small modular polynomials is available in \cite{MP1,MP2}.

\section{Tiny integrals via complex number approximation}

We present a method to compute tiny integrals. We first write the $1$-forms or cusp forms as a power series in a chosen uniformiser. We compute the Taylor coefficients of the cusp forms and uniformiser around a point and recover the power series coefficients as algebraic approximations of the complex solutions of a system of equations. The algebraic approximations can be done via an $LLL$-type algorithm from known implementations. We find the corresponding Hecke images of the points in the same residue disc as zeros of modular polynomials since they correspond to elliptic curves that are $p$-isogeneous to our point. Finally, we formally integrate and evaluate at these endpoints.

\begin{algorithm}Computing $\sum_{i=0}^{p}\int^Q_{Q_i} \omega$\label{alg:tiny_integral}

\textbf{Input:}
\begin{itemize}
    \item $\tau_0 \in \HH$ such that $\Gamma\tau_0$ corresponds to a rational point $Q$ on $X$, and $q_0 := e^{2\pi i \tau_0/h}$ where $h$ is the width of the cusp.
    \item A good prime $p$ which does not divide $j(Q)$ or $j(Q)-1728$. 
    \item A cusp form $f\in \mathcal{S}_2(\Gamma)$ given by its $q$-expansion where $q = e^{2\pi i \tau/h}$. We denote the corresponding $1-$form by $\omega$.


\end{itemize}

\textbf{Output:}
\begin{itemize}
    \item The sum of tiny Coleman integrals $\sum^p_{i=0}\int^Q_{Q_i} \omega \in \QQ_p$, where $T_p(Q) = \sum_{i=0}^p Q_i$.
\end{itemize}

\textbf{Steps:}
\begin{enumerate}
%\item Find $\tau_0\in \mathcal{H}^+$ such that the $\Gamma_0(N)\tau_0$ corresponds to the point $Q$, i.e., $(E,C)$. This $\tau_0$ can be found by first computing $\widetilde{\tau}_0$ such that $\slz\widetilde{\tau}_0$ corresponds to $E$ under \textcolor{red}{add later!!} and then iterate through coset representatives $\gamma_i$ of $\Gamma_0(N)/\slz$ to find $i$ such that $\gamma_i(\widetilde{\tau_0})$ satisfies: \[j(\gamma_i(\widetilde{\tau_0})) = j(N\gamma_i(\widetilde{\tau_0})) = j(E).\] We use $q_0$ to denote $e^{2\pi i\tau_0}$.
\item[1.] \label{algstep:tiny_1} (Writing $\omega$ as a power series in terms of an uniformiser $u$) Fix a precision $n$. Find $x_i \in \QQ$ such that
\begin{align} \label{eq:omega_j_exp}
    \omega = (\sum_{i=0}^n x_i(u)^n + \mathcal{O}(u^{n+1}))d(u).
\end{align}

These $x_i$'s can be found using the following steps:
\begin{enumerate}
    \item[a.] Write $u$ and $\omega_i$ as power series expansions of $q-q_0$ by differentiating their $q$-expansions and evaluating at $q_0$:
    \begin{align*}
    & u = \sum_{i=1}^{C_1} a_i(q-q_0)^i + O((q-q_0)^{C_1+1}),\\
    & \omega = \sum_{i=0}^{C_2} b_i(q-q_0)^i + O((q-q_0)^{C_2+1})dq,\\
    & d(u) = (\sum_{i=1}^{C_1} ia_i(q-q_0)^{i-1} + O((q-q_0)^{C_1}))dq,
\end{align*}
    where $C_1,C_2$ are some fixed precision determined by $n$ and the norm of $q_0$. The coefficients $a_i,\,b_i$'s are in $\CC$.
    \item[b.] Replace $\omega,\,u$ and $d(u)$ by their power series expansions in $q-q_0$ as in equation (\ref{eq:omega_j_exp}). Comparing the coefficients of $(q-q_0)^k$ on both sides gives us the following linear system:
    \[
\begin{pmatrix}
    a_1       & 0 & 0 & \dots & 0 \\
    2a_2       & a_1^2 & 0 & \dots & 0 \\
    3a_3       & 3a_1a_2 & a_1^3 & \dots & 0\\
    \vdots       & \vdots & \vdots & \ddots & \vdots    \\
    (n+1)a_{n+1}      & \sum_{i=1}^{n}a_i(n+1-i)a_{n+1-i} &* & \dots & a_1^{n+1}
\end{pmatrix} \cdot
\begin{pmatrix}
    x_0      \\
    x_1      \\
    x_2     \\
    \vdots   \\
    x_n
\end{pmatrix} = 
\begin{pmatrix}
    b_0   \\
    b_1     \\
    b_2  \\
    
    \vdots   \\
    b_n
\end{pmatrix}
\]
\item[c.] Solve this system of equations and get complex solutions $x_i$'s. These $x_i$'s can be recovered as elements in $\QQ$ using \texttt{algdep} from \texttt{PARI/GP}. This is likely to succeed given sufficient complex precision.

\end{enumerate}


\item[2.] \label{algstep:tiny_2} Calculate $u(Q_i)$ as algebraic numbers. In practice, we use the $j$-invariant function as an uniformiser. We calculate $j(Q_i)$ transcendentally by evaluating the $q$-expansion of the $j$-function on $\beta_i(\tau_0)$ and then obtain the algebraic approximation. On the other hand, the roots of the modular polynomial $\Phi_p(x,j(Q)) = 0$ are the $j$-invariants of elliptic curves that are $p$-isogeneous to $Q$. This gives another (algebraic) method to compute $j(Q_i)$.

\item[3.] \label{algstep:tiny_3} Compute the sum of tiny integrals $\sum\limits_{i=0}^p\int_{Q}^{Q_i}\omega \approx \sum\limits_{i=0}^p \int^{u(Q_i)}_0 (\sum_{j=0}^n x_j u^j du)$ with its $p$-adic expansion.
\end{enumerate}
\end{algorithm}
\section{Computing a basis of cusp forms}\label{basis:zyinwa}

The spaces of cusp forms for the congruence subgroups $\Gamma(N), \Gamma_1(N)$ and $\Gamma_0(N)$ are available in software packages \cite{sagemath} and \cite{magma}. For $H \leq GL_2(\ZZ/N\ZZ)$ satisfying the conditions above, $\mathcal{S}_2(\Gamma(N))^H$, the space of weight $2$ cusp forms invariant under $H$, is isomorphic to $\mathcal{S}_2(\Gamma_H)$ and therefore isomorphic to $H^0(X_H,\Omega^1)$.

The problem of computing a basis of cusp form reduces to computing the action of $H \leq \GL_2(\ZZ/N\ZZ)$ on $\mathcal{S}_2(\Gamma(N))$.We follow \cite{Zywina2020ComputingAO,Brunault2020} to compute the (well-defined) action. Note that $\SL_2(\ZZ)$ is freely generated by the two matrices $S = \begin{psmallmatrix} 0 & -1 \\ 1 & 0 \end{psmallmatrix}$ and $T = \begin{psmallmatrix} 1 & 1 \\ 0 & 1 \end{psmallmatrix}$. Since cusp forms of $\mathcal{S}_2(\Gamma(N))$ have $q_N$-expansions, where $q_N = e^{2\pi i /N}$, the slash-$k$ operator by $T$ introduces a factor $\zeta_N^n$ for the $n$-th Fourier coefficient. On the other hand, the action by $S$ is given by a linear combination of the basis of cusp forms on $\Gamma(N)$ where the coefficients depend on a certain Atkin-Lehner operator $W_N$. Since $\GL_2(\ZZ/N\ZZ)/\SL_2(\ZZ/N\ZZ) \xrightarrow{\cong} (\ZZ/N\ZZ)^\times$, we have:

\begin{itemize}
    \item There is an action of $\SL_2(\ZZ/N\ZZ)$ induced from $\SL_2(\ZZ)$ on the cusp forms,
    \item $\begin{psmallmatrix} 1 & 0 \\ 0 & d \end{psmallmatrix}$ acts on the coefficients of the $q_N$-expansion by $\zeta_N \mapsto \zeta_N^d$, where $\zeta_N$ is a $N$-th root of unity. 
\end{itemize}

For congruence subgroups $\Gamma_0^+(N) := \Gamma_0(N)/w_N$ with an Atkin-Lehner involution, we modify Zywina's Magma implementation to compute our examples.

\begin{remark}
In general, the map $H^0(X_H, \Omega_{X_H}^{\otimes k}) \rightarrow \mathcal{S}_{2k}(\Gamma(N),\QQ(\zeta_N))^H$ is injective \cite{Zywina2020ComputingAO}. It is an isomorphism when $k = 1$, which is what we use here.
\end{remark}
\section{Hecke operators as double coset operators}

Hecke operators act on both cusp forms and the divisor group of the modular curve. To compute them as a double coset operator, we need to compute the coset representatives $\Gamma_H \backslash \Gamma_H \alpha \Gamma_H$ for congruence subgroups $\Gamma_H$. A few key lemmas will give us a procedure to compute the coset representatives.

\begin{lemma}{(\cite{Shurman} Lemmata 5.1.1, 5.1.2)}\label{lemma:coset_rep}
Let $\Gamma, \Gamma_1, \Gamma_2$ be congruence subgroups and  $\alpha \in GL_2^+(\QQ)$. Then,

\begin{enumerate}
    \item $\alpha^{-1} \Gamma \alpha \cap SL_2(\ZZ) \leq SL_2(\ZZ)$ is a congruence subgroup.
    \item There is a bijection:

    \begin{align*}
        (\alpha^{-1} \Gamma_1 \alpha \cap \Gamma_2 )\backslash \Gamma_2 &\leftrightarrow \Gamma_1 \backslash \Gamma_1 \alpha \Gamma_2 \\
         (\alpha^{-1} \Gamma_1 \alpha \cap \Gamma_2 )\gamma_2 &\mapsto \Gamma_1 \alpha \gamma_2
    \end{align*}

    More concretely, $\{\gamma_{2,i}\}$ is a set of coset representatives for $(\alpha^{-1} \Gamma_1 \alpha \cap \Gamma_2 )\backslash \Gamma_2$ if and only if $\{\alpha \gamma_{2,i}\}$ is a set of coset representatives of $\Gamma_1 \backslash \Gamma_1 \alpha \Gamma_2$.
\end{enumerate}
\end{lemma}

\begin{lemma}{(\cite{shimura} Lemma 3.29(5))} \label{lemma:shimura_coset}
Let $\alpha \in M_2(\ZZ)$ be such that $\det(\alpha) = p$ and $\alpha \pmod{N} \in H$. If $\Gamma_H \alpha \Gamma_H = \bigcup_i \Gamma_H \alpha_i$ is a disjoint union, then $SL_2(\ZZ) \alpha SL_2(\ZZ) = \bigcup_i SL_2(\ZZ)\alpha_i$ is a disjoint union.
\end{lemma}

The procedure for computing the Hecke operator as a double coset operator is as follows:

\begin{enumerate}
    \item Find $\alpha \in M_2(\ZZ)$ satisfying $\det(\alpha) = p$, $\alpha \pmod{N} \in H$,
    \item Find the coset representatives $\{\alpha_i\}$ in $(\alpha^{-1} SL_2(\ZZ) \alpha \cap  SL_2(\ZZ))\backslash SL_2(\ZZ)$. Usually, $\alpha$ will be chosen such that $(\alpha^{-1} SL_2(\ZZ) \alpha \cap  SL_2(\ZZ))$ has a clear description. By Lemma \ref{lemma:coset_rep}, $SL_2(\ZZ)\backslash SL_2(\ZZ) \alpha SL_2(\ZZ)$ has coset representatives $\{\alpha \alpha_i\}$,
    \item By Lemma \ref{lemma:shimura_coset}, for each $\alpha \alpha_i$, find $\beta_i \in SL_2(\ZZ)$ such that $\beta_i \alpha \alpha_i \in \Gamma_H$. Then $\{ \beta_i \alpha \alpha_i\}$ will be the desired coset representatives for $\Gamma_H \backslash \Gamma_H \alpha \Gamma_H$.
\end{enumerate}


\section{Tiny integrals via complex number approximation}

We present a method to compute tiny integrals by comparing Taylor coefficients of a system of equations and recovering them as algebraic number approximations from complex solutions.

\begin{algorithm}Computing $\sum_{i=0}^{p}\int^Q_{Q_i} \omega$\label{alg:tiny_integral}

\textbf{Input:}
\begin{itemize}
    \item $\tau_0 \in \HH$ such that $\Gamma\tau_0$ corresponds to a rational point $Q$ on $X$, and $q_0 := e^{2\pi i \tau_0/h}$ where $h$ is the width of the cusp.
    \item A good prime $p$ which does not divide $j(Q)$ and $j(Q)-1728$. 
    \item A cusp form $f\in \mathcal{S}_2(\Gamma)$ given by its $q$-expansion where $q = e^{2\pi i \tau/h}$. We denote the corresponding $1-$form by $\omega$.
    



\end{itemize}

\textbf{Output:}
\begin{itemize}
    \item The sum of tiny Coleman integrals $\sum^p_{i=0}\int^Q_{Q_i} \omega \in \QQ_p$, where $T_p(Q) = \sum_{i=0}^p Q_i$.
\end{itemize}

\textbf{Steps:}
\begin{enumerate}
%\item Find $\tau_0\in \mathcal{H}^+$ such that the $\Gamma_0(N)\tau_0$ corresponds to the point $Q$, i.e., $(E,C)$. This $\tau_0$ can be found by first computing $\widetilde{\tau}_0$ such that $\slz\widetilde{\tau}_0$ corresponds to $E$ under \textcolor{red}{add later!!} and then iterate through coset representatives $\gamma_i$ of $\Gamma_0(N)/\slz$ to find $i$ such that $\gamma_i(\widetilde{\tau_0})$ satisfies: \[j(\gamma_i(\widetilde{\tau_0})) = j(N\gamma_i(\widetilde{\tau_0})) = j(E).\] We use $q_0$ to denote $e^{2\pi i\tau_0}$.
\item[1.] \label{algstep:tiny_1} (Writing $\omega$ as a power series in terms of an uniformiser $u$) Fix a precision $n$. Find $x_i \in \QQ$ such that
\begin{align} \label{eq:omega_j_exp}
    \omega = (\sum_{i=0}^n x_i(u)^n + \mathcal{O}(u^{n+1}))d(u).
\end{align}

These $x_i$'s can be found using the following steps:
\begin{enumerate}
    \item[a.] Write $u$ and $\omega_i$ as power series expansions of $q-q_0$ by differentiating their $q$-expansions and evaluating at $q_0$:
    \begin{align*}
    & u = \sum_{i=1}^{C_1} a_i(q-q_0)^i + O((q-q_0)^{C_1+1}),\\
    & \omega = \sum_{i=0}^{C_2} b_i(q-q_0)^i + O((q-q_0)^{C_2+1})dq,\\
    & d(u) = (\sum_{i=1}^{C_1} ia_i(q-q_0)^{i-1} + O((q-q_0)^{C_1}))dq,
\end{align*}
    where $C_1,C_2$ are some fixed precision determined by $n$ and the norm of $q_0$. The coefficients $a_i,\,b_i$'s are in $\CC$.
    \item[b.] Replace $\omega,\,u$ and $d(u)$ by their power series expansions in $q-q_0$ as in equation (\ref{eq:omega_j_exp}). Comparing the coefficients of $(q-q_0)^k$ on both sides gives us the following linear system:
    \[
\begin{pmatrix}
    a_1       & 0 & 0 & \dots & 0 \\
    2a_2       & a_1^2 & 0 & \dots & 0 \\
    3a_3       & 3a_1a_2 & a_1^3 & \dots & 0\\
    \vdots       & \vdots & \vdots & \ddots & \vdots    \\
    (n+1)a_{n+1}      & \sum_{i=1}^{n}a_i(n+1-i)a_{n+1-i} &* & \dots & a_1^{n+1}
\end{pmatrix} \cdot
\begin{pmatrix}
    x_0      \\
    x_1      \\
    x_2     \\
    \vdots   \\
    x_n
\end{pmatrix} = 
\begin{pmatrix}
    b_0   \\
    b_1     \\
    b_2  \\
    
    \vdots   \\
    b_n
\end{pmatrix}
\]
\item[c.] Solve this system of equations and get complex approximations of $x_i$'s. These $x_i$'s can be recovered as elements in $\QQ$ using \texttt{algdep} from \texttt{PARI/GP}. This is likely to succeed given sufficient complex precision.

\end{enumerate}


\item[2.] \label{algstep:tiny_2} Calculate $u(Q_i)$ as algebraic numbers. In practice, we use the $j$-invariant function as an uniformiser. We calculate $j(Q_i)$ transcendentally by evaluating the $q$-expansion of the $j$-function on $\beta_i(\tau_0)$ and then obtain the algebraic approximation. On the other hand, the roots of the modular polynomial $\Phi_p(x,j(Q)) = 0$ are the $j$-invariants of elliptic curves that are $p$-isogeneous to $Q$. This gives another (algebraic) method to compute $j(Q_i)$.

\item[3.] \label{algstep:tiny_3} Compute the sum of tiny integrals $\sum\limits_{i=0}^p\int_{Q}^{Q_i}\omega \approx \sum\limits_{i=0}^p \int^{u(Q_i)}_0 (\sum_{j=0}^n x_j u^j du)$ with its $p$-adic expansion.
\end{enumerate}
\end{algorithm}

\part{Decoding Failures of BIKE}
\include{2_bike/2_bike}

%% APPENDIX
\appendix
\chapter{Final notes}
  Remove me in case of abdominal pain.



%% END MATTER
% \printindex %% Uncomment to display the index
% \nocite{}  %% Put any references that you want to include in the bib 
%               but haven't cited in the braces.
\bibliographystyle{alpha}  %% This is just my personal favorite style. 
%                              There are many others.
%\setlength{\bibleftmargin}{0.25in}  % indent each item
%\setlength{\bibindent}{-\bibleftmargin}  % unindent the first line
%\def\baselinestretch{1.0}  % force single spacing
%\setlength{\bibitemsep}{0.16in}  % add extra space between items
\bibliography{template}  %% This looks for the bibliography in template.bib 
%                          which should be formatted as a bibtex file.
%                          and needs to be separately compiled into a bbl file.
\singlespace  %to force bibilography environment to use single spacing for each entry 
              %double spacing between entries remains
\end{document}

