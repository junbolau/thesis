\subsection{Coleman integration on modular curves}

\begin{frame}{Coleman integration on modular curves}

For a congruence subgroup $\Gamma$ and two points $P,Q \in X(\Gamma)$, we compute $\int_P^Q \omega$ in the following way:

\begin{enumerate}
\item (Reduction step) Write any arbitrary Coleman integral as a sum of tiny integrals,
\item (Basis and uniformiser) Find a basis of holomorphic $1$-forms and a suitable uniformiser,
\item (Hecke action) Compute the action of Hecke operator on differential forms and points,
\item (Linear algebra over $\mathbb{C}$) Solve a system of equations and recover algebraic solutions,
\item (Evaluation) Formally integrate and evaluate the end points.
\end{enumerate}
\end{frame}

\begin{frame}{(Reduction) Fundamental system of equations}
Let $\Gamma$ be a congruence subgroup, $\omega_i \in H^0(X(\Gamma), \Omega^1)$, $P,Q \in X(\QQ_p)$, $p >2$ a prime of good reduction. We have:

\[
\int_Q^R T_p^* \omega_i = \sum_j \int_Q^R A_{ij} \omega_j = \sum_j \sum_k \int_{Q_k}^{R_k} \omega_j.
\]

where $T_p Q = \sum_k Q_k.$ This gives the following system of equations: 

\begin{equation*}
   ((p+1)I-A)\begin{pmatrix} \int^Q_R\omega_1 \\\vdots \\ \int^Q_R\omega_g \end{pmatrix} =  \begin{pmatrix} \sum_{i=0}^{p}\int^Q_{Q_i} \omega_1 - \sum_{i=0}^{p}\int^R_{R_i} \omega_1 \\\vdots \\ \sum_{i=0}^{p}\int^Q_{Q_i} \omega_g - \sum_{i=0}^{p}\int^R_{R_i} \omega_g \end{pmatrix}.
\end{equation*}

\end{frame}

\begin{frame}{Next steps}

\begin{equation*}
   ((p+1)I-\textcolor{red}{A})\begin{pmatrix} \int^Q_R\omega_1 \\\vdots \\ \int^Q_R\omega_g \end{pmatrix} =  \begin{pmatrix} \textcolor{blue}{\sum_{i=0}^{p}\int^Q_{Q_i} \omega_1 - \sum_{i=0}^{p}\int^R_{R_i} \omega_1} \\\vdots \\ \textcolor{blue}{\sum_{i=0}^{p}\int^Q_{Q_i} \omega_g - \sum_{i=0}^{p}\int^R_{R_i} \omega_g} \end{pmatrix}.
\end{equation*}

\begin{itemize}
\item \textcolor{red}{Action of Hecke operator at $p$ on basis of cusp forms},
\item \textcolor{blue}{Sums of tiny integrals},
\begin{itemize}
\item Action of Hecke operator on points,
\item Basis of cusp forms in terms of uniformiser
\end{itemize}
\end{itemize}

\end{frame}

\begin{frame}{Action of Hecke operator on cusp forms}

For the congruence subgroups $\Gamma_H$ induced by $H \leq \GL_2(\ZZ/N\ZZ)$, $H^0(X(\Gamma_H),\Omega^1) \cong S_2(\Gamma(N))^H$. A modification of Zywina's Magma implementation \footnote{D. Zywina, \href{https://arxiv.org/abs/2001.07270}{Computing actions on cusp forms}} allows us to compute a basis of $H^0(X(\Gamma_H),\Omega^1)$.

Then, using the double coset definition of Hecke operators, one expresses $f|_2 [\Gamma_H \alpha \Gamma_H]$ as a linear combination of basis elements.
\end{frame}

\begin{frame}{Computing $\sum_i \int_{Q_i}^Q \omega $}
\begin{enumerate}
\item Pick a uniformiser $u$ and write $\omega = \sum_j x_j u^j du$,
\item Calculate $u(Q_i)$ as algebraic numbers,
\item Evaluate $\sum_i \int_{Q_i}^Q \omega = \sum_i \int_{u(Q_i)}^{u(Q)} \omega(u) du$.
\end{enumerate}
\end{frame}

\begin{frame}{Examples}

The algorithm works for:

\begin{itemize}
\item Modular curves coming from maximal proper subgroups of $\GL_2(\ZZ/N\ZZ)$ (Serre's Uniformity Conjecture),
\item Modular curves quotiented by the action of Atkin-Lehner involutions.
\end{itemize}

Each example has its input data:

\begin{itemize}
\item Uniformisers,
\item Basis of cusp forms,
\item Known rational points,
\item Hecke action.
\end{itemize}
\end{frame}