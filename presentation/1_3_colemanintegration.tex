\subsection{Coleman integration on modular curves}

\begin{frame}{Coleman integration on modular curves}

We consider congruence subgroups $\Gamma \leq \SL_2(\ZZ)$ where $\Gamma$ is a lift of $H \leq \GL_2(\ZZ/N\ZZ)$ satisfying:

\begin{itemize}
\item $-I \in H$,
\item $\det:H \rightarrow \ZZ/N\ZZ$ is surjective.
\end{itemize}

The general strategy is as follows:

\begin{enumerate}
\item Write any arbitrary Coleman integral as a sum of tiny integrals,
\item Find a basis of holomorphic $1$-forms and a suitable uniformiser,
\item Formally integrate and evaluate at the end points.
\end{enumerate}
\end{frame}

\begin{frame}{Fundamental system of equations}
Let $\Gamma$ be a congruence subgroup, $\omega_i \in H^0(X(\Gamma), \Omega^1)$, $P,Q \in X(\QQ_p)$, $p >2$ a prime of good reduction. We have:

\[
\int_P^Q T_p^* \omega_i = \sum_j \int_P^Q A_{ij} \omega_j = \sum_j \sum_k \int_{P_k}^{Q_k} \omega_j.
\]

where $T_p P = \sum_k P_k.$ This gives the following system of equations: 

\begin{equation*}
   ((p+1)I-A)\begin{pmatrix} \int^Q_R\omega_1 \\\vdots \\ \int^Q_R\omega_g \end{pmatrix} =  \begin{pmatrix} \sum_{i=0}^{p}\int^Q_{Q_i} \omega_1 - \sum_{i=0}^{p}\int^R_{R_i} \omega_1 \\\vdots \\ \sum_{i=0}^{p}\int^Q_{Q_i} \omega_g - \sum_{i=0}^{p}\int^R_{R_i} \omega_g \end{pmatrix}.
\end{equation*}

\end{frame}

\begin{frame}{Next steps}

\begin{equation*}
   ((p+1)I-\textcolor{red}{A})\begin{pmatrix} \int^Q_R\omega_1 \\\vdots \\ \int^Q_R\omega_g \end{pmatrix} =  \begin{pmatrix} \textcolor{blue}{\sum_{i=0}^{p}\int^Q_{Q_i} \omega_1 - \sum_{i=0}^{p}\int^R_{R_i} \omega_1} \\\vdots \\ \textcolor{blue}{\sum_{i=0}^{p}\int^Q_{Q_i} \omega_g - \sum_{i=0}^{p}\int^R_{R_i} \omega_g} \end{pmatrix}.
\end{equation*}

\begin{itemize}
\item \textcolor{red}{Action of Hecke operator at $p$ on basis of cusp forms}
\item \textcolor{blue}{Sums of tiny integrals}
\end{itemize}

\end{frame}

\begin{frame}{Action of Hecke operator on cusp forms}

Let $\Gamma(N) := \{ A \equiv \begin{psmallmatrix} 1 & 0 \\ 0 & 1 \end{psmallmatrix} \pmod{N} \}$. For the congruence subgroups $\Gamma_H$ induced by $H \leq \GL_2(\ZZ/N\ZZ)$, $H^0(X(\Gamma_H),\Omega^1) \cong S_2(\Gamma(N))^H$. A modification of Zywina's Magma implementation \footnote{D. Zywina, \href{https://arxiv.org/abs/2001.07270}{Computing actions on cusp forms}} allows us to compute a basis of $H^0(X(\Gamma_H),\Omega^1)$.

Then, using the double coset definition of Hecke operators, one expresses $f|_2 [\Gamma_H \alpha \Gamma_H]$ as a linear combination of basis elements.
\end{frame}

\begin{frame}{Computing $\sum_i \int_{Q_i}^Q \omega $}
\begin{enumerate}
\item Pick a uniformiser $u$ and write $\omega = \sum_j x_i u^i du$,
\item Calculate $u(Q_i)$ as algebraic numbers,
\item Evaluate $\sum_i \int_{Q_i}^Q \omega $.
\end{enumerate}
\end{frame}

\begin{frame}{Examples}

The algorithm works for modular curves of the form:

\begin{itemize}
\item Serre's Uniformity Conjecture,
\item Atkin-Lehner quotients.
\end{itemize}

Each example has its input data:

\begin{itemize}
\item Uniformisers,
\item Basis of cusp forms,
\item Known rational points,
\item Hecke action.
\end{itemize}
\end{frame}