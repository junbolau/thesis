\subsection{Modular curves and Hecke operators}

\begin{frame}{Contributions (joint work with Chen-Kedlaya)}
\begin{itemize}
\item We develop an algorithm that computes Coleman integrals on modular curves.
\item The algorithm does not rely on models.
\item We compute some examples of well-known modular curves.
\end{itemize}
\end{frame}

\begin{frame}{Congruence subgroups}
Let $\Gamma \leq \SL_2(\ZZ)$ be a subgroup. 

\begin{definition}
We say $\Gamma$ is a congruence subgroup of level $N$ if $\Gamma$ contains $\{ \begin{psmallmatrix} a & b \\ c & d \end{psmallmatrix} \equiv \begin{psmallmatrix} 1 & 0 \\ 0 & 1 \end{psmallmatrix} \pmod{N} \}$ for some positive integer $N$.
\end{definition}

\begin{itemize}
\item $\Gamma(N) := \{ \begin{psmallmatrix} a & b \\ c & d \end{psmallmatrix} \equiv \begin{psmallmatrix} 1 & 0 \\ 0 & 1 \end{psmallmatrix} \pmod{N} \}$.
\item $\Gamma_1(N) := \{ \begin{psmallmatrix} a & b \\ c & d \end{psmallmatrix} \equiv \begin{psmallmatrix} 1 & \ast \\ 0 & 1 \end{psmallmatrix} \pmod{N} \}$.
\item $\Gamma_0(N) := \{ \begin{psmallmatrix} a & b \\ c & d \end{psmallmatrix} \equiv \begin{psmallmatrix} \ast & \ast \\ 0 & \ast \end{psmallmatrix} \pmod{N} \}$.
\end{itemize}
\end{frame}

\begin{frame}{Modular forms and cusp forms}
\begin{definition}
Let $k$ be a positive integer and $\Gamma$ a congruence subgroup we say $f: \mathbb{H} \rightarrow \mathbb{C}$ is a \textit{modular form of weight $k$ and level $\Gamma$} if

\begin{itemize}
\item $f$ is holomorphic,
\item $(c\tau + d)^{-k} f(\gamma \tau) = f(\tau) $ for all $\gamma = \begin{psmallmatrix} a & b \\ c & d \end{psmallmatrix} \in \Gamma$,
\item $f$ satisfies a certain growth condition at the cusps.
\end{itemize}
\end{definition}
\end{frame}

\begin{frame}{Modular curves}
Let $\Gamma \leq \SL_2(\ZZ)$ be a congruence subgroup. The (compactified) quotient space $X(\Gamma) := \Gamma \backslash (\HH \cup \PP^1(\QQ))$ is called the \textit{modular curve with level $\Gamma$}.

\begin{itemize}
\item Riemann surfaces (genus/ramification theory, Riemann-Hurwitz, Riemann-Roch, etc.)
\item Algebraic curves (curves-fields correspondence)
\item Moduli spaces of elliptic curves with torsion
\end{itemize}

\begin{theorem}
There is an isomorphism of $\mathbb{C}$-vector spaces between the space of holomorphic $1$-forms on modular curves and weight $2$ cusp forms.
\end{theorem}
\end{frame}

\begin{frame}{Hecke operators}
Let $\Gamma_1,\Gamma_2 \leq \SL_2(\ZZ)$ be congruence subgroups and $\alpha \in \GL_2^+(\QQ)$. We define the double coset $\Gamma_1 \alpha \Gamma_2$ as the set

\[
\Gamma_1 \alpha \Gamma_2 := \{ \gamma_1 \alpha \gamma_2: \gamma_1 \in \Gamma_1, \gamma_2 \in \Gamma_2 \}
\]

These sets give rise to \textit{double coset operators} which act on both the $1$-forms and the points on the modular curves.

\end{frame}

\begin{frame}{Hecke operators on $1$-forms}

For a congruence subgroup $\Gamma$, we have an isomorphism between the space of holomorphic $1$-forms on $X(\Gamma)$ and the space of weight $2$ cusp forms of level $\Gamma$. For $\alpha \in \GL_2^+(\QQ)$, we define the double coset operator:

\[
f|_2 [\Gamma_1 \alpha \Gamma_2] = \sum_i f|_2 \beta_i
\]

where $f$ is a weight $2$ cusp form of level $\Gamma_1$, $\Gamma_1 \alpha \Gamma_2 = \cup_i \Gamma_1 \beta_i$. Hecke operators at $p$ are double coset operators when $\Gamma_1= \Gamma_2 = \Gamma$ and $\det(\alpha) = p$.

\end{frame}

\begin{frame}{Hecke operators on points}

For $\Gamma_1, \Gamma_2$ congruence subgroups, $\alpha \in GL_2^+(\QQ) $, define $\Gamma_3 := \alpha^{-1} \Gamma_1 \alpha \cap \Gamma_2$ and $\Gamma_3' := \alpha \Gamma_3 \alpha^{-1}$. We have a diagram at the level of groups and the corresponding modular curves:

\begin{align*}
\Gamma_2 \hookleftarrow \Gamma_3 \xrightarrow{\cong} \Gamma_3' \hookrightarrow \Gamma_1 \\
X_2 \xleftarrow{\pi_2} X_3 \xrightarrow{\cong} X_3' \xrightarrow{\pi_1} X_1
\end{align*}

Suppose $\Gamma_3 / \Gamma_2 = \bigcup_j \Gamma_3 \gamma_{2,j}$ and $\beta_j = \alpha \gamma_{2,j}$. Then the double coset operator induces a map on the divisor groups:

\begin{align*}
    \Div(X_2) &\rightarrow \Div(X_1) \\
    \Gamma_2 \tau &\mapsto \sum_j \Gamma_1 \beta_j \tau
\end{align*}

\end{frame}

\begin{frame}{Hecke operators on points}

Consider fiber product $X(\Gamma, p) := X_0(p) \times_{X(1)} X(\Gamma)$. There are two degeneracy maps $\alpha,\beta: X(\Gamma,p) \rightarrow X(\Gamma)$ defining the Hecke operator at $p$ where one forgets the cyclic group of order $p$ and the other quotients out by the cyclic group of order $p$.

\[
X(\Gamma) \xleftarrow{\alpha} X(\Gamma,p) \xrightarrow{\beta} X(\Gamma)
\]

This gives an algebraic description of the Hecke operator at $p$:
\[
T_p(E,\mathfrak{n)} := \alpha^* \beta_* (E,\mathfrak{n}) = \sum_{f:E\rightarrow E', deg(f) = p} (E',f(\mathfrak{n})).
\]

\end{frame}

