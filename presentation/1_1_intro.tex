\section{$p$-adic Integration on Modular Curves}

\subsection{Motivation}

\begin{frame}{Motivation}
\begin{theorem}[Faltings '83]
Let $C/K$ be a nice curve of genus $g \geq 2$. Then the set of rational points $C(K)$ is finite.
\end{theorem}
\begin{itemize}
\item Subsequent proofs by Vojta ('91), Bombieri ('90), Lawrence-Venkatesh ('20)
\end{itemize}
\end{frame}

\begin{frame}{Motivation (cont.)}
\begin{theorem}[Chabauty '41]
Let $C$ be a nice curve of genus $g \geq 2$. Let $J$ be its Jacobian and $r:= rank(J(\QQ))$. If $r < g$, then $C(\QQ)$ is finite.
\end{theorem}

\begin{theorem}[Coleman '85]
Suppose $r < g$ and $p > 2g$ prime of good reduction. Then

\[
|C(\QQ)| \leq |C(\FF_p)| + 2g -2 
\]

In fact, one can compute $C(\QQ)$ in terms of $p$-adic line integrals of $1$-forms.
\end{theorem}
\end{frame}

\begin{frame}{Coleman integrals}
Suppose $C/\QQ_p$ is a nice curve with good reduction at a prime $p$. We can define a $p$-adic integral:

\[
\int_P^Q \omega \in \bar{\QQ}_p
\]

where $P,Q \in C(\bar{\QQ}_p), \omega \in H^0(C, \Omega^1)$. These integrals satisfy many properties: linearity, additivity of endpoints, change of variables, etc.
\end{frame}

\begin{frame}{One crucial property}
If $P,Q$ reduce to the same point in $X_{\FF_p}(\bar{\FF}_p)$, the integral $\int_P^Q \omega$ becomes a \textit{tiny integral}:

\[
\int_P^Q \omega = \int_{t(P)}^{t(Q)} \omega(t) = \int_{t(P)}^{t(Q)} \sum a_i t^i dt = \sum \frac{a_i}{i+1} t(Q)^{i+1} - t(P)^{i+1}
\]
\end{frame}


\begin{frame}{Chabauty-Coleman method}
Suppose $r < g$. Let $b \in C(\QQ)$ be a basepoint. Then

\[
\{ x \in C(\QQ_p): \int_b^x \omega = 0 \} \supseteq C(\QQ)
\]

along with Newton polygons and Mordell-Weil sieves would recover the set of rational points.

\end{frame}

\begin{frame}{Bottleneck}
If $P,Q$ lie in different residue discs, one would need to compute the action of Frobenius on the MW cohomology via Kedlaya's algorithm. After that, one solves the following system of equations:

\[
\sum_j (M - I)_{ij}(\int_P^Q \omega_j) = f_i(P) - f_i(Q) - \int_P^{\phi(P)} \omega_i - \int_{\phi(Q)}^Q \omega_i
\]

where $\phi$ is the Frobenius map which acts on the differentials via $\phi^* \omega_i = df_i + \sum_j M_{ij}\omega_j$. \textbf{This relies on a model of the curve}.
\end{frame}


