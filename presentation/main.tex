%%%%%%%%%%%%%%%%%%%%%%%%%%%%%%%%%%%%%%%%%%%%%%%%%%%%%%%%%%%%%%%
%
% Welcome to Overleaf --- just edit your LaTeX on the left,
% and we'll compile it for you on the right. If you open the
% 'Share' menu, you can invite other users to edit at the same
% time. See www.overleaf.com/learn for more info. Enjoy!
%
%%%%%%%%%%%%%%%%%%%%%%%%%%%%%%%%%%%%%%%%%%%%%%%%%%%%%%%%%%%%%%%
\documentclass{beamer}

\usefonttheme{professionalfonts}
\mode<presentation> {
\usetheme{default}
\usecolortheme{seahorse}
}
\usepackage{ulem}
\usepackage{algorithm2e,algorithmic}
\usepackage{url,amsmath, tikz-cd}
\usepackage{graphicx} % Allows including images
\usepackage{booktabs} % Allows the use of \toprule, \midrule and \bottomrule in tables
\usepackage{mathtools}


\newcommand{\NN}{\mathbb{N}} % Natural numbers
\newcommand{\ZZ}{\mathbb{Z}} % Integers
\newcommand{\QQ}{\mathbb{Q}} % Rationals
\newcommand{\RR}{\mathbb{R}} % Real numbers
\newcommand{\CC}{\mathbb{C}} % Complex numbers
\newcommand{\FF}{\mathbb{F}} % Finite field
\newcommand{\PP}{\mathbb{P}} % Projective space
\newcommand{\Spec}{\text{Spec}} %Spec
\newcommand{\Max}{\text{Max}} %MaxSpec
\newcommand{\HH}{\mathbb{H}}


\newtheorem{proposition}{Proposition}
\newtheorem{remark}{Remark}
\newtheorem{question}{Question}
\BeforeBeginEnvironment{question}{
  \setbeamercolor{block title}{use=example text,fg=white,bg=example text.fg!75!black}
  \setbeamercolor{block body}{parent=normal text,use=block title example,bg=block title example.bg!10!bg}
}
\AfterEndEnvironment{question}{
  \setbeamercolor{block title}{use=structure,fg=white,bg=structure.fg!75!black}
  \setbeamercolor{block body}{parent=normal text,use=block title,bg=block title.bg!10!bg}
}
\newtheorem{answer}{Answer}
\newtheorem{hypothesis}{Hypothesis}
\newtheorem{conjecture}{Conjecture}
\BeforeBeginEnvironment{conjecture}{
  \setbeamercolor{block title}{use=alerted text,fg=white,bg=alerted text.fg!75!black}
  \setbeamercolor{block body}{parent=normal text,use=block title alerted,bg=block title alerted.bg!10!bg}
}
\AfterEndEnvironment{conjecture}{
  \setbeamercolor{block title}{use=structure,fg=white,bg=structure.fg!75!black}
  \setbeamercolor{block body}{parent=normal text,use=block title,bg=block title.bg!10!bg}
}
\newtheorem{metaconjecture}{Meta-Conjecture}
\newtheorem{philosophy}{Philosophy}
\newtheorem{claim}{Claim}
\newtheorem{exercise}{Exercise}

\newtheorem{warning}{Warning}
\BeforeBeginEnvironment{warning}{
  \setbeamercolor{block title}{use=alerted text,fg=white,bg=alerted text.fg!75!black}
  \setbeamercolor{block body}{parent=normal text,use=block title alerted,bg=block title alerted.bg!10!bg}
}
\AfterEndEnvironment{warning}{
  \setbeamercolor{block title}{use=structure,fg=white,bg=structure.fg!75!black}
  \setbeamercolor{block body}{parent=normal text,use=block title,bg=block title.bg!10!bg}
  
}

\DeclareMathOperator{\coker}{coker}
\DeclareMathOperator{\End}{End}
\DeclareMathOperator{\Ext}{Ext}
\DeclareMathOperator{\GL}{GL}
\DeclareMathOperator{\SL}{SL}
\DeclareMathOperator{\Gal}{Gal}
\DeclareMathOperator{\Gr}{Gr}
\DeclareMathOperator{\Hom}{Hom}
\DeclareMathOperator{\OG}{OG}
\DeclareMathOperator{\PGL}{PGL}
\DeclareMathOperator{\Proj}{Proj}
\DeclareMathOperator{\PSL}{PSL}
\DeclareMathOperator{\SO}{SO}
\DeclareMathOperator{\SpG}{SpG}
\DeclareMathOperator{\Stab}{Stab}
\DeclareMathOperator{\Sym}{Sym}
\DeclareMathOperator{\Trace}{Trace}
% \DeclareMathOperator{\det}{det}
\DeclareMathOperator{\Div}{Div}



\renewcommand{\footnotesize}{\fontsize{5pt}{7}\selectfont}



\title{$p$-adic Integration on Modular Curves and Code-based Cryptography}
\author{Jun Bo Lau}
\institute{UCSD}
\date{31 May 2023}

\begin{document}

\frame{\titlepage}

\begin{frame}
\frametitle{Overview} % Table of contents slide, comment this block out to remove it
\tableofcontents % Throughout your presentation, if you choose to use \section{} and \subsection{} commands, these will automatically be printed on this slide as an overview of your presentation
\end{frame}


%----------------------------------------------------------------------------------------
%	PRESENTATION SLIDES
%----------------------------------------------------------------------------------------

\section{$p$-adic Integration on Modular Curves}

\subsection{Motivation}

\begin{frame}{Motivation}
\begin{theorem}[Faltings '83]
Let $C/K$ be a nice curve of genus $g \geq 2$. Then the set of rational points $C(K)$ is finite.
\end{theorem}
\begin{itemize}
\item Subsequent proofs by Vojta ('91), Bombieri ('90), Lawrence-Venkatesh ('20)
\end{itemize}
\end{frame}

\begin{frame}{Motivation (cont.)}
\begin{theorem}[Chabauty '41]
Let $C$ be a nice curve of genus $g \geq 2$. Let $J$ be its Jacobian and $r:= rank(J(\QQ))$. If $r < g$, then $C(\QQ)$ is finite.
\end{theorem}

\begin{theorem}[Coleman '85]
Suppose $r < g$ and $p > 2g$ prime of good reduction. Then

\[
|C(\QQ)| \leq |C(\FF_p)| + 2g -2 
\]

In fact, one can compute $C(\QQ)$ in terms of $p$-adic line integrals of $1$-forms.
\end{theorem}
\end{frame}

\begin{frame}{Coleman integrals}
Suppose $C/\QQ_p$ is a nice curve with good reduction at a prime $p$. We can define a $p$-adic integral:

\[
\int_P^Q \omega \in \bar{\QQ}_p
\]

where $P,Q \in C(\bar{\QQ}_p), \omega \in H^0(C, \Omega^1)$. These integrals satisfy many properties: linearity, additivity of endpoints, change of variables, etc.
\end{frame}

\begin{frame}{One crucial property}
If $P,Q$ reduce to the same point in $X_{\FF_p}(\bar{\FF}_p)$, the integral $\int_P^Q \omega$ becomes a \textit{tiny integral}:

\[
\int_P^Q \omega = \int_{t(P)}^{t(Q)} \omega(t) = \int_{t(P)}^{t(Q)} \sum a_i t^i dt = \sum \frac{a_i}{i+1} t(Q)^{i+1} - t(P)^{i+1}
\]
\end{frame}


\begin{frame}{Chabauty-Coleman method}
Suppose $r < g$. Let $b \in C(\QQ)$ be a basepoint. Then

\[
\{ x \in C(\QQ_p): \int_b^x \omega = 0 \} \supseteq C(\QQ)
\]

along with Newton polygons and Mordell-Weil sieves would recover the set of rational points.

\end{frame}

\begin{frame}{Bottleneck}
If $P,Q$ lie in different residue discs, one would need to compute the action of Frobenius on the MW cohomology via Kedlaya's algorithm. After that, one solves the following system of equations:

\[
\sum_j (M - I)_{ij}(\int_P^Q \omega_j) = f_i(P) - f_i(Q) - \int_P^{\phi(P)} \omega_i - \int_{\phi(Q)}^Q \omega_i
\]

where $\phi$ is the Frobenius map which acts on the differentials via $\phi^* \omega_i = df_i + \sum_j M_{ij}\omega_j$. \textbf{This relies on a model of the curve}.
\end{frame}



\subsection{Modular curves and Hecke operators}

\begin{frame}{Contributions (joint work with Chen-Kedlaya)}
\begin{itemize}
\item We develop an algorithm that computes Coleman integrals on modular curves.
\item The algorithm does not rely on models.
\item We compute some examples of well-known modular curves.
\end{itemize}
\end{frame}

\begin{frame}{Congruence subgroups}
Let $\Gamma \leq \SL_2(\ZZ)$ be a subgroup. 

\begin{definition}
We say $\Gamma$ is a congruence subgroup of level $N$ if $\Gamma$ contains $\{ \begin{psmallmatrix} a & b \\ c & d \end{psmallmatrix} \equiv \begin{psmallmatrix} 1 & 0 \\ 0 & 1 \end{psmallmatrix} \pmod{N} \}$ for some positive integer $N$.
\end{definition}

\begin{itemize}
\item $\Gamma(N) := \{ \begin{psmallmatrix} a & b \\ c & d \end{psmallmatrix} \equiv \begin{psmallmatrix} 1 & 0 \\ 0 & 1 \end{psmallmatrix} \pmod{N} \}$.
\item $\Gamma_1(N) := \{ \begin{psmallmatrix} a & b \\ c & d \end{psmallmatrix} \equiv \begin{psmallmatrix} 1 & \ast \\ 0 & 1 \end{psmallmatrix} \pmod{N} \}$.
\item $\Gamma_0(N) := \{ \begin{psmallmatrix} a & b \\ c & d \end{psmallmatrix} \equiv \begin{psmallmatrix} \ast & \ast \\ 0 & \ast \end{psmallmatrix} \pmod{N} \}$.
\end{itemize}
\end{frame}

\begin{frame}{Modular forms and cusp forms}
\begin{definition}
Let $k$ be a positive integer and $\Gamma$ a congruence subgroup we say $f: \mathbb{H} \rightarrow \mathbb{C}$ is a \textit{modular form of weight $k$ and level $\Gamma$} if

\begin{itemize}
\item $f$ is holomorphic,
\item $(c\tau + d)^{-k} f(\gamma \tau) = f(\tau) $ for all $\gamma = \begin{psmallmatrix} a & b \\ c & d \end{psmallmatrix} \in \Gamma$,
\item $f$ satisfies a certain growth condition at the cusps.
\end{itemize}
\end{definition}
\end{frame}

\begin{frame}{Modular curves}
Let $\Gamma \leq \SL_2(\ZZ)$ be a congruence subgroup. The (compactified) quotient space $X(\Gamma) := \Gamma \backslash (\HH \cup \PP^1(\QQ))$ is called the \textit{modular curve with level $\Gamma$}.

\begin{itemize}
\item Riemann surfaces (genus/ramification theory, Riemann-Hurwitz, Riemann-Roch, etc.)
\item Algebraic curves (curves-fields correspondence)
\item Moduli spaces of elliptic curves with torsion
\end{itemize}

\begin{theorem}
There is an isomorphism of $\mathbb{C}$-vector spaces between the space of holomorphic $1$-forms on modular curves and weight $2$ cusp forms.
\end{theorem}
\end{frame}

\begin{frame}{Hecke operators}
Let $\Gamma_1,\Gamma_2 \leq \SL_2(\ZZ)$ be congruence subgroups and $\alpha \in \GL_2^+(\QQ)$. We define the double coset $\Gamma_1 \alpha \Gamma_2$ as the set

\[
\Gamma_1 \alpha \Gamma_2 := \{ \gamma_1 \alpha \gamma_2: \gamma_1 \in \Gamma_1, \gamma_2 \in \Gamma_2 \}
\]

These sets give rise to \textit{double coset operators} which act on both the $1$-forms and the points on the modular curves.

\end{frame}

\begin{frame}{Hecke operators on $1$-forms}

For a congruence subgroup $\Gamma$, we have an isomorphism between the space of holomorphic $1$-forms on $X(\Gamma)$ and the space of weight $2$ cusp forms of level $\Gamma$. For $\alpha \in \GL_2^+(\QQ)$, we define the double coset operator:

\[
f|_2 [\Gamma_1 \alpha \Gamma_2] = \sum_i f|_2 \beta_i
\]

where $f$ is a weight $2$ cusp form of level $\Gamma_1$, $\Gamma_1 \alpha \Gamma_2 = \cup_i \Gamma_1 \beta_i$. Hecke operators at $p$ are double coset operators when $\Gamma_1= \Gamma_2 = \Gamma$ and $\det(\alpha) = p$.

\end{frame}

\begin{frame}{Hecke operators on points}

For $\Gamma_1, \Gamma_2$ congruence subgroups, $\alpha \in GL_2^+(\QQ) $, define $\Gamma_3 := \alpha^{-1} \Gamma_1 \alpha \cap \Gamma_2$ and $\Gamma_3' := \alpha \Gamma_3 \alpha^{-1}$. We have a diagram at the level of groups and the corresponding modular curves:

\begin{align*}
\Gamma_2 \hookleftarrow \Gamma_3 \xrightarrow{\cong} \Gamma_3' \hookrightarrow \Gamma_1 \\
X_2 \xleftarrow{\pi_2} X_3 \xrightarrow{\cong} X_3' \xrightarrow{\pi_1} X_1
\end{align*}

Suppose $\Gamma_3 / \Gamma_2 = \bigcup_j \Gamma_3 \gamma_{2,j}$ and $\beta_j = \alpha \gamma_{2,j}$. Then the double coset operator induces a map on the divisor groups:

\begin{align*}
    \Div(X_2) &\rightarrow \Div(X_1) \\
    \Gamma_2 \tau &\mapsto \sum_j \Gamma_1 \beta_j \tau
\end{align*}

\end{frame}

\begin{frame}{Hecke operators on points}

Consider fiber product $X(\Gamma, p) := X_0(p) \times_{X(1)} X(\Gamma)$. There are two degeneracy maps $\alpha,\beta: X(\Gamma,p) \rightarrow X(\Gamma)$ defining the Hecke operator at $p$ where one forgets the cyclic group of order $p$ and the other quotients out by the cyclic group of order $p$.

\[
X(\Gamma) \xleftarrow{\alpha} X(\Gamma,p) \xrightarrow{\beta} X(\Gamma)
\]

This gives an algebraic description of the Hecke operator at $p$:
\[
T_p(E,\mathfrak{n)} := \alpha^* \beta_* (E,\mathfrak{n}) = \sum_{f:E\rightarrow E', deg(f) = p} (E',f(\mathfrak{n})).
\]

\end{frame}


\subsection{Coleman integration on modular curves}

\begin{frame}{Coleman integration on modular curves}

We consider congruence subgroups $\Gamma \leq \SL_2(\ZZ)$ where $\Gamma$ is a lift of $H \leq \GL_2(\ZZ/N\ZZ)$ satisfying:

\begin{itemize}
\item $-I \in H$,
\item $\det:H \rightarrow \ZZ/N\ZZ$ is surjective.
\end{itemize}

The general strategy is as follows:

\begin{enumerate}
\item Write any arbitrary Coleman integral as a sum of tiny integrals,
\item Find a basis of holomorphic $1$-forms and a suitable uniformiser,
\item Formally integrate and evaluate at the end points.
\end{enumerate}
\end{frame}

\begin{frame}{Fundamental system of equations}
Let $\Gamma$ be a congruence subgroup, $\omega_i \in H^0(X(\Gamma), \Omega^1)$, $P,Q \in X(\QQ_p)$, $p >2$ a prime of good reduction. We have:

\[
\int_P^Q T_p^* \omega_i = \sum_j \int_P^Q A_{ij} \omega_j = \sum_j \sum_k \int_{P_k}^{Q_k} \omega_j.
\]

where $T_p P = \sum_k P_k.$ This gives the following system of equations: 

\begin{equation*}
   ((p+1)I-A)\begin{pmatrix} \int^Q_R\omega_1 \\\vdots \\ \int^Q_R\omega_g \end{pmatrix} =  \begin{pmatrix} \sum_{i=0}^{p}\int^Q_{Q_i} \omega_1 - \sum_{i=0}^{p}\int^R_{R_i} \omega_1 \\\vdots \\ \sum_{i=0}^{p}\int^Q_{Q_i} \omega_g - \sum_{i=0}^{p}\int^R_{R_i} \omega_g \end{pmatrix}.
\end{equation*}

\end{frame}

\begin{frame}{Next steps}

\begin{equation*}
   ((p+1)I-\textcolor{red}{A})\begin{pmatrix} \int^Q_R\omega_1 \\\vdots \\ \int^Q_R\omega_g \end{pmatrix} =  \begin{pmatrix} \textcolor{blue}{\sum_{i=0}^{p}\int^Q_{Q_i} \omega_1 - \sum_{i=0}^{p}\int^R_{R_i} \omega_1} \\\vdots \\ \textcolor{blue}{\sum_{i=0}^{p}\int^Q_{Q_i} \omega_g - \sum_{i=0}^{p}\int^R_{R_i} \omega_g} \end{pmatrix}.
\end{equation*}

\begin{itemize}
\item \textcolor{red}{Action of Hecke operator at $p$ on basis of cusp forms}
\item \textcolor{blue}{Sums of tiny integrals}
\end{itemize}

\end{frame}

\begin{frame}{Action of Hecke operator on cusp forms}

Let $\Gamma(N) := \{ A \equiv \begin{psmallmatrix} 1 & 0 \\ 0 & 1 \end{psmallmatrix} \pmod{N} \}$. For the congruence subgroups $\Gamma_H$ induced by $H \leq \GL_2(\ZZ/N\ZZ)$, $H^0(X(\Gamma_H),\Omega^1) \cong S_2(\Gamma(N))^H$. A modification of Zywina's Magma implementation \footnote{D. Zywina, \href{https://arxiv.org/abs/2001.07270}{Computing actions on cusp forms}} allows us to compute a basis of $H^0(X(\Gamma_H),\Omega^1)$.

Then, using the double coset definition of Hecke operators, one expresses $f|_2 [\Gamma_H \alpha \Gamma_H]$ as a linear combination of basis elements.
\end{frame}

\begin{frame}{Computing $\sum_i \int_{Q_i}^Q \omega $}
\begin{enumerate}
\item Pick a uniformiser $u$ and write $\omega = \sum_j x_i u^i du$,
\item Calculate $u(Q_i)$ as algebraic numbers,
\item Evaluate $\sum_i \int_{Q_i}^Q \omega $.
\end{enumerate}
\end{frame}

\begin{frame}{Examples}

The algorithm works for modular curves of the form:

\begin{itemize}
\item Serre's Uniformity Conjecture,
\item Atkin-Lehner quotients.
\end{itemize}

Each example has its input data:

\begin{itemize}
\item Uniformisers,
\item Basis of cusp forms,
\item Known rational points,
\item Hecke action.
\end{itemize}
\end{frame}
\chapter{Preliminaries}


\section{Introduction}
Most cryptosystems implemented today rely on certain hard problems in number theory, such as factorisation or the discrete log problem. These problems fall into the general category of Hidden Subgroup Problems. Recently, there has been significant research on quantum computers and quantum algorithms which make use of quantum phenomena to solve some of these problems that are deemed difficult on classical computers(\cite{Shor,Jozsa}). 

While building a large-scale quantum computer is an engineering challenge, some scientists predict that within the twenty to fifty years, sufficiently powerful quantum computers will be built to break most if not all current public key cryptography infrastructure. Taking into account the amount of time to implement quantum resistant cryptosystems in public, the National Institute of Standards and Technology (NIST) initiated a process in 2016 to standardise post-quantum digital signature algorithms (DSA), public-key encryption (PKE), and key-encapsulation mechanisms (KEM). Initially, there were 82 submissions. As of April 2023, there 4 algorithms are selected for standardisation while there are three code-based candidates that are still going through evaluation. There is also an on-ramp call for new DSA's in order to diversify the signature portfolio to include signature schemes that are not based on lattices.


\begin{center}
\textbf{Table 4.1}: NIST Post-Quantum Standardisation Process - Round 4
\end{center}
\begin{table}[h]
\centering
\begin{tabular}{lll}
\hline
 & PKE/KEM & DSA \\ \hline
Selected &  &  \\ \hline
Lattice & 1 & 2 \\
Hash & 0 & 1 \\ \hline
Candidates &  &  \\ \hline
Code & 3 & 0 \\ \hline
\end{tabular}

\end{table}

In this document, we focus on code-based cryptography, more specifically, one of the 4th round candidates in NIST's standardisation process, BIt-flipping Key Encapsulation (BIKE) \cite{BIKE}. In 1978, McEliece introduced the use of error-correcting codes in cryptography \cite{mceliece}. Originally, error-correcting codes are used in telecommunications in which one party transmits a message through a noisy channel and the recipient recovers the original message from a noisy codeword. In McEliece's proposal, one would use a structured code and hide a message by adding as many errors as the decoder can remove so that the codewords are indistinguishable from random codes. So far, there are no major classical or quantum attacks on the McEliece system but the downside is that it suffers from having large key sizes which make implementations costly.

BIKE is an instance of a more general scheme, called Quasi-Circulant Moderate Density Parity Check (QC-MDPC) codes \cite{QCMDPC}. QC-MDPC codes have much smaller key sizes compared to the McEliece cryptosystem and have not suffered from major attacks. One difference between QC-MDPC codes and McEliece's variants is that QC-MDPC codes use decoders which depend on probabilitistic properties, not algebraic ones. Therefore, one expects decoding failures to occur. Furthermore, decoding failures also reveal information about the secret key. An attack by \cite{GJS} exploits these failures by collecting a set of failure-causing inputs and recover the secret key. With this in mind, one needs to consider the use of ephemeral versus static keys in applications and also verify certain security conditions, called indistinguishability under chosen cipher attack (IND-CCA). NIST has considered BIKE as one of the promising candidates and has expressed concerns about its IND-CCA security and decoding failure analysis. 

By design, it is not feasible to directly compute the average Decoding Failure Rate (DFR) for BIKE at cryptographic security levels. It is possible to measure DFR's via extrapolation methods to estimate the DFR for larger parameters from smaller ones \cite{SV:2019:extrapolate,DGK20a}. But one needs to consider a phenomenon known as the \textit{error floor} region of DFR curves to avoid an underestimate of DFR for larger code sizes.  It is known that for LDPC and MDPC codes, the logarithm of the DFR drops significantly faster than linearly, and then linearly as the signal-to-noise ratio is increased \cite{bgf,Richardson03}. Thus a typical DFR curve contains a concave \textit{waterfall} region followed by a near-linear \textit{error floor} region. One must accurately predict the error floor of a DFR curve to accurately predict the DFR for cryptographically relevant code sizes.

For LDPC codes, the error floor regions have been studied extesively via their Tanner graph representations. Recent work \cite{Vasseur-thesis,Vasseur:2021:eprint} has considered several factors affecting the DFR of QC-MPDC codes: choice of decoder \cite{tillich:2018:decoding,SV:2019:extrapolate}, classes of weak keys, and sets of problematic error patterns.

Our approach to this problem is to study a scaled-down version of BIKE, and identify various properties of QC-MDPC codes and their decoding failures through extensive experiements.
\subsection{BIKE and QC-MDPC codes}

\begin{frame}{Binary linear codes and QC-MDPC codes}

\begin{itemize}
    \item A \textbf{binary linear code} $C = C(n,k)$ is a $k$-dimensional subspace of $\FF_2^n$. Elements are called \textbf{codewords}.
\pause    \item A \textbf{generator matrix} $G$ of $C$, is a $k \times n$ matrix such that the rows are a basis of $C$. The nullspace of $G$, represented as a $(n-k)  \times n$ matrix $H$ ,is called \textbf{parity check matrix}. Note that $v \in C \iff Hv^T=0$.
\pause    \item For any $x \in \FF_2^n$, we call $Hx^T =: s$ the \textbf{syndrome} of $x$.
    \end{itemize}

\end{frame}

\begin{frame}{Binary linear codes and QC-MDPC codes}

\begin{itemize}
    \item A MDPC (\textbf{M}oderate \textbf{D}ensity \textbf{P}arity \textbf{C}heck) code is a binary linear code $C(n,k)$ that has a parity check matrix with row weight $w \approx \sqrt{n}$.
\pause    \item A \textbf{circulant} matrix is a matrix such that each row is a cyclic shift of its previous row. For example:
    \[
    \begin{bmatrix}
    1 & 0 & 0 & 1 \\
    1 & 1 & 0 & 0 \\
    0 & 1 & 1 & 0 \\
    0 & 0 & 1 & 1
    \end{bmatrix}
    \]
\pause    \item A \textbf{quasi-cyclic} (or \textbf{quasi-circulant}) is a block sum of circulant matrices.
\end{itemize}
    
\end{frame}

\begin{frame}{Binary linear codes and QC-MDPC codes}

BIKE:

\begin{itemize}
    \item binary linear QC-MDPC code $C(2r,r)$, i.e., $H = $ block sum of two circulant matrices of size $r$.
    \item $r$ is a prime such that $x^r -1 \in \FF_2[x]$ has two irreducible factors.
    \item row weight $w \approx \sqrt{n}$ and column weight $w/2$.
\end{itemize}

For example, $r = 3, w = 2$:

\[
G = \begin{bmatrix}
0 & 1 & 0 & 1 & 0 & 0 \\
0 & 0 & 1 & 0 & 1 & 0 \\
1 & 0 & 0 & 0 & 0 & 1
\end{bmatrix}, \ \ 
H = \begin{bmatrix}
1 & 0 & 0 & 0 & 0 & 1 \\
0 & 1 & 0 & 1 & 0 & 0 \\
0 & 0 & 1 & 0 & 1 & 0
\end{bmatrix}
\]

\end{frame}

\begin{frame}{Syndrome decoding}

\begin{block}{Syndrome decoding problem}
    Given a parity check matrix $H$ and syndrome $s = He^T$, find $e$.
\end{block}

\pause \begin{itemize}
    \item Syndrome decoding for a general linear code is NP-hard.
    \item We decode syndromes using a bit-flipping decoder; only depends on $H$.
    \begin{itemize}
        \item In practice, use different public $\bar{H}$ to encode that makes decoding hard.
        \item Decoder has to run in constant time to be secure.
        \item Not always successful, even with many iterations, but failures are quite rare.
    \end{itemize}
\end{itemize}
    
\end{frame}

\begin{frame}{Basic bit-flipping decoder}


\begin{algorithm}[H]
\textbf{input:} A QC-MDPC matrix $H \in \FF_2^{(n-k) \times n}$, a syndrome $s = He^T \in \FF_2^{n-k}$.\\
\textbf{output:} An error pattern $e' \in \FF_2^n$ such that $He'^T = s$. \\
$e' \leftarrow 0$;\\
$s' \leftarrow s$;\\
$T \leftarrow \texttt{threshold}(context)$;\\
\While{$s' \ne 0$}
{
\For{$j \in \{ 0,\ldots, n-1\}$}
{
\If{ $|h_j \star s | \geq T$}
{
$e'_j \leftarrow 1- e'_j$
}
$s' \leftarrow s - He'^T$
}
}

\caption{Bit-flipping decoder}
\label{alg:seq}
\end{algorithm}

    
\end{frame}

\begin{frame}{Black-Gray-Flip and BIKE}
    \begin{itemize}
        \item BIKE uses the Black-Gray-Flip decoder.
        \item In addition to the threshold $T$, we introduce an uncertainty $\tau$.
        \begin{itemize}
            \item Run $1$ iteration of the previous bit-flip decoder. Record flipped positions in the "black" list and positions with counters at least $T - \tau$ in the "gray" list.
            \item Using the updated $e$, check black positions. Flip if counter $> T$.
            \item Using the updated $e$, check gray positions. Flip if counter $> T$.
            \item Run several more iterations of the bit-flip decoder.
            \end{itemize}
            \item Threshold $T$ is selected using a pre-determined method.
    \end{itemize}
\end{frame}



\subsection{Experiments: DFR at 20-bit security}


\begin{frame}{Our approach (joint with Arpin, Bilingsley, Hast, Perlner, Robinson) }
    \begin{itemize}
        \item Compute average DFR using simulations for security level $\lambda= 20$.
        \item Study contributing factors: classes of weak keys, sets of problematic error patterns, properties of decoders.
    \end{itemize}
\end{frame}



\begin{frame}{20-bit DFR experiments}
        \begin{itemize}
        \item BIKE security specifications require DFR $< 2^{-\lambda}$ for security levels $\lambda = 128,192,256$.
        \item $\lambda = 20$-bit parameters:
        \begin{itemize}
            \item column weight $w/2 = 30$;
            \item error weight $t = 18$;
            \item block size $r \in [389,827]$.
        \end{itemize}
        \item To achieve $\lambda$ bits of security against information set decoding:
        
        \[\lambda \approx t - \frac{1}{2}log_2 r \approx w - log_2 r.
        \]
    \end{itemize}
\end{frame}

\begin{frame}{20-bit DFR experiments}
We ran the simulation on Boston University's Shared Computing Cluster.

\begin{columns}

\column{0.5 \textwidth}
\begin{enumerate}
    \item Generate a random parity check matrix $H$, an error vector $e$.
    \item Compute the syndrome $s = He^T$.
    \item Run the BGF decoder on inputs $H,s$.
    \item Record decoding failures.
\end{enumerate}

\column{0.5 \textwidth}
\begin{figure}
    \centering
    \includegraphics[scale=.12]{Images/buscc.jpg}
\end{figure}
\end{columns}

\end{frame}

\begin{frame}{20-bit DFR experiments}

\begin{columns}

\column{0.5 \textwidth}
\begin{itemize}
    \item Tested $10^8, 10^9$ for $r \in [389, 827]$.
    \item Plotted with 95\% confidence interval.
    \item Fit lines are quadratic for waterafall region and linear for error floor region.
\end{itemize}

\column{0.5 \textwidth}
\begin{figure}
    \centering
    \includegraphics[scale=.07]{Images/DFR-plot-T3.png}
\end{figure}

\end{columns}
    
\end{frame}

\begin{frame}{Problematic error vectors}
In syndrome decoding, one issue can arise:

\begin{block}{Problem}
Given $H,e$, we may get $e_1$,$e_2$ such that:
     \[  He_1^T = s = He_2^T \iff e_1- e_2 \in C\]

    
\end{block}
Richardson considered \textbf{$(u,v)$ near-codewords}: error vectors $|e| = u$ such that $|He^T| = v$. 
    
    \begin{itemize}
        \item With $u,v$ small relative to $n$, the decoder might be trapped and return $|x| \leq t$ but $Hx^T = s$.
    \end{itemize}
    

\end{frame}


\begin{frame}{Problematic error vectors}

    Following Richardson, Vassuer identified $3$ problematic sets of error vectors:

    \begin{itemize}
        \item $\mathcal{C}$: the set of weight $w$ codewords;
        \item $\mathcal{N}$: the set of half rows of $H$;
        \item $2\mathcal{N}$: the set of sum of two elements of $\mathcal{N}$.
    \end{itemize}
    
    Idea:
    \begin{itemize}
        \item Correspond to Richardson's low weight codewords and $(u,v)$ near-codewords.
        \item Vectors close to $\mathcal{S}$ are more likely to be indistinguishable in the syndrome decoding problem.
    \end{itemize}

    
\end{frame}

\begin{frame}{Problematic error vectors}
        For $\mathcal{S} \in \{ \mathcal{C},\mathcal{N}, 2\mathcal{N}\}$, Vasseur defines:
    
    \[
    \mathcal{A}_{t,\ell}(\mathcal{S}) = \bigcup_{v \in \mathcal{S}} \{ e \in \FF_2^n| |e| = t, |e \star v| = \ell \}.
    \]
    
    If $e \in \mathcal{A}_{t,\ell}(\mathcal{S})$, there exists $v\in \mathcal{S}$ such that $|e \star v| = \ell$. We define the distance $\delta:= |v| + t - 2\ell$.
    
\end{frame}


\begin{frame}{The sets $\mathcal{C},\mathcal{N}$ and $ 2\mathcal{N}$ - example.}
    
    \[
G = \begin{bmatrix}
0 & 1 & 0 & 1 & 0 & 0 \\
0 & 0 & 1 & 0 & 1 & 0 \\
1 & 0 & 0 & 0 & 0 & 1
\end{bmatrix}, \ \ 
H = \begin{bmatrix}
1 & 0 & 0 & 0 & 0 & 1 \\
0 & 1 & 0 & 1 & 0 & 0 \\
0 & 0 & 1 & 0 & 1 & 0
\end{bmatrix}
\]

\begin{itemize}
    \item $\begin{bmatrix} 0 & 1 & 0 & 1 & 0 & 0 \end{bmatrix}^T \in \mathcal{C}$.
    \item $\begin{bmatrix} 1 & 0 & 0 & 0 & 0 & 0 \end{bmatrix}^T \in \mathcal{N}$.
    \item $\begin{bmatrix} 0 & 1 & 0 & 0 & 0 & 1 \end{bmatrix}^T \in 2\mathcal{N}$.
    \item $\begin{bmatrix} 1 & 0 & 0 & 1 & 0 & 0 \end{bmatrix}^T \in \mathcal{A}_{2,1}(\mathcal{C})$
\end{itemize}


\end{frame}


\begin{frame}{DFR for special sets}
    For $r = 523,587,659$, we computed DFR on $\mathcal{A}_{18,\ell}(\mathcal{S})$ for various $\delta$.
    
    \begin{block}{Question}
    How does the DFR of $t \in \mathcal{A}_{18,\ell}(\mathcal{S})$ compare to the DFR of generic vectors $|e| = 18$ ?
    \end{block}
\end{frame}

\begin{frame}{DFR for special sets}
    \begin{figure}
    \centering
    \includegraphics[scale=.6]{Images/DFR_20bit_CN2N_new.png}
    \caption{DFR vs $\delta$ for $r=587$}
\end{figure}

\end{frame}

\begin{frame}{Problematic vs generic error vectors}
    
    In our experiments, we recorded decoding failures. 
    
    \begin{block}{Question}
    How many overlaps do decoding failure vectors have with $\mathcal{C},\mathcal{N}, 2\mathcal{N}$?
    \end{block}
    
    For $r = 587$, we found the maximum number of overlaps with $\mathcal{S} \in \{\mathcal{C},\mathcal{N}, 2\mathcal{N} \} $ for each decoding failure vector. We repeat the experiment with random vectors and compare. 
    
\end{frame}


\begin{frame}{Problematic vs generic error vectors - data.}
\begin{columns}
    \column{0.5 \textwidth}
    \begin{figure}
    \includegraphics[scale=.06]{Images/Rplot-587-df.png}
    \caption{Decoding failure vectors}
    \end{figure}
    
    \column{0.5 \textwidth}
    \begin{figure}
    \includegraphics[scale=.06]{Images/Rplot-587-random.png}
    \caption{Randomly generated vectors}
    \end{figure}
\end{columns}

\begin{itemize}
\item Some proportion of decoding failures can be attributed to $\mathcal{N}, 2\mathcal{N}$.
\item A significant proportion of decoding failures do not have more overlap than typical random vectors.
\end{itemize}
\end{frame}

\begin{frame}{Syndrome weights of decoding failure vectors}
    From the DFR experiments of $\mathcal{A}_{t,\ell}(\mathcal{S})$, we observed that syndrome weights is an indicator of decoding failure.
    
    \begin{figure}
        \centering
        \includegraphics[scale=0.6]{Images/average_sw_generic_vs_DF_T3.png}
    \end{figure}
\end{frame}

\begin{frame}{Syndrome weights and distance from $\mathcal{A}_{t,\ell}(\mathcal{S})$.}

For decoding failure vectors, we have:

\begin{itemize}
    \item $\mathcal{C}:$ mean $\ell \approx 3.31$.
    \item $\mathcal{N}:$ mean $\ell \approx 3.42$.
    \item $2\mathcal{N}:$ mean $\ell \approx 5.77$.
\end{itemize}
    \begin{figure}
        \centering
        \includegraphics[scale=0.5]{Images/synweights_587.png}
    \end{figure}
\end{frame}

\begin{frame}{Ongoing work}
    \begin{itemize}
        \item Classify error vectors contributing to decoding failures: are there new categories, in addition to $\mathcal{C},\mathcal{N},2\mathcal{N}$ and $\mathcal{A}_{t,\ell}(\mathcal{S})$?
        \item Apply graph theoretic techniques to study Tanner graphs of QC-MDPC codes.
    \end{itemize}
\end{frame}


\begin{frame}{}
    \centering
    Thank you!
  \end{frame}


\end{document}