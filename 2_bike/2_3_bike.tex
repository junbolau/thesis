\chapter{BIKE}

\section{Parameters and design}

In this section, we focus on BIKE and its security. Details about the choice of parameters and designs can be found on the BIKE website \cite{BIKE}.

BIKE was originally designed for ephemeral use, i.e., in settings where a KEM key pair is generated for every key exchange instance. This provides a countermeasure against the GJS attack \cite{gjs} where an attacker can use decoding failures to recover the private key of the scheme. Since the second round of the NIST PQC process, BIKE proporsed parameter sets that can security in the static-key setting.

The parameters of BIKE are as follows:

\begin{itemize}
\item $r = $ the block size of a parity check matrix,
\item $n = 2r$; the length of the code,
\item $d = $ the column weight of a parity check matrix,
\item $w = 2d$; the row weight of a parity check matrix,
\item $t = $ the error weight of an error pattern,
\item $\lambda = $ the security parameter, in bits.
\end{itemize}

The design principles of BIKE are as follows:

\begin{itemize}
\item $r$ is a prime such that $x^r -1 \in \FF_2[x]$ has only two irreducible factors,
\item $w = 2d = O(\sqrt{n})$,
\item $\lambda \approx t - \frac{1}{2}\log_2(r) \approx w - \frac{1}{2}\log_2(r)$.
\end{itemize}

\begin{table}[h]
\centering
\begin{tabular}{cccc}
\hline
$\lambda$ & $r$ & $w$ & $t$ \\ \hline
128 & 12,323 & 142 & 134 \\
192 & 24,659 & 206 & 199 \\
256 & 40,973 & 274 & 264 \\ \hline
\end{tabular}
\caption{BIKE parameters}
\end{table}

Following the Niederreiter framework, BIKE is defined in the following manner:

\begin{itemize}
\item ( Key generation) Randomly sample two rows $h_0, h_1 \in \FF_2^r$ of fixed weight $d \in \ZZ_{> 0}$ which form the (private) $r \times 2r$ QC-MDPC matrix $\mathbf{H} = [ \mathbf{H}_0 | \mathbf{H_1}]$. The public key is $\mathbf{H}' = [\mathbf{I} | \mathbf{H}_0^{-1}\mathbf{H}_1]$ in systematic form.
\item ( Encapsulation) The message $m$ will be encoded as an error vector $e \in \FF_2^{2r}$ of weight at  most $t$. The ciphertext will be the syndrome $s = He^\top$.
\item ( Decapsulation) Solve the syndrome decoding problem on inputs $H,s$ using the Black-Grey-Flip (BGF) decoder \cite{bgf}.
\end{itemize}

\todo{include BGF decoder here}

\section{Decoding failures, weak keys and near-codewords}

We are interested in the decoding failure rate (DFR) for BIKE using the BGF decoder. In particular, we want to understand the factors causing the error floor phenomenon. In this subsection, we provide definitions of decoding failures and the factors causing them.

\begin{defn}
For a syndrome decoding instance $(H,s)$, where $s = He^\top$ for some unknown $e \in \FF_2^{2r}$, we say a decoding failure occured when the output $e' := Decode(H,s)$ is such that $He'^\top \neq s$.
\end{defn}

In \cite{Vasseur-thesis}, the author identifies three classes of weak keys for BIKE. These are parity check matrices which have higher than usual DFR's compared to random parity check matrices of the same parameters.

\begin{itemize}
\item (Type I) Parity check matrices whose rows in one of the circulant blocks contain consecutive nonzero bits \cite{DGK20b}.
\item (Type II) Parity check matrices such that there are many nonzero bits at regular intervals in each row of one of the circulant blocks.
\item (Type III) Parity check matrices such that there are many intersections between the columns of the two circulant blocks.
\end{itemize}

There are also some sets of vectors that are likely to cause decoding failures than random vectors.

\begin{defn}
Given a parity check matrix $\mathbf{H}$ of a linear code, we say an error vector $e \in \FF_2^{2r}$ is a $(u,v)$-near codeword if $|e| = u$ and $|\mathbf{H}e^\perp|=v$.
\end{defn}

When $u, v$ are small, these near codewords can be likely to cause decoding failures \cite{Richardson03}. Based on the structure of BIKE, Vasseur defines three sets with small $u,v$ as follows: Given a parity check matrix $\mathbf{H} = [\mathbf{H}_0 | \mathbf{H}_1 ]$ with first rows defined by $\mathbf{h}_0, \mathbf{h}_1$ respectively,

\begin{itemize}
\item $\mathcal{C} := row(\mathbf{G})$ where $\mathbf{G} = [ \mathbf{H}_1^\perp | \mathbf{H}_0^\perp ]$ is the generator matrix. This is the set of rows of $\mathbf{G}$ consisting of $(2d,0)$-near codewords.
\item $\mathcal{N} := row([\mathbf{H}_0 | \mathbf{0}]) \cup row([\mathbf{0} | \mathbf{H}_1])$. This is the set of rows of the individual circulant matrices in $\mathbf{H}$ consisting of $(d,d)$-near codewords.
\item $2\mathcal{N} := \mathcal{N} + \mathcal{N}$. This is the set formed by sum of two elements from $\mathcal{N}$ consisting of $(2d - \epsilon_0, 2d - \epsilon_1)$-near codewords for some small $\epsilon_0,\epsilon_1\geq 0$.
\end{itemize}

Furthermore, Vasseur identified and studied the proximity of error vectors to any of the three problematic sets $\mathcal{S} \in \{ \mathcal{C}, \mathcal{N}, 2\mathcal{N} \}$ and how they affect the average DFR. These are the defined by:

\[
\mathcal{A}_{t, \ell}(\mathcal{S}) := \{ v \in \FF_2^{2r} : |v \star c | = \ell \text{ for some } c \in \mathcal{S} \}
\]

