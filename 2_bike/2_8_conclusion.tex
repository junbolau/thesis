\section{Conclusion and future work}

In the analysis of the BIKE cryptosystem at the $20$-bit security level, we have reproduced the error floor phenomenon and obtained large amount of data for analysis. We found that decoding failure error vectors have lower syndrome weights than those of random vectors. Furthermore, as identified in \cite{Vasseur-thesis,Vasseur:2021:eprint}, the three classes of problematic error vectors $\mathcal{C}, \mathcal{N}, 2\mathcal{N}$ and their proximity sets $\mathcal{A}_{t,\ell}(\mathcal{S})$ contain many elements that cause decoding failures. However, our experiments showed these sets are not responsible the bulk of decoding failures.

It therefore remains to further identify classes of error vectors causing decoding failures in our experiments. As part of an ongoing work, the small parameters allow us to adopt a graph theoretic approach to study the Tanner graph representations of these QC-MDPC codes, which allow us to study interesting behaviours coming from absorbing and trapping sets.