\section{Background}

\subsection{Modular forms}

In this section, we give a brief introduction of modular forms, following \cite{Shurman}.

Let $\HH := \{ \tau \in \CC: Im(\tau) > 0 \}$ be the upper half complex plane. The special linear group $SL_2(\ZZ)$ acts on $\HH$ via fractional linear transformations:

\[
\gamma \cdot \tau = \frac{a\tau + b}{c \tau + d}
\] 

where $\gamma = \begin{psmallmatrix} a & b \\ c & d \end{psmallmatrix}, \tau \in \HH$. 

\begin{defn}
Let $f$ be function $f: \HH \rightarrow \CC$ and $k \in \ZZ$. 

\begin{itemize}
    \item The \textit{automorphy factor} is a function \begin{align*}j: GL_2^+(\RR) \times \HH &\rightarrow \CC \\ (\gamma,z) &\mapsto cz+d \end{align*}
    where $\gamma = \begin{psmallmatrix} a & b \\ c & d \end{psmallmatrix}$
    \item The \textit{weight-$k$ slash operator} is defined as 
    \begin{align*}
( \ \cdot \ )|_k (\ \cdot \ ): \Hom(\HH,\CC) \times GL_2^+(\RR) &\rightarrow \Hom(\HH,\CC) \\
(f(z),\gamma) &\mapsto (f|_k \gamma)(z) := (\Det \gamma)^{k-1} j(\gamma,z)^{-k} f(\gamma \cdot z)
    \end{align*}
    \end{itemize}
\end{defn}

The automorphy factory satisfies a cocycle relation $j(\gamma_1\gamma2,z) = j(\gamma_1,\gamma_2 z)j(\gamma_2,z)$ which implies that $GL_2^+(\RR)$ acts on $\Hom(\HH,\CC)$ via $f|_k\gamma_1\gamma_2 = (f|_k\gamma_1)|_k\gamma_2$.

Consider the projection map $\pi: SL_2(\ZZ) \rightarrow SL_2(\ZZ/N\ZZ)$. We define congruence subgroups in the following way.

\begin{example}\label{example:1_congsgp}
Here are some common examples of preimages of $\pi$:

\begin{itemize}
    \item $\Gamma(N) = \pi^{-1} (\begin{psmallmatrix}
        1 & 0 \\ 0 & 1
    \end{psmallmatrix}) = \{ \begin{psmallmatrix}
        a & b \\ c & d 
    \end{psmallmatrix} \in SL_2(\ZZ): \begin{psmallmatrix}
        a & b \\ c & d 
    \end{psmallmatrix} \equiv \begin{psmallmatrix}
        1 & 0 \\ 0 & 1 
    \end{psmallmatrix} \pmod{N}\}$
    \item $\Gamma_1(N) = \pi^{-1} (\{\begin{psmallmatrix}
        1 & \ast \\ 0 & 1
    \end{psmallmatrix}\}) = \{ \begin{psmallmatrix}
        a & b \\ c & d 
    \end{psmallmatrix} \in SL_2(\ZZ): \begin{psmallmatrix}
        a & b \\ c & d 
    \end{psmallmatrix} \equiv \begin{psmallmatrix}
        1 & \ast \\ 0 & 1 
    \end{psmallmatrix} \pmod{N}\}$
    \item $\Gamma_0(N) = \pi^{-1} (\{\begin{psmallmatrix}
        \ast & \ast \\ 0 & \ast
    \end{psmallmatrix}\}) = \{ \begin{psmallmatrix}
        a & b \\ c & d 
    \end{psmallmatrix} \in SL_2(\ZZ): \begin{psmallmatrix}
        a & b \\ c & d 
    \end{psmallmatrix} \equiv \begin{psmallmatrix}
        \ast & \ast \\ 0 & \ast 
    \end{psmallmatrix} \pmod{N}\}$
\end{itemize}
\end{example}

\begin{defn}
$\Gamma \leq SL_2(\ZZ)$ is a \textit{congruence subgroup} if there exists an integer $N \geq 1$ such that $\Gamma(N) \leq \Gamma$. The minimal such $N$ is called the \textit{level} of $\Gamma$.

It follows immediately that congruence subgroups of $SL_2(\ZZ)$ have finite index and correspond to subgroups of $SL_2(\ZZ/N\ZZ)$. The above examples are allcongruence subgroups of level $N$.
\end{defn}

\begin{defn}
    Let $\Gamma \leq SL_2(\ZZ)$ be a congruence subgroup of level $N$, $k \geq 0$ an integer. We say a function $f: \HH \rightarrow \CC$ is a \textit{modular form of weight $k$ with level $\Gamma$} if

    \begin{enumerate}
        \item $f$ is holomorphic,
        \item $f$ is weight-$k$ invariant under $\Gamma$, i.e., $f|_k\gamma = f$ for all $\gamma \in \Gamma$,
        \item $f|_k\alpha $ is holomorphic at $\infty$ for all $\alpha \in SL_2(\ZZ)$, i.e., $(f|_k \alpha )(z)$ is bounded as $z \rightarrow i\infty$.
    \end{enumerate}

    If, in addition, $f|_k \alpha$ vanishes at infinity for all $\alpha \in SL_2(\ZZ)$, we say that $f$ is a \textit{cusp form}. We denote the set of weight-$k$ modular forms with respect to $\Gamma$ (resp. cusp forms) as $\mathcal{M}_k(\Gamma)$ (resp. $\mathcal{S}_k(\Gamma)$).
\end{defn}

Suppose $f$ is a modular form of weight $k$ with level $\Gamma$. Since $\Gamma$ is a congruence subgroup, $\begin{psmallmatrix}
    1 & h \\ 0 & 1
\end{psmallmatrix} \in \Gamma$ for some minimal integer $h\geq 1$, this integer is the \textit{width} of the cusp $\infty$. Since a modular form satisfies $f|_k \gamma = f$ for $\gamma \in \Gamma$, we have $(f|_k \begin{psmallmatrix}
    1 & h \\ 0 & 1
\end{psmallmatrix})(z) = f(z+h) = f(z) $, so $f(z)$ is $h\ZZ$-periodic and admits a Fourier expansion $f(\tau) = \sum_{n=0}^\infty a_n q_{h}^n$ where $q_h = exp(2\pi i\tau / h)$. The third condition of modular forms implies that the Fourier expansion begins at index $0$ and cusp forms satisfy $a_0 = 0$.

\begin{example}
Let $G_k(\tau) = \sum_{c,d) \not = (0,0)} 1/(c\tau + d)^k$. This is a modular form of weight $k$ for $SL_2(\ZZ)$ called \textit{Eisenstein series}. 

    The \textit{$j$-invariant} is a modular form weight $0$, i.e., a modular function and an element of $\CC(X(\SL_2(\ZZ)))$, with $q$-expansion:

    \begin{align*}
        j: \HH \rightarrow \CC, j(\tau) = 1728 \frac{(60 G_4(\tau))^3}{(60 G_4(\tau))^3 - 27 (140 G_6(\tau))^2} = \frac{1}{q} + 744 + 196884q + \ldots.
    \end{align*}
\end{example}

It is a standard result that $\mathcal{M}_k(\Gamma) \supseteq \mathcal{S}_k(\Gamma)$ are finite dimensional complex vector spaces. Modular forms and modular curves are related by the fact that there is an isomorphism between the space of weight $2$ cusp forms and the space of holomorphic differentials on the modular curve $X(\Gamma)$ (see next section). 

\begin{align*}
\mathcal{S}_2(\Gamma) &\xrightarrow{\cong} H^0(X(\Gamma),\Omega^1) \\
f(\tau) &\mapsto f(\tau) d\tau
\end{align*}