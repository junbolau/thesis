
\subsection{Modular curves}
In this section, we define our object of study. Modular curves have rich structures as Riemann surfaces, algebraic curves and moduli spaces of elliptic curves (with some torsion information). We frequently use properties from various perspectives interchangeably.

\subsubsection{As Riemann surfaces}
Let $\Gamma \leq \SL_2(\ZZ)$ be a subgroup of finite index. $\HH$ inherits the Euclidean topology from $\CC$ and so $Y(\Gamma) := \Gamma \backslash \HH$ carries the quotient topology that is Hausdorff. $Y(\Gamma)$ can be compactified by adjoining cusps, which are orbits of $\PP^1(\QQ)$ under the action of $\Gamma$. The resulting quotient space $X(\Gamma) := \Gamma \backslash \HH^*$ where $\HH^* := \HH \cup \PP^1(\QQ)$ is called the modular curve associated to $\Gamma$. One could further show that by considering elliptic points and cusps, one can choose suitable charts, therefore giving $Y(\Gamma)$ and $X(\Gamma)$ the structure of Riemann surface.

This approach allows us to use techniques from Riemann surfaces, e.g., genus/ramification theory, Riemann-Hurwitz formula, Riemann-Roch, etc. to study modular curves.

\subsubsection{As algebraic curves}

For a finite index subgroup $\Gamma \leq \SL_2(\ZZ)$. The associated modular curve $X(\Gamma)$ has the structure of a compact Riemann surface. Compact Riemann surfaces and complex algebraic curves are equivalent notions \cite{forster}. Note that we are also considering modular curves where the determinant map on the subgroup $H \leq \GL_2(\ZZ/N\ZZ)$ is surjective. By Theorem 7.6.3 in \cite{Shurman}, these algebraic curves are in fact defined over $\QQ$. We have a Galois-theoretic correspondence between curves and their function fields:

\begin{theorem}{(Curves-Fields Correspondence)} For any field $k$, there is a bijection:

\begin{align*}
\{\text{$C/k$ smooth projective algebraic curves}\}/\cong &\leftrightarrow \{\text{$K/k$ function field extensions over $k$}\}/\sim \\
C &\mapsto k(C)
\end{align*}

Furthermore, this is contravariant: a nonconstant morphism from  algebraic curves $C$ to $C'$ over $k$ corresponds to a field morphism from $k(C')$ to $k(C)$.

\end{theorem}

The above theorem allows us to work with simpler objects, i.e., we can replace curves and their morphisms by fields and field injections. In particular, the function field of the modular curve $X(\Gamma)$ consists of modular functions of weight $0$ and level $\Gamma$. 

\subsubsection{As moduli spaces of elliptic curves}

For each $\tau \in \HH$, one could associate it with a lattice $\Lambda_\tau := \ZZ + \tau \cdot \ZZ \subseteq \CC$. The resulting quotient space $\CC/\Lambda_\tau$ is a compact Riemann surface of genus $1$, an elliptic curve. Conversely, for any elliptic curve, as a genus 1 compact Riemann surface, the homology group of the elliptic curve $H_1(E,\ZZ)$ is generated by two loops, $\gamma_1, \gamma_2$. For an invariant differential $\omega$ of the elliptic curve, we can construct the lattice generated by the periods $\Lambda_E = (\int_{\gamma_1} \omega) \cdot \ZZ + (\int_{\gamma_2} \omega) \cdot \ZZ$. This can be renormalised so that $\Lambda_E = \ZZ + \tau \cdot \ZZ$ with $\tau = (\int_{\gamma_1} \omega) /(\int_{\gamma_2} \omega) \in \HH$. In particular the points on $\HH$ correspond to elliptic curves.

For $\Gamma \leq \SL_2(\ZZ)$, the modular curve $X(\Gamma)(\bar{\QQ})$ parametrise elliptic curves with some torsion data, i.e., a pair $(E, \phi)$ where $E$ is an elliptic curve defined over $\bar{\QQ}$ and $\phi$ is an isomorphism of its $N$-torsion points $\phi: E[N] \rightarrow (\ZZ/N\ZZ)^2$. Two points $(E_1,\phi_1), (E_2,\phi_2)$ are isomorphic if there is an isomorphism of elliptic curves $\psi: E_1 \rightarrow E_2$ and some matrix $M \in \Gamma$ such that the diagram commutes:

\begin{center}
  \begin{tikzcd}
    E_1[N] \arrow[r, "\phi_1"] \arrow[d, "\psi"] & (\ZZ/N\ZZ)^2 \arrow[d,  "M"] \\
    E_2[N] \arrow[r, "\phi_2"]                &  (\ZZ/N\ZZ)^2.                               
  \end{tikzcd}
\end{center}


Furthermore, there is an action of the absolute Galois group $Gal(\bar{\QQ}/\QQ)$ on $(E,\phi)$ and we say that $(E,\phi)$ is a $\QQ$-rational point if it is invariant under the action. We can view points on modular curves as elliptic curves with certain torsion structures which allows us to apply properties of elliptic curves to study the rational points on $X(\Gamma)$.

\begin{example}
Let $H \leq \GL_2(\ZZ/N\ZZ)$ be a subgroup such that
\begin{itemize}
    \item $-I \in H$,
    \item the determinant map $\det: H \rightarrow (\ZZ/N\ZZ)^\times$ is surjective.
\end{itemize}
Then for an integer $N \geq 1$, we have the congruence subgroup $\Gamma_H(N) = \{ A \in \SL_2(\ZZ) : A \pmod{N} \in H \}$, which gives rise to the modular curves $X_H := X(\Gamma_H(N))$.

Following Example \ref{example:1_congsgp}, the corresponding modular curves parametrise:

\begin{itemize}
    \item $X(N):= X(\Gamma(N))$ consists of $(E,(P,Q))$ an elliptic curve and a pair of points generating the $N$-torsion subgroup of $E$.
    \item $X_1(N) := X(\Gamma_1(N))$ consists of $(E,Q)$ an elliptic curve and a point of order $N$.
    \item $X_0(N) := X(\Gamma_0(N))$ consists of $(E,C)$ an elliptic curve and a cyclic subgroup of order $N$.
\end{itemize}
\end{example}