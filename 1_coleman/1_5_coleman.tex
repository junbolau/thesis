\subsection{Coleman integrals}\label{sec:coleman_integration}

Coleman's construction of $p$-adic line integrals share many similar properties as their complex-analytic analogue. Below we record some properties of Coleman integrals from \cite{Coleman1,coleman85} that will be used in our calculations.

\begin{theorem} \label{coleman_def}
Let $X/\QQ_p$ be a smooth, projective, and geometrically irreducible curve with good reduction at $p$, let $J$ be the Jacobian of $X$.Then there is a $p$-adic integral 

\[ \int_P^Q \omega \in \overline{\QQ}_p\]

with $P,Q \in X(\overline{\QQ}_p), \omega \in H^0(X,\Omega^1)$ satisfying:

\begin{enumerate}
    \item The integral is $\overline{\QQ}_p$ linear in $\omega$,
    \item We have additivity of endpoints:
    \begin{equation*}
        \int_P^Q \omega = \int_P^R \omega + \int_R^Q \omega,
    \end{equation*}
    \item 
    \begin{equation*}
        \int_P^Q \omega + \int_{P'}^{Q'} \omega = \int_P^{Q'} \omega + \int_{P'}^Q \omega
    \end{equation*}
        Thus, we can define $\int_D \omega$, where $D \in Div^0_X(\overline{\QQ}_p)$. Also, if $D$ is principal, $\int_D \omega = 0$,
        
    \item There is an open subgroup of $J(\QQ_p)$ such that $\int_P^Q \omega$ can be computed in terms of power series in some uniformiser by formal term-by-term integration. In particular, $\int_P^P \omega = 0$,



        \item The integral is compatible with the action of $Gal(\overline{\QQ}_p/\QQ_p)$. In particular, if $P,Q \in X(\QQ_p)$ then $\int_P^Q \omega \in \QQ_p$.
        \item Let $P_0 \in X(\overline{\QQ}_p)$ be fixed and $\omega \neq 0$. Then the set of $P \in X(\overline{\QQ}_p)$ reducing to $X(\overline{\FF}_p)$ such that $\int_{P_0}^P \omega = 0$ is finite,

        \item If $U \subseteq X, V \subseteq Y $ are wide open subspaces of the rigid analytic spaces $X,Y$, $\omega$ a 1-form on $V$, and $\phi:U \rightarrow V$ a rigid analytic map, then we have the change of variables formula:
        
        \begin{equation*}
            \int_P^Q \phi^* \omega = \int_{\phi(P)}^{\phi(Q)} \omega,
        \end{equation*}
        \item We have an analogue of the Fundamental Theorem of Calculus: $\int_P^Q df = f(Q) - f(P)$,

\end{enumerate}
\end{theorem}

\begin{defn}\label{def:tiny_integral}
The Coleman integral $\int_P^Q \omega$ is called a \emph{tiny integral} if $P$ and $Q$ reduce to the same point in $X_{\FF_p}(\overline{\FF}_p)$, i.e., they lie in the same residue disc.

\end{defn}


Explicitly, if $P$ and $Q$ are in the same residue disc, then the differential form can be expressed as a power series in terms of a uniformiser at $P$. The tiny integral can be computed by formally integrating the power series and evaluated at the endpoints: \[\int_P^Q \omega = \int_{t(P)}^{t(Q)} \omega(t) = \int_{t(P)}^{t(Q)}\sum a_i t^i dt= \sum \frac{a_i}{i+1} (t(Q) - t(P))^{i+1}.\]


Coleman’s construction is suitable for computations. In \cite{BBK10}, the authors demonstrated an algorithm to compute single Coleman integrals for hyperelliptic curves. Their method is based on Kedlaya's algorithm for computing the Frobenius action on the de Rham cohomology of hyperelliptic curves \cite{Kedlaya_coho_hyper} and this is generalized to arbitrary smooth curves \cite{balatuit, Tui16, Tui17}. Despite recent developments in this direction, the current implementations require nice affine plane models for the curves as inputs. Since modular curves tend to have large gonality, such models are not readily available and are often bottlenecks in existing algorithms.