\section{Conclusion and future work}

In this project, we developed a model-free algorithm to compute single Coleman integrals on modular curves. The algorithm works for all modular curves which arise from maximal proper subgroups of $\GL_2(\FF_p)$ (see Serre's Uniformity Conjecture) and Atkin-Lehner quotients of modular curves.

As a continuation project, we hope to get an analogous model-free algorithm for double integrals in order to apply these methods to quadratic Chabauty since modular curves satisfy the quadratic Chabauty bound \cite{Siksek}. There exists combinatorial identities for iterated Coleman integrals but some of the properties such as additivity of endpoints and functoriality no longer hold since pushforwards of $1$-forms are not well-behaved.

Furthermore, when writing the $1$-forms as power series expansions in terms of the uniformiser, we recover the coefficients by algebraic approximations of the complex solution of a system of equations. This process depends on the complex precision that we allow. Another approach to solve this problem is to look at the differential equations satisfied by modular forms.

One could also consider other choices of uniformisers. Our approach exploits the fact that $p$-isogeneous elliptic curves arise as roots of the modular polynomials. However, modular polynomials often have large coefficients and the splitting fields often have large degrees. One hope is to replace uniformisers by ones that satisfy a simpler modular polynomial. Moreoever, certain rational combinations of the $j$-invariant function yield smaller coefficients in the uniformiser expansion of the $1$-form and it will be interesting to understand the reason behind such choices.