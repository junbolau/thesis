\section{$X_{ns}^+(p)$}

For a prime $p$, we first define the nonsplit Cartan subgroup $C_{ns}$ and its normaliser $C_{ns}^+$. Let $\{ 1, \alpha \}$ be a $\FF_p$- basis of $\FF_{p^2}$. Suppose that $\alpha$ satisfies a minimal polynomial $X^2 - tX + n \in \FF_p[X]$. For any $\beta = x + y\alpha \in \FF_{p^2}^\times$, there is a multiplication-by-$\beta$ map with respect to the basis $\{ 1, \alpha \}$:

\begin{align*}
i_\alpha : \FF_{p^2}^\times &\rightarrow \GL_2(\FF_p) \\
\beta &\mapsto \begin{psmallmatrix} x & -ny \\ y & x + ty \end{psmallmatrix}
\end{align*}

Given this choice of basis, we define the nonsplit Cartan subgroup $C_{ns}(p) \leq \GL_2(\FF_p)$ as the image of $i_\alpha$. The normaliser of the nonsplit Cartan subgroup $C_{ns}^+(p)$ is the subgroup generated by $C_{ns}(p)$ and the conjugation map under $i_\alpha$ coming from $\Gal(\FF_{p^2}/\FF_p)$. $\alpha$ can be chosen to be the squareroot of a quadratic nonresidue $\epsilon$ in $\FF_{p^2}$ satisfying $X^2 - \epsilon^2 = 0$. Then, we have:

\[
 C_{ns}^+(p) = \langle \begin{psmallmatrix} x & \epsilon^2 y \\ y & x\end{psmallmatrix} , \begin{psmallmatrix} 1 & 0 \\ 0 & -1 \end{psmallmatrix} : (x,y) \in \FF_p^2 \backslash (0,0) \rangle.
 \]
 
 If $\langle \beta \rangle= \FF_{p^2}^\times$, then we can write down the generators of $C_{ns}^+(p)$.
 
 \begin{example}
 Let $p = 13, \epsilon = \sqrt{7}, \FF_{p^2}^\times = \langle 1 + \epsilon \rangle$. Then 
 
 \[
 C_{ns}^+(13) = \langle \begin{psmallmatrix} 1 & 7 \cdot 1 \\ 1 & 1 \end{psmallmatrix}, \begin{psmallmatrix} 1 & 0 \\ 0 & -1 \end{psmallmatrix} \rangle.
 \]
 \end{example}
 
 The modular curve corresponding to the normaliser of nonsplit Cartan subgroup $C_{ns}^+(p)$ is defined as the compactification of the quotient of the upper half plane by the lift of $C_{ns}^+(p)$ to a subgroup $\Gamma_{ns}^+(p) \leq \SL_2(\ZZ)$.
 
 Finding a basis of $\mathcal{S}_2(\Gamma_{ns}^+(p))$ can be done following Zywina's $\Magma$ implementation as in Section \ref{basis:zyinwa}. For the purpose of exposition, suppose $\mathcal{S}_2(\Gamma_{ns}^+(p)) = \{ f_1, \ldots, f_g \}$.
 
 To find the upper half plane representatives of the expected rational points, we follow a similar procedure for $X_0(N)$. First, in the list of class number one discriminants $\mathcal{D}$, the expected points correspond to the discriminants $\Delta$ such that $p$ is inert in the corresponding order $\mathcal{O}_\Delta$ \cite{Mazur77}. Once we have the list of expected points $\{ P_1, \ldots P_r \}$, one could use the same method of inverting the $j$-invariant function to find $\SL_2(\ZZ)$-orbits $\{ \tau_1, \ldots, \tau_r \}$. The cosets of $\Gamma_{ns}^+(p) \backslash \SL_2(\ZZ)$ allow us to find the correct upper half plane representatives corresponding to $\{ P_1, \ldots P_r \}$. The problem now reduces to computing $\Gamma_{ns}^+(p) \backslash \SL_2(\ZZ)$. There is bijection:
 
 \begin{align*}
    \SL_2(\ZZ)/\Gamma_{ns}^+(p) &\rightarrow \SL_2((\ZZ/p\ZZ)/C_{ns}^+(p) \cap \SL_2(\ZZ/p\ZZ)\\
      \Gamma_{ns}^+(p) \gamma &\mapsto (C_{ns}^+(p) \cap \SL_2(\ZZ/p\ZZ))\bar{\gamma}.
\end{align*}

Therefore, once we obtained coset representatives $\{ \gamma_i \}$ of $\SL_2((\ZZ/p\ZZ)/C_{ns}^+(p) \cap \SL_2(\ZZ/p\ZZ)$, we can verify if $\gamma_i \tau$ is a $\QQ$-rational point on $X$ for $\tau \in \{ \tau_1, \ldots, \tau_r\}$ by considering the canonical embedding, i.e., we can check if $(f_1(\gamma_i \cdot \tau) : \ldots : f_g(\gamma_i \cdot \tau)) \in \PP^{g-1}(\QQ)$.

On the cusp forms, there are two major steps to computing the Hecke operator: find the double coset representatives and then decompose these representatives into products on simpler matrices, for which there are algorithms to compute the slash-$k$ operators \cite{Zywina2020ComputingAO,Shurman}. The Hecke operator at the prime $\ell$ acts as a double coset operator:

\begin{align*}
[\Gamma_{ns}^+(p)\alpha\Gamma_{ns}^+(p)]_2 f = \sum f|_2\alpha_i , \hspace{5mm}
\end{align*}

where $\{ \alpha_i \}_{i = 0, \ldots, p}$ are the double coset representatives of $\Gamma_{ns}^+(p)\backslash\Gamma_{ns}^+(p)\alpha\Gamma_{ns}^+(p)$. It turns out that the representatives have the form $\alpha_i = \epsilon\epsilon'\begin{psmallmatrix} 1 & 0 \\ 0 & \ell \end{psmallmatrix}\beta$ or $\epsilon\epsilon' \beta \begin{psmallmatrix} \ell & 0 \\ 0 & 1 \end{psmallmatrix}$, where $\epsilon,\epsilon' \in \SL_2(\ZZ)$ depends on $\alpha$ and $\beta$ comes from the standard cosets of $\Gamma^0(\ell)\backslash \SL_2(\ZZ)$. The motivation for this decomposition is that Zywina's algorithm \cite{Zywina2020ComputingAO} can compute the slash-$k$ operator on determinant 1 matrices and the two matrices on the right can be resolved using techniques from \cite{Shurman}

In the first case, $f|_2 \alpha_i = f|_2 \epsilon\epsilon'\begin{psmallmatrix} 1 & 0 \\ 0 & \ell \end{psmallmatrix}\beta$ is given by Zywina's algorithm and explicit formulas found in Chapter 5, Section 2 of \cite{Shurman}. For the second case, one uses the fact that $\begin{psmallmatrix}
1 & 0\\
0 & \ell
\end{psmallmatrix}\begin{psmallmatrix}
m\ell &  n\\
N & 1
\end{psmallmatrix} = \begin{psmallmatrix}
m &  n\\
N & \ell
\end{psmallmatrix}\begin{psmallmatrix}
\ell & 0\\
0 & 1
\end{psmallmatrix}$ where $m\ell - nN = 1$. So the last coset $\alpha_\ell$ is of the form $\varepsilon\varepsilon\beta \begin{psmallmatrix}
p & 0 \\ 0 & 1
\end{psmallmatrix}$. Again, Zywina's algorithm allows us to compute the slask-$k$ operator for the first three matrices of determinant 1 while $\begin{psmallmatrix} \ell & 0 \\ 0 & 1 \end{psmallmatrix}$ acts by shifting the indices by multiples of $\ell$.

Since we already have a basis $\{f_1, \ldots, f_g \}$ of weight $2$ cusp forms on $\Gamma_{ns}^+(p)$ by Zywina's algorithm, writing $[\Gamma_{ns}^+(p)\alpha\Gamma_{ns}^+(p)]_2 f_i $ as a linear combination of the basis elements of $\mathcal{S}_2(\Gamma(p), \QQ(\zeta_p))$ would give us the Hecke matrix.

The Hecke operator on points can be computed in two ways as before. Firstly, if we have the double coset representatives, we can evaluate the points. Secondly, we could find the roots of the modular polynomial. Each approach has its (dis)advantages: we can evaluate cusp forms on explicit representatives but this will require a closer analysis of the group structure of $C_{ns}^+(p)$ and high enough complex precision; the modular polynomials give us the $j$-invariants of $\ell$-isogeneous points but they have large coefficients.

\subsection{Example: $X_{ns}^+(13)$}

We consider the cursed curve $X = X_{ns}^+(13)$ of genus $3$ \cite{cursed-curve}. Let $C_{ns}^+(13)$ be defined by choosing the quadratic non-residue to be $7$ as in the previous example, and let $\Gamma_{ns}^+(13)$ be the lift of $C_{ns}^+(13)$ in $\SL_2(\ZZ)$.

\begin{itemize}



\item \textbf{Basis of differential forms:} Using Zywina's Magma implementation \cite{Zywina2020ComputingAO} , we obtain a basis of cusp forms as follows:
\begin{align*} f_0 = &(3\zeta_{13}^{11} + \zeta_{13}^9 + 3\zeta_{13}^8 + \zeta_{13}^7 + \zeta_{13}^6 + 3\zeta_{13}^5 + \zeta_{13}^4 + 3\zeta_{13}^2 + 1)q \\ &+ (-\zeta_{13}^{10} - 2\zeta_{13}^9 - \zeta_{13}^7 - \zeta_{13}^6 - 2\zeta_{13}^4 - \zeta_{13}^3 - 2)q^2 + O(q^3)\\ f_1 = &(4\zeta_{13}^{11} + 2\zeta_{13}^9 + 5\zeta_{13}^8 + 5\zeta_{13}^5 + 2\zeta_{13}^4 + 4\zeta_{13}^2)q \\ &+
        (-3\zeta_{13}^{11} - 5\zeta_{13}^{10} - 4\zeta_{13}^9 - 4\zeta_{13}^8 - 4\zeta_{13}^7 - 
        4\zeta_{13}^6 - 4\zeta_{13}^5 - 4\zeta_{13}^4 - 5\zeta_{13}^3 - 3\zeta_{13}^2 - 2)q^2 + O(q^3) \\ f_2 =  &(\zeta_{13}^{10} - 2\zeta_{13}^7 - 2\zeta_{13}^6 + \zeta_{13}^3)q \\ &+ (-\zeta_{13}^{11} - 2\zeta_{13}^{10} - 
        2\zeta_{13}^8 - 2\zeta_{13}^5 - 2\zeta_{13}^3 - \zeta_{13}^2 + 2)q^2 + O(q^3), \end{align*} where $\zeta_{13}$ is a $13$-th primitive root of unity and $q = e^{\frac{2\pi i\tau}{13}}$.
        
\item \textbf{Curve data:} The method of canonical embedding \cite{Galbraith_1996} gives us the following model:
        
        \begin{align*}\label{eq:cursed_curve}
\begin{split}
    &X^4 - \frac{7}{12}X^3Y - \frac{37}{30}X^2Y^2 + \frac{37}{30}XY^3 - \frac{3}{10}Y^4 - \frac{61}{60}X^3Z + \frac{41}{15}X^2YZ  \\
    &- \frac{103}{60}XY^2Z+ \frac{19}{60}Y^3Z - \frac{23}{6}X^2Z^2 + \frac{87}{20}XYZ^2 - \frac{14}{15}Y^2Z^2 - \frac{199}{60}XZ^3 \\
    &+ \frac{97}{60}YZ^3 - \frac{11}{15}Z^4 = 0,
\end{split}
\end{align*}
        
where $X,\,Y$ and $Z$ corresponds to $f_0,\,f_1$ and $f_2$ respectively. The rational points can be found by a box search: \[\{ (\frac{3}{5}:2:1), (-2:2:1), (-2:\frac{-9}{2}:1), (-2: \frac{-7}{3}:1), (\frac{7}{3}:2:1), (\frac{5}{4}: 2:1), (11: \frac{43}{2}:1) \}\].

\item \textbf{Uniformisers:} $\mathcal{S}_2(\Gamma_{ns}^+(13)) \subseteq \mathcal{S}_2(\Gamma(N), \QQ(\zeta_N))$ so the $j$-function is still a modular function for the normaliser of nonsplit Cartan and therefore can be used as an uniformiser.

\item \textbf{Rational points: } Among the class number one discriminants $\Delta$ in $\mathcal{D}$, we find $\Delta$ such that $13$ is inert in the corresponding order $\mathcal{O}_\Delta$. The set $\{ -7,-8,-11,-19,-28,-67,-163 \}$ contains discriminants that give rise to 7 expected rational points on $X$. We pick $Q$ to be the point that corresponds to discriminant $-7$, and $R$ to be the point that corresponds to discriminant $-11$. Following the notations in previous section, we have $\tau_7 = \frac{1}{2} + \frac{1}{2}\sqrt{-7}$ and $\tau_{11} = \frac{1}{2} + \frac{1}{2}\sqrt{-11}$. We then compute the coset representatives of $\SL_2(\ZZ)/\Gamma_{ns}^+(13)$,
\[\{g_0,\ldots,g_{77}\} = \{T^i,\, (T^2)ST^i,\, (T^3)ST^i,\,(T^4)ST^i,\,(T^5)ST^i,\,(T^{12})ST^i \mbox{ for } i = 0,\ldots,12\},\] where $T =\begin{psmallmatrix}
1 & 1\\
0 & 1
\end{psmallmatrix} ,\,S=\begin{psmallmatrix}
0 & -1\\
1 & 0
\end{psmallmatrix}$ are the two generators of $\SL_2(\ZZ)$. By evaluating $f_0,\,f_1,\,f_2$ at $g_i(\tau_7)$ and $g_i(\tau_{11})$ for $i=0,\ldots,77$, we obtain the correct $\Gamma_{ns}^+(13)$-orbit representatives for $Q$ and $R$, $\tau_Q = \frac{4 + 2\sqrt{-7}}{3 + \sqrt{-7}}, \tau_R = \frac{13 + \sqrt{-11}}{2}$. As in the previous section, the correct representative for $Q$ can be found by evaluating $\frac{f_0(g_i(\tau_7))}{f_2(g_i(\tau_7))}$ and $\frac{f_1(g_i(\tau_7))}{f_2(g_i(\tau_7))}$ for different coset representatives $g_i$ so that the ratios are rational numbers. Applying the same method to all the 7 discriminants, we get their corresponding rational points as computed from the model above.


\item \textbf{Hecke action on forms: }  We choose $p=11$. Let $\alpha = \begin{psmallmatrix}
-13 & 44 \\ 42 & -143 
\end{psmallmatrix} \begin{psmallmatrix}
1 & 0 \\ 0 & 11
\end{psmallmatrix}$ be the element $\alpha \in M_2(\ZZ)$ with $\det(\alpha) = 11$, $\alpha \pmod{13} \in C_{ns}^+(13).$ To find the double coset representatives we start with finding the coset representatives for $\mathcal{S} := (\alpha^{-1} \slz \alpha \cap \slz) \backslash \slz = \Gamma^0(11) \backslash \slz$. For each $\beta \in \mathcal{S}$, we found a corresponding $\gamma \in \Gamma^0(11)$ such that the representative $\beta' = \gamma \beta \in \Gamma_{ns}^+(13)$. We define the set of coset representatives to be $\mathcal{S}' := (\alpha^{-1} \Gamma_{ns}^+(13) \alpha \cap \Gamma_{ns}^+ (13)) \backslash \Gamma_{ns}^+(13)$ and the set of corresponding $\gamma$'s to be $\Gamma$: \begin{align*} 
\mathcal{S} &= \{ \begin{psmallmatrix}
1 & i \\ 0 & 1
\end{psmallmatrix} , i = 0,1, \ldots, 10 \} \cup \{ \begin{psmallmatrix}
66 & 5 \\ 13 & 1
\end{psmallmatrix}\}, \\
\Gamma &= \{ \begin{psmallmatrix}
1 & 0 \\ 0 & 1
\end{psmallmatrix},
\begin{psmallmatrix}
1 & 0 \\ -2 & 1
\end{psmallmatrix}, \begin{psmallmatrix}
1 & 11 \\ 0 & 1
\end{psmallmatrix}, \begin{psmallmatrix}
1 & -55 \\ 0 & 1
\end{psmallmatrix},\begin{psmallmatrix}
1 & 22 \\ 0 & 1
\end{psmallmatrix},\begin{psmallmatrix}
1 & -44 \\ 0 & 1
\end{psmallmatrix},\begin{psmallmatrix}
1 & 33 \\ 0 & 1
\end{psmallmatrix},\begin{psmallmatrix}
1 & -33 \\ 0 & 1
\end{psmallmatrix},\begin{psmallmatrix}
1 & 44 \\ 0 & 1
\end{psmallmatrix},\\ &\begin{psmallmatrix}
1 & -22 \\ 0 & 1
\end{psmallmatrix},\begin{psmallmatrix}
-1 & -55 \\ 0 & -1
\end{psmallmatrix}, 
\begin{psmallmatrix}
1 & -44 \\ 0 & 1
\end{psmallmatrix}\}, \\
\mathcal{S}' &= \{\begin{psmallmatrix}
1 & 0 \\ 0 & 1
\end{psmallmatrix},\begin{psmallmatrix}
1 & 1 \\ -2 & 1
\end{psmallmatrix}, \begin{psmallmatrix}
1 & 13 \\ 0 & 1
\end{psmallmatrix},\begin{psmallmatrix}
1 & -52 \\ 0 & 1
\end{psmallmatrix},\begin{psmallmatrix}
1 & 26 \\ 0 & 1
\end{psmallmatrix},\begin{psmallmatrix}
1 & -39 \\ 0 & 1
\end{psmallmatrix},\begin{psmallmatrix}
1 & 39 \\ 0 & 1
\end{psmallmatrix}, \begin{psmallmatrix}
1 & -26 \\ 0 & 1
\end{psmallmatrix},\begin{psmallmatrix}
1 & 52 \\ 0 & 1
\end{psmallmatrix},\\ & \begin{psmallmatrix}
1 & -13 \\ 0 & 1
\end{psmallmatrix},\begin{psmallmatrix}
-1 & -65 \\ 0 & -1
\end{psmallmatrix},\begin{psmallmatrix}
-506 & -39 \\ 13 & 1
\end{psmallmatrix}\}.
\end{align*}

From the bijection \begin{align*} \Gamma_{ns}^+(13)\backslash \Gamma_{ns}^+(13) \alpha \Gamma_{ns}^+(13) &\rightarrow (\alpha^{-1} \Gamma_{ns}^+(13) \alpha \cap \Gamma_{ns}^+(13)) \backslash \Gamma_{ns}^+(13) \\ \Gamma_{ns}^+(13) \delta &\mapsto (\alpha^{-1} \Gamma_{ns}^+(13) \alpha \cap \Gamma_{ns}^+(13)) \alpha^{-1} \delta \end{align*} we can get the double coset representatives of $\Gamma_{ns}^+(13)\backslash \Gamma_{ns}^+(13) \alpha \Gamma_{ns}^+(13)$: \begin{align*}
\mathcal{S}_\alpha &= \{
\begin{psmallmatrix}
-13 & 4 \\ 42 & -13
\end{psmallmatrix}
\begin{psmallmatrix}
1 & 0 \\ 0 & 11
\end{psmallmatrix}
\begin{psmallmatrix}
1 & 0 \\ 0 & 1
\end{psmallmatrix}
\begin{psmallmatrix}
1 & 0 \\ 0 & 1
\end{psmallmatrix},
\begin{psmallmatrix}
-13 & 4 \\ 42 & -13
\end{psmallmatrix}
\begin{psmallmatrix}
1 & 0 \\ 0 & 11
\end{psmallmatrix}
\begin{psmallmatrix}
1 & 0 \\ -2 & 1
\end{psmallmatrix}
\begin{psmallmatrix}
1 & 1 \\ 0 & 1
\end{psmallmatrix},
\ldots, \\ & \begin{psmallmatrix}
-13 & 4 \\ 42 & -13
\end{psmallmatrix}
\begin{psmallmatrix}
1 & 0 \\ 0 & 11
\end{psmallmatrix}
\begin{psmallmatrix}
-1 & -55 \\ 0 & -1
\end{psmallmatrix}
\begin{psmallmatrix}
1 & 10 \\ 0 & 1
\end{psmallmatrix},
\begin{psmallmatrix}
-13 & 4 \\ 42 & -13
\end{psmallmatrix}
\begin{psmallmatrix}
1 & 0 \\ 0 & 11
\end{psmallmatrix}
\begin{psmallmatrix}
1 & -44 \\ 0 & 1
\end{psmallmatrix}
\begin{psmallmatrix}
66 & 5 \\ 13 & 1
\end{psmallmatrix}
\} \\ 
&= \{ \begin{psmallmatrix}
-13 & 4 \\ 42 & -13
\end{psmallmatrix}
\begin{psmallmatrix}
1 & 0 \\ 0 & 1
\end{psmallmatrix}
\begin{psmallmatrix}
1 & 0 \\ 0 & 11
\end{psmallmatrix}
\begin{psmallmatrix}
1 & 0 \\ 0 & 1
\end{psmallmatrix},
\begin{psmallmatrix}
-13 & 4 \\ 42 & -13
\end{psmallmatrix}
\begin{psmallmatrix}
1 & 0 \\ -22 & 1
\end{psmallmatrix}
\begin{psmallmatrix}
1 & 0 \\ 0 & 11
\end{psmallmatrix}
\begin{psmallmatrix}
1 & 1 \\ 0 & 1
\end{psmallmatrix}, \ldots, \\ 
& \begin{psmallmatrix}
-13 & 4 \\ 42 & -13
\end{psmallmatrix}
\begin{psmallmatrix}
-1 & -5 \\ 0 & -1
\end{psmallmatrix}
\begin{psmallmatrix}
1 & 0 \\ 0 & 11
\end{psmallmatrix}
\begin{psmallmatrix}
1 & 10 \\ 0 & 1
\end{psmallmatrix}, \\ &
\begin{psmallmatrix}
-13 & 4 \\ 42 & -13
\end{psmallmatrix}
\begin{psmallmatrix}
1 & -4 \\ 0 & 1
\end{psmallmatrix}
\begin{psmallmatrix}
1 & 0 \\ 0 & 11
\end{psmallmatrix}
\begin{psmallmatrix}
66 & 5 \\ 13 & 1
\end{psmallmatrix} = \begin{psmallmatrix}
-13 & 4 \\ 42 & -13
\end{psmallmatrix}
\begin{psmallmatrix}
1 & -4 \\ 0 & 1
\end{psmallmatrix}
\begin{psmallmatrix}
6 &  5 \\ 13 & 11
\end{psmallmatrix}
\begin{psmallmatrix}
11 & 0 \\ 0 & 1
\end{psmallmatrix}
\}.
\end{align*}

Following the discussion in the previous section, the Hecke matrix is $A = \begin{psmallmatrix}
0 & -1 & 2 \\ 4 & -4 & 3 \\ -1 & 1 & 4
\end{psmallmatrix}$ in our fundamental equation $$((p+1)I - A) (\int_Q^R \omega_i) = (\sum_j \int_Q^{Q_j} \omega_i - \sum_j \int_R^{R_j} \omega_i).$$

\item \textbf{Algorithm \ref{alg:tiny_integral} and results:} In Step 1 of Algorithm \ref{alg:tiny_integral}, linear algebra over $\CC$ gives a power series expansion of the differential form $\omega_0$ at $j=j(Q)$:\begin{align*}
    \omega_0 &= \frac{1}{3^4 \cdot 5^2 \cdot 13
} +  \frac{23}{3^{10} \cdot 5^5 \cdot 13}(j-j(Q)) +  \frac{4}{3^{13} \cdot 5^7 \cdot 13}(j-j(Q))^2 \\ &+ \frac{437174}{3^{22} \cdot 5^{10} \cdot 13^3
}(j-j(Q))^3 + \frac{138504533 }{3^{28} \cdot 5^{13} \cdot 13^4
}(j-j(Q))^4 + O((j-j(Q))^5) \ \ d(j-j(Q)).
\end{align*}

The Hecke images can be found by computing the roots of the modular polynomial
equation $\Phi_{11}(j(Q),x) = 0$. Next, we compute the integrals as in Step 3. We record our results in the following table.

    \begin{center}
    \textbf{Table 3.3}: Coleman Integrations on $X_{ns}^+(13))$
    \end{center}

\begin{table}[h]
\label{table:X_{ns}_13}

    \centering
    \begin{tabular}{|l|l|}
    
    \hline
    \rule{0pt}{4ex}    
        $\sum_{i=0}^{11}\int^Q_{Q_i} \omega_0 $    & $10\cdot 11^{-1} + 9 + 9\cdot 11  + 6 \cdot 11^2 + 7\cdot 11^3 + 9\cdot 11^4 + O(11^5)$ 
            \rule{0pt}{4ex} \\
    
    \hline 
    \rule{0pt}{4ex}
    $\sum_{i=0}^{11}\int^Q_{Q_i} \omega_1 $ & $8\cdot 11^{-1} + 7 + 7\cdot 11 + 2 \cdot 11^2 + 6\cdot 11^3 + 6\cdot 11^4 + O(11^5)$
\\\hline

    \rule{0pt}{4ex}
    $\sum_{i=0}^{11}\int^Q_{Q_i} \omega_2 $ & $10\cdot 11^{-1} + 8 + 8\cdot 11 + 11^2  + 9\cdot 11^4 + O(11^5)$
\\\hline

    \rule{0pt}{4ex}
    $\sum_{i=0}^{11}\int^R_{R_i} \omega_0 $ & $7\cdot 11^{-1} + 2 + 3\cdot 11 + 9 \cdot 11^2  + 3\cdot 11^3 + 5\cdot 11^4 + O(11^5) $
\\\hline

    \rule{0pt}{4ex}
    $\sum_{i=0}^{11}\int^R_{R_i} \omega_1 $ & $6 + 6\cdot 11 + 11^3 + 5\cdot 11^4 + O(11^5)$ 
\\\hline

    \rule{0pt}{4ex}
    $\sum_{i=0}^{11}\int^R_{R_i} \omega_2 $ & $7\cdot 11^{-1} + 4 + 11 + 10 \cdot 11^2 +  10\cdot 11^3 + 5\cdot 11^4 + O(11^5)$ 
\\\hline
    \end{tabular}

    \label{table:X_{ns}^+(13)_results}
\end{table}

\end{itemize}
        