\section{$X_0(N)$}\label{sec:X_0_N}

Let $N$ be positive integer. The modular curve $X = X_0(N)$ is defined to be the quotient of the upper half plane by the congruence subgroup $\Gamma_0(N) = \{ \begin{psmallmatrix} \ast & \ast \\ 0 & \ast \end{psmallmatrix} \pmod{N} \} \leq \SL_2(\ZZ)$. As a moduli space, the $\QQ$-rational points of $X$ correspond elliptic curves $E$ defined over $\QQ$ such that $E$ admits a $\QQ$-rational isogeny of degree $N$ to another elliptic curve $E'$. Given a point $Q$ on $X$, to find the coset representative on the upper half plane, one first finds the ratio of periods $\tau' \in \HH$ of the elliptic curve $Q$, which corresponds to $\SL_2(\ZZ) \tau' \in \SL_2(\ZZ)\backslash \HH$ satisfying $j(\tau') = j(E)$. This can be done by finding an elliptic curve $E_{\tau}/\CC$ with $j$-invariant $j_E$ via the universal elliptic curve: \[
y^2 + xy = x^3 - \frac{36}{j_E -1728}x - \frac{1}{j_E - 1728}
\]

Then, one iterates through the cosets of $\SL_2(\ZZ)/\Gamma_0(N)$ to find $\gamma$ such that its $j$-invariant satisfies:

\[
j(\gamma \tau') = j( N\gamma \tau') = j(E)
\]

As a result, the point $Q$ corresponds to $\Gamma_0(N) \gamma \tau' \in \Gamma_0(N) \backslash \HH^+$. One could find a basis of weight $2$ cusp forms $\mathcal{S}_2(\Gamma_0(N))$ and the action of Hecke operators on the basis of cusp forms using well known methods that are implemented in \SageMath \cite{steinmf,sagemath}. Let $\omega \in H^0(X,\Omega^1)$, $Q \in X(\QQ)$. We follow Algorithm \ref{alg:tiny_integral} for computing $\sum_{i=0}^{p}\int^Q_{Q_i} \omega$. We choose $j(\tau)-j(Q)$ as our uniformiser.

\subsection{Example: $X_0(37)$}

\begin{itemize}
\item \textbf{Curve data: } We consider the modular curve $X = X_0(37)$. $X$ is a hyperelliptic curve. Comparing relations between $q$-expansions of rational functions $x,\,y \in \CC(X)$, we obtain a plane model $y^2 = - x^6 - 9x^4 - 11x^2 + 37$. It known that there are four $\QQ$-rational points $Q = (1,-4),\,R = (-1,-4),\, S=(1,4),\,T = (-1,4)$, where $Q,R$ are noncuspidal rational points and $S,T$ are cuspidal rational points \cite{maswd}.

\item \textbf{Rational points:}  Since the $j$-function is a
modular function on $X_0(37)$ and that $X_0(37)$ is hyperelliptic, we could
express $j$-function as a rational function of $x$ and $y$ to compute
that

\begin{align*}
  j(Q) &= -9317 = - 7 \cdot 11^3, \\
  j(R) &= -162677523113838677= - 7 \cdot 137^3 \cdot 2083^3.
\end{align*}
          

The points $Q,R$ corresponds to the elliptic curve $E_Q,E_R$ with
$j$-invariants $j(Q),j(R)$ containing a cyclic subgroup of order $37$ (or equivalently, with a degree $37$-isogeny). This information could be verified in LMFDB \cite{lmfdb}. Following the method in Section \ref{sec:X_0_N}, we obtain the upper half plane representatives of $Q,R$ as follows: \begin{align*} \tau_Q &\approx 0.5 + 0.17047019819380\cdot i \in \HH , \\ \tau_R &\approx 0.5 + 0.39635999889406\cdot i \in \HH. \end{align*} 


\item \textbf{Basis of differential forms:} One could compute that $\mathcal{S}_2(\Gamma_0(N))$ has $\CC$-dimension $2$ and a basis of the space of weight $2$ cusp forms using $\SageMath$. Furthermore, the action of Hecke operators on the basis of cusp forms is available on $\SageMath$. Linear algebra yields an eigenbasis $\{f_0,\,f_1\}$ of the $\CC$-vector space $\mathcal{S}_2(\Gamma_0(37))$ with the following $q$-expansions:
\begin{align*} f_0 &= q+q^3-2q^4+O(q^6), \\ f_1 &= q-2q^2-3q^3+2q^4-2q^5+O(q^6) .\end{align*}

\item \textbf{Hecke action:} We choose $p=3$, and $T_3(f_0) = f_0,\,T_3(f_1) = -3f_1$. Therefore the Hecke operator matrix $T_3$ is $\begin{psmallmatrix}
1&0\\
0&-3
\end{psmallmatrix}$. Furthermore, we find $j(Q_i), j(R_i)$ as roots of the modular polynomials $\Phi_3(j(Q),X) = 0$, $\Phi_3(j(R),X) = 0$ where $\Phi_3(X,Y)$ is the modular polynomial of level $3$. 

\item \textbf{Algorithm \ref{alg:tiny_integral} and results:} Let $\omega_0,\omega_1$ be 1-forms that corresponds to cusp forms $-\frac{1}{2}f_0,\,-\frac{1}{2}f_1$ respectively in order to obtain $\omega_0 = \frac{dx}{y}$ and $\omega_1 = \frac{xdx}{y}$. This way, we can get a direct comparison with $\Magma$'s implementation of computing Coleman integrals on hyperelliptic curves. Now, we proceed with computing the Coleman integrals on $\omega_0,\,\omega_1$. 


%We start by fixing various numerical precisions, i.e., complex number precision, number of terms in the $q$-expansion, and $p$-adic precision. 


%Our point $Q$ corresponds to $\tau \approx 0.500000000000000 + 0.170470198193806 \cdot i \in \mathcal{H}$.  
By comparing complex coefficients and using $\texttt{algdep}$ to algebraically approximate complex numbers, we obtain rational coefficients $x_i$ in the expansion of $\omega_1$ about $j=j(Q)$:
\begin{align*} \omega_1  = &(-9317) + \frac{717409}{2 \cdot 37 \cdot 47}(j-j(Q))
                             + \frac{253086749261192}{37^2 \cdot 47^3}(j-j(Q))^2
  \\ &+ \frac{176804544077038351043955}{37^3 \cdot 47^5}(j-j(Q))^3 +
       O((j-j(Q))^4) \ \ d(j-j(Q)). \end{align*}
     After that, we substitute the roots into a sum of local power series: 
\begin{align*}
    \sum_{i=0}^3 \int_{Q_i}^Q \omega_1 = &\sum_{i=0}^3 \int_{j(Q_i)-j(Q)}^{0} (-9317) + \frac{717409}{2 \cdot 37 \cdot 47}t + \frac{253086749261192}{37^2 \cdot 47^3}t^2 \\&+ \frac{176804544077038351043955}{37^3 \cdot 47^5}t^3 +  \cdots  dt 
\end{align*}


Our results are listed in the table below. One can verify the results by comparing with the hyperelliptic model of this curve or with the $\Magma$ hyperelliptic curves implementation from \cite{balatuit}.
\end{itemize}

\begin{center}
    \textbf{Table 3.1}: Coleman integration on $X_0(37)$
    \end{center}
\begin{table}[htb]\label{table:X_0_37_results}
\centering

    \begin{tabular}{|l|l|}
    \hline
    \rule{0pt}{4ex}    
        $\sum_{i=0}^{3}\int^Q_{Q_i} \omega_0 $   &$O(3^{14}) $
            \rule{0pt}{4ex} \\
\hline
            \rule{0pt}{4ex}
        $\sum_{i=0}^{3}\int^Q_{Q_i} \omega_1 $  & $3^{2} + 3^{3} + 3^{9} + 3^{10} + 2\cdot 3^{11} +  3^{12} + 2\cdot 3^{13}+ O(3^{14})$
            \rule{0pt}{4ex}
\\\hline
            \rule{0pt}{4ex}
       $\sum_{i=0}^{3}\int^R_{R_i} \omega_0 $  &$O(3^{14}) $    \rule{0pt}{4ex}    
\\\hline
           \rule{0pt}{4ex}    
        $\sum_{i=0}^{3}\int^R_{R_i} \omega_1 $ &$3^{2} + 3^{3} + 3^{9} + 3^{10} + 2\cdot 3^{11} +  3^{12} + 2\cdot 3^{13}+ O(3^{14})$      \rule{0pt}{4ex}    
\\\hline            
    \end{tabular}
\end{table}

