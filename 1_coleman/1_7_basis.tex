\section{Computing a basis of cusp forms}\label{basis:zyinwa}

The spaces of cusp forms for the congruence subgroups $\Gamma(N), \Gamma_1(N)$ and $\Gamma_0(N)$ are available in software packages \cite{sagemath} and \cite{magma}. For $H \leq GL_2(\ZZ/N\ZZ)$ satisfying the conditions above, $\mathcal{S}_2(\Gamma(N))^H$, the space of weight $2$ cusp forms invariant under $H$, is isomorphic to $\mathcal{S}_2(\Gamma_H)$ and therefore isomorphic to $H^0(X_H,\Omega^1)$.

The problem of computing a basis of cusp form reduces to computing the action of $H \leq \GL_2(\ZZ/N\ZZ)$ on $\mathcal{S}_2(\Gamma(N))$.We follow \cite{Zywina2020ComputingAO,Brunault2020} to compute the (well-defined) action. Note that $\SL_2(\ZZ)$ is freely generated by the two matrices $S = \begin{psmallmatrix} 0 & -1 \\ 1 & 0 \end{psmallmatrix}$ and $T = \begin{psmallmatrix} 1 & 1 \\ 0 & 1 \end{psmallmatrix}$. Since cusp forms of $\mathcal{S}_2(\Gamma(N))$ have $q_N$-expansions, where $q_N = e^{2\pi i /N}$, the slash-$k$ operator by $T$ introduces a factor $\zeta_N^n$ for the $n$-th Fourier coefficient. On the other hand, the action by $S$ is given by a linear combination of the basis of cusp forms on $\Gamma(N)$ where the coefficients depend on a certain Atkin-Lehner operator $W_N$. Since $\GL_2(\ZZ/N\ZZ)/\SL_2(\ZZ/N\ZZ) \xrightarrow{\cong} (\ZZ/N\ZZ)^\times$, we have:

\begin{itemize}
    \item There is an action of $\SL_2(\ZZ/N\ZZ)$ induced from $\SL_2(\ZZ)$ on the cusp forms,
    \item $\begin{psmallmatrix} 1 & 0 \\ 0 & d \end{psmallmatrix}$ acts on the coefficients of the $q_N$-expansion by $\zeta_N \mapsto \zeta_N^d$, where $\zeta_N$ is a $N$-th root of unity. 
\end{itemize}

For congruence subgroups $\Gamma_0^+(N) := \Gamma_0(N)/w_N$ with an Atkin-Lehner involution, we modify Zywina's Magma implementation to compute our examples.

\begin{remark}
In general, the map $H^0(X_H, \Omega_{X_H}^{\otimes k}) \rightarrow \mathcal{S}_{2k}(\Gamma(N),\QQ(\zeta_N))^H$ is injective \cite{Zywina2020ComputingAO}. It is an isomorphism when $k = 1$, which is what we use here.
\end{remark}